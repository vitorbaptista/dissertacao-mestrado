In Brazil, there are tools for monitoring the behaviour of legislators in
roll-call votes, such as O Estado de São Paulo's Basômetro and Radar
Parlamentar. These tools are used both by journalists and political scientists
for analysis.
 Although they are great analysis tools, their usefulness for monitoring is
limited because they require a manual follow-up, which makes it a lot of work
when we consider the volume of data. Only in the House of Representatives, 513
parliamentarians participate on average over than 400 roll-call votes by
legislature. It is possible to decrease the amount of data analysing the
parties as a whole, but in contrast we lose the ability to detect individuals'
drives or intra-party groups such as countertops.
 In order to mitigate this problem, I developed in this work a statistical
model that predicts when a parliamentary will enter or leave the governmental
coalition through ideal points estimates using the W-NOMINATE. It can be used
individually or integrated to tools such as Basômetro, providing a filter for
researchers find parliamentarians who most significantly changed their
behaviour.
 The study universe is composed of parliamentarians from the Chamber of
Deputies from the 50th to the 54th legislatures, starting in the first term of
Fernando Henrique Cardoso in 1995 until the end of the first term of Dilma
Rousseff in 2015.

\textbf{Keywords:} Legislative Analysis, Political Science, Data Science,
Predictive Models, Machine Learning.
