\chapter{Conclusão}\label{cap:conclusao}

\section{Limitações}
\label{cap:conclusao:limitacoes}

Na seção \ref{cap:miolo:estimando-mudanca-comportamento}, descrevi a metodologia usada
para analisar mudanças de comportamento baseado na de \citeonline{Poole2005}
descrita na seção \ref{cap:fundamentacao:comparando-pontos-ideais-no-tempo}.
Nela, levantei três limitações deste trabalho: a definição da duração dos
períodos de análise, o número de repetições da estimativa dos pontos ideais para
calcular seu erro, e a definição do ``raio de influência'' de uma mudança de
posicionamento. Nesta seção falarei sobre cada uma delas.

\subsection{Definição dos períodos de análise}

Na seção \ref{cap:miolo:periodos-de-analise}, defini os períodos de análise
como sendo de 12 meses divididos em dois subperíodos de mesma duração.

\subsubsection*{Duração dos períodos de análise}

Como o objetivo final deste trabalho não é análise, mas sim o desenvolvimento
de um modelo preditivo, precisei definir um conjunto de períodos de análise,
cujo tamanho foi definido arbitrariamente como um ano dividido em dois momentos
de seis meses. Uma forma possivelmente melhor seria analisar os dados para
encontrar o menor período que contenha o número de votações não-unânimes mínimo
definidos para o cálculo dos pontos ideais (neste trabalho, 20 votações com no
máximo 97,5\% votos para a maioria).

\subsubsection*{Raio de influência da mudança de posicionamento}

Uma vez definida a duração do período de análise, pode-se definir o raio de
influência buscando a maior mudança de comportamento entre os momentos dentro
do período. Por exemplo, digamos que o período definido seja de 12 meses
divididos em dois momentos de 6 meses cada. Poderíamos estimar os pontos ideais
nessa janela de tempo, buscando em que momento ocorre a maior alteração.
