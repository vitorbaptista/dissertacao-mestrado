

\chapter{A saída do PSB da coalizão em Dilma I}\label{cap:analise-saida-psb}

\section{Introdução}

O PSB esteve na base de apoio do governo do PT desde que o partido conquistou a
Presidência da República com Lula em 2003. Dez anos depois, em 2013, penúltimo
ano do primeiro governo de Dilma, o PSB anuncia a candidatura de Eduardo Campos
para a presidência, rompendo a aliança com o PT.

Para entender a repercussão dessa mudança na Câmara dos Deputados, analisaremos
as votações nominais ocorridas naquela legislatura (54\textordfeminine, que
durou de 2011 até 2015) seguindo a metodologia proposta por Keith Poole em
\emph{Spatial models of parliamentary voting} \cite{Poole2005}.
% Não estamos entendendo a repercussão da mudança na CD como um todo, mas sim
% só no comportamento de votação dos parlamentares.

\section{Metodologia}{\label{sec:analise-saida-psb:metodologia}

Para entender a influência da saída do PSB no comportamento de votação dos
parlamentares, precisamos de uma forma de comparar o comportamento deles antes
e depois dessa mudança. Este é um problema complicado, estudado por diversos
autores (CITATION NEEDED). Nesta análise seguimos a metodologia de
\cite{Poole2005}.

Considerando um parlamentar por vez, o substituímos por dois parlamentares
``virtuais'' na tabela de votações, um com os votos antes da mudança e outro
com os votos depois dela, e executamos o algoritmo W-NOMINATE nessa nova
tabela. Esse procedimento é repetido para cada parlamentar separadamente,
enquanto mantém todos os outros sem modificações. Ao final, teremos dois pontos
para cada parlamentar: um para antes e outro para depois da mudança. A
diferença entre esses pontos representa o nível da mudança de comportamento no
período de análise. 

Por exemplo, considere a tabela \ref{table:exemplo-mudanca-de-comportamento}
contendo 3 parlamentares e 4 votações. Para definir qual foi a mudança de
comportamento de Samara entre as votações 2 e 3, dividimos seus votos em dois
parlamentares ``virtuais'', Samara 1 e Samara 2 (ver tabela
\ref{table:exemplo-parlamentar-virtual}), e executamos o algoritmo W-NOMINATE.
Guardamos os resultados de Samara e repetimos os mesmos passos para Pedro e
depois Maria. Ao final, teremos dois pontos para cada parlamentar: um relativo
a sua posição antes, e outro depois da mudança (ver figura
\ref{fig:exemplo-mudanca-de-comportamento}). A distância desses dois pontos é
uma medida da intensidade da mudança do parlamentar.  Comparando as distâncias
de cada parlamentar, podemos determinar se a mudança é significante ou não.

% TODO: Falar sobre o parametric bootstrap
% TODO: Falar sobre os problemas em comparar esses valores.
% \begin{quote}
%   ``(...) all matrices have the same number of nonmissing entries, and they
%   differ only in the fact that each has a different senator divided into two
%   records. Consequently, I believe it's safe to capture the 72 distances with
%   each other, because the differences between the configurations will be
%   trivial.''\cite{Poole2005}
% \end{quote}

\begin{table}
  \begin{minipage}{\textwidth}
    \centering
    \begin{tabular}{ l l l l l }
      nome & votação1 & votação2 & votação3 & votação4 \\
      \hline
      Samara & Sim & Sim & Não & Não \\
      Pedro & Sim & Sim & Sim & Sim \\
      Maria & Não & Não & Não & Não \\
    \end{tabular}
    \caption{Dados originais de votação}
    \label{table:exemplo-mudanca-de-comportamento}
  \end{minipage}
  \begin{minipage}{\textwidth}
    \centering
    \begin{tabular}{ l l l l l }
      nome & votação1 & votação2 & votação3 & votação4 \\
      \hline
      Samara 1 & Sim & Sim & & \\
      Samara 2 & & & Não & Não \\
      Pedro & Sim & Sim & Sim & Sim \\
      Maria & Não & Não & Não & Não \\
    \end{tabular}
    \caption{Dados de votação com Romário dividido no meio}
    \label{table:exemplo-parlamentar-virtual}
  \end{minipage}
\end{table}

\begin{knitrout}
\definecolor{shadecolor}{rgb}{0.969, 0.969, 0.969}\color{fgcolor}\begin{figure}
\includegraphics[width=\maxwidth]{figure/exemplo-mudanca-de-comportamento-1} \caption[Posições dos parlamentares antes e depois da data em análise]{Posições dos parlamentares antes e depois da data em análise}\label{fig:exemplo-mudanca-de-comportamento}
\end{figure}


\end{knitrout}

% Pseudocódigo
%
% \begin{enumerate}
%   \item Para cada parlamentar, faça:
%   \begin{enumerate}
%     \item Crie dois parlamentares ``virtuais'', um com os votos do parlamentar
%       antes da mudança, e o outro com os votos depois dela;
%     \item Execute o W-NOMINATE na matriz completa de votações com todos os
%       parlamentares e o parlamentar em análise substituído pelos parlamentares
%       ``virtuais'';
%     \item Guarde o resultado;
%   \end{enumerate}
%   \item Compara a mudança de cada parlamentar de interesse com todos os outros,
%     para descobrir se ela (a mudança) foi significativa.
% \end{enumerate}

\section{Extração dos dados}



Os dados das votações nominais foram extraídos a partir da API disponibilizada
no site da Câmara dos Deputados. Eles compreendem 131552 votos
proferidos por 644 parlamentares em 432
votações. A figura \ref{fig:votacoes-por-mes} mostra a distribuição das
votações no período e a figura \ref{fig:deputados-por-partido} mostra o número
de deputados federais por partido. Só consideramos votos \verb|Sim| ou
\verb|Não|.  

\begin{knitrout}
\definecolor{shadecolor}{rgb}{0.969, 0.969, 0.969}\color{fgcolor}\begin{figure}
\includegraphics[width=\maxwidth]{figure/votacoes-por-mes-1} \caption[Distribuição de votações por mês na 54\textordfeminine{} legislatura]{Distribuição de votações por mês na 54\textordfeminine{} legislatura}\label{fig:votacoes-por-mes}
\end{figure}


\end{knitrout}

\begin{knitrout}
\definecolor{shadecolor}{rgb}{0.969, 0.969, 0.969}\color{fgcolor}\begin{figure}
\includegraphics[width=\maxwidth]{figure/deputados-por-partido-1} \caption[Distribuição de Deputados Federais por partido na 54\textordfeminine{} legislatura]{Distribuição de Deputados Federais por partido na 54\textordfeminine{} legislatura}\label{fig:deputados-por-partido}
\end{figure}


\end{knitrout}

\section{Análise}



Seguindo a metodologia descrita na sessão
\ref{sec:analise-saida-psb:metodologia}, dividimos a
54\textordfeminine{} legislatura na votação mais próxima da data de interesse,
no nosso caso, a data em que o PSB deixa a coalizão do governo. De acordo com
os dados disponíveis no banco de dados do CEBRAP, isso ocorreu em 03/10/2013. A
primeira votação depois dessa data ocorreu em
08/10/2013 (identificador
367), relacionada ao MPV 621/2013. Houve
303 votações antes da ruptura e
133 votações após. Dos 643
parlamentares que votaram ao menos uma vez nesse período,
470 passaram pelo nosso filtro de
participação em ao menos 20 votações cuja minoria teve 2,5\% ou mais dos votos.

A figura \ref{fig:boxplot-diff-coalizao} mostra a distribuição da diferença
entre as posições dos parlamentares antes e depois da saída do PSB da base do
governo, agrupado pelos que entraram ou saíram da coalizão e os que mantiveram
sua posição. Nela podemos ver que, no geral, quem entra na coalizão se torna
mais governista, quem sai se torna mais oposicionista, e os outros mantém sua
posição, mas existem diversos \emph{outliers}.

\begin{knitrout}
\definecolor{shadecolor}{rgb}{0.969, 0.969, 0.969}\color{fgcolor}\begin{figure}
\includegraphics[width=\maxwidth]{figure/boxplot-diff-coalizao-1} \caption[Distribuição do nível de mudança da posição dos deputados federais antes e depois da saída do PSB da coalizão do governo]{Distribuição do nível de mudança da posição dos deputados federais antes e depois da saída do PSB da coalizão do governo}\label{fig:boxplot-diff-coalizao}
\end{figure}


\end{knitrout}

Na tabela \ref{table:top-10-diffs} estão os dez parlamentares com maior mudança
de posição em valores absolutos. Desses, analisemos os que não entraram nem
saíram da coalizão: 

\begin{description}
\item[Dudimar Paxiuba] Apesar de continuar fora da coalizão, ele saiu do PSDB
para o PROS em 2013, o que explicaria essa aproximação ao
governo\footnote{\url{http://www2.camara.leg.br/deputados/pesquisa/layouts_deputados_biografia?pk=193033&tipo=0}}.
\item[Rebecca Garcia] Não consegui explicar. A única coisa que encontrei foi
que ela ia se candidatar a prefeitura de Manaus em 2012, mas desistiu porque
vazou um áudio de uma conversa dela com seu amante na época, o ex-vereador Ari
Moutinho. Não me parece ter relação.
\item[Vicente Candido]Também não encontrei uma razão.
\item[Dr. Grilo]Saiu do PSL para o Solidariedade, que pode explicar o
afastamento do governo.
\end{description}

\begin{table}
\centering
\begin{knitrout}
\definecolor{shadecolor}{rgb}{0.969, 0.969, 0.969}\color{fgcolor}
\begin{tabular}{l|l|l|r|r|r}
\hline
Nome & Partido & Coalizão & Antes & Depois & Diferença\\
\hline
Wladimir Costa & PMDB & saiu & 0.32 & -0.60 & -0.92\\
\hline
Dudimar Paxiuba & PROS & nao\_mudou & -0.71 & 0.17 & 0.87\\
\hline
Luiz Nishimori & PL+PRONA>PR & entrou & -0.86 & -0.08 & 0.77\\
\hline
Domingos Dutra & PT & saiu & 0.52 & -0.16 & -0.68\\
\hline
Romário & PSB & saiu & 0.28 & -0.37 & -0.65\\
\hline
Genecias Noronha & PMDB & saiu & -0.01 & -0.64 & -0.63\\
\hline
Rebecca Garcia & PDS>PP & nao\_mudou & 0.39 & -0.23 & -0.62\\
\hline
Antonio Balhmann & PROS & saiu & 0.08 & 0.69 & 0.61\\
\hline
Vicente Candido & PT & nao\_mudou & 0.99 & 0.39 & -0.60\\
\hline
Dr. Grilo & PSL & nao\_mudou & 0.05 & -0.52 & -0.57\\
\hline
\end{tabular}


\end{knitrout}
\caption{Os dez parlamentares com maior diferença absoluta entre suas posições
antes e depois da saída do PSB da base do governo.}
\label{table:top-10-diffs}
\end{table}

\subsection{Os deputados do PSB mudaram de comportamento?}



De acordo com a figura \ref{fig:mudanca-psb}, no geral os deputados do PSB se
afastaram do governo, enquanto a maioria dos parlamentares dos outros partidos
mantiveram suas posições. Há um \emph{outlier} nos deputados do PSB que, ao
contrário da maioria, se aproximou do governo: Alexandre Toledo, que
passou de \ensuremath{-0.61} para \ensuremath{-0.1} (diferença
de 0.51). Analisando este caso, a razão se torna clara: ele
era do PSDB e passou para o PSB em 2013. Apesar do PSB ter se tornado oposição,
passando de uma posição mediana \ensuremath{-0.01}
para \ensuremath{-0.33}, o PSDB é um opositor mais
ferrenho ao governo, com posição mediana \ensuremath{-0.87}.

Com base nesses dados, podemos concluir que o que já esperávamos: o PSB se
afastou do governo.

\begin{knitrout}
\definecolor{shadecolor}{rgb}{0.969, 0.969, 0.969}\color{fgcolor}\begin{figure}
\includegraphics[width=\maxwidth]{figure/mudanca-psb-1} \caption[Distribuição do nível de mudança da posição dos deputados federais antes e depois da saída do PSB da coalizão do governo]{Distribuição do nível de mudança da posição dos deputados federais antes e depois da saída do PSB da coalizão do governo}\label{fig:mudanca-psb}
\end{figure}


\end{knitrout}

\section{Modelagem}

Nosso objetivo agora é determinar se, baseado somente nas posições dos
deputados federais de antes e depois da saída do PSB da coalizão, conseguimos
inferir se um parlamentar mudou de lado (se era governo, se tornou oposição, ou
vice-versa). Para isso, treinaremos um modelo \emph{Random Forest}.

O primeiro passo é alterar a coluna ``Coalizão'' para somente dois valores: sim
ou não. Nesse momento, nosso objetivo é determinar se o parlamentar mudou de
lado, e não pra que lado ele foi\footnote{Sabendo se o parlamentar mudou de
lado e qual era sua posição anterior, podemos inferir qual é sua nova posição
(e.x.: se ele era oposição e mudou de lado, agora virou governo).}. Também
retiraremos informações como nome e partido e parlamentares cuja posição não
conseguimos estimar pois não houve votos o suficiente, e adicionamos o
desvio-padrão de cada valor\footnote{Calculados usando parametric bootstrap.
TODO: Explicar melhor}. No final, teremos uma tabela como a
\ref{table:clean-data}.

\begin{table}
\centering
\begin{knitrout}
\definecolor{shadecolor}{rgb}{0.969, 0.969, 0.969}\color{fgcolor}
\begin{tabular}{r|r|r|r|r|l}
\hline
Antes & Antes SD & Depois & Depois SD & Diferença & Mudou de posição\\
\hline
0.32 & 0.17 & -0.60 & 0.16 & -0.92 & Sim\\
\hline
-0.71 & 0.12 & 0.17 & 0.08 & 0.87 & Não\\
\hline
-0.86 & 0.10 & -0.08 & 0.19 & 0.77 & Sim\\
\hline
0.52 & 0.13 & -0.16 & 0.13 & -0.68 & Sim\\
\hline
0.28 & 0.10 & -0.37 & 0.20 & -0.65 & Sim\\
\hline
-0.01 & 0.12 & -0.64 & 0.15 & -0.63 & Sim\\
\hline
\end{tabular}


\end{knitrout}
\caption{Os dez parlamentares com maior diferença absoluta entre suas posições
antes e depois da saída do PSB da base do governo.}
\label{table:clean-data}
\end{table}

Com os dados nesse formato, agora precisamos dividí-lo em dois grupos: um para
treino e um para teste. Precisamos fazer isso para que as estimativas de
precisão do nosso modelo sejam confiáveis. Escolhemos manter 80\% para treino e
20\% para testes. Também usamos o \emph{10-fold stratified cross validation}.

\begin{knitrout}
\definecolor{shadecolor}{rgb}{0.969, 0.969, 0.969}\color{fgcolor}\begin{kframe}
\begin{verbatim}
## note: only 30 unique complexity parameters in default grid. Truncating the grid to 30 .
\end{verbatim}
\end{kframe}
\end{knitrout}

O modelo foi treinado com todos os atributos na tabela \ref{table:clean-data} e
a interação entre eles. Analisamos diversos números de preditores e, como pode
ser visto na figura \ref{fig:roc-vs-mtry}, o melhor modelo com relação ao área
sob a curva ROC usou 14 preditores e chegou a área de
0.86 com desvio-padrão de 0.06.

\begin{knitrout}
\definecolor{shadecolor}{rgb}{0.969, 0.969, 0.969}\color{fgcolor}\begin{figure}
\includegraphics[width=\maxwidth]{figure/roc-vs-mtry-1} \caption[ROC como função do número de preditores]{ROC como função do número de preditores.}\label{fig:roc-vs-mtry}
\end{figure}


\end{knitrout}

Usando os dados de teste, podemos avaliar nosso modelo. A figura \ref{fig:roc}
mostra a curva ROC. Ela mostra a sensibilidade e especificidade do nosso modelo
para diversos pontos de corte. O valor marcado é o ponto ``ideal'', definido
como o ponto mais próximo do canto superior esquerdo do gráfico (sensibilidade
e especificidade iguais a 1). Na tabela \ref{table:roc} podemos ver todos
máximos locais dos pontos de corte e as respectivas sensibilidades e
especificidades, com mínimos e máximos dentro de margem de confiança de
95\%\footnote{Calculado a partir de 2.000 stratified bootstrap replications.
TODO: explicar melhor}.

\begin{knitrout}
\definecolor{shadecolor}{rgb}{0.969, 0.969, 0.969}\color{fgcolor}\begin{figure}
\includegraphics[width=\maxwidth]{figure/roc-1} \caption[Curva ROC com ponto ``ideal'' marcado]{Curva ROC com ponto ``ideal'' marcado.}\label{fig:roc}
\end{figure}


\end{knitrout}

\begin{table}
\centering
\begin{knitrout}
\definecolor{shadecolor}{rgb}{0.969, 0.969, 0.969}\color{fgcolor}
\begin{tabular}{l|r|r|r|r|r|r}
\hline
cutpoint & sens.low & sens.median & sens.high & spec.low & spec.median & spec.high\\
\hline
0.917 & 0.00 & 0.13 & 0.33 & 0.96 & 0.99 & 1.00\\
\hline
0.877 & 0.00 & 0.20 & 0.40 & 0.94 & 0.97 & 1.00\\
\hline
0.599 & 0.07 & 0.27 & 0.53 & 0.91 & 0.96 & 1.00\\
\hline
0.429 & 0.13 & 0.33 & 0.60 & 0.81 & 0.88 & 0.95\\
\hline
0.424 & 0.20 & 0.40 & 0.67 & 0.79 & 0.87 & 0.94\\
\hline
0.273 & 0.33 & 0.60 & 0.87 & 0.74 & 0.83 & 0.91\\
\hline
0.25 & 0.53 & 0.73 & 0.93 & 0.72 & 0.81 & 0.88\\
\hline
0.212 & 0.60 & 0.80 & 1.00 & 0.67 & 0.77 & 0.86\\
\hline
0.136 & 0.67 & 0.87 & 1.00 & 0.55 & 0.67 & 0.77\\
\hline
0.108 & 0.80 & 0.93 & 1.00 & 0.51 & 0.63 & 0.72\\
\hline
0.035 & 1.00 & 1.00 & 1.00 & 0.37 & 0.47 & 0.58\\
\hline
\end{tabular}


\end{knitrout}
\caption{Especificidade e sensibilidade nos máximos locais da curva ROC}
\label{table:roc}
\end{table}
