\chapter{Miolo da sua dissertação}\label{cap:miolo}

Nessa parte pretendo falar sobre a ferramenta desenvolvida, mostrar sua arquitetura e como cada parte funciona e pode ser modificada.

\section{Componentes}

Mostrar em linhas gerais como é organizada a ferramenta, indo de um nível de abstração mais alto (como o usuário vê a ferramenta) até uma arquitetura técnica.


(Alexandre) Apesar de eu não ser muito fã acho importante apresentar algum tipo de diagrama de componentes nesta seção, para permitir uma visualização da arquitetura da solução.


\subsection{Extração dos dados}

Falar sobre as fontes de dados (câmara dos deputados, banco de dados legislativos do Cebrap, etc.), a forma de extração, passos de limpeza, banco de dados, frequência de atualização, se é pull ou push, e como disponibilizamos esses dados para as outras etapas. Ao término dessa seção, o leitor deverá entender nosso processo de data wrangling e saber como ele próprio pode, se quiser, baixar os dados ao final dessa etapa e fazer suas próprias análises. Talvez tenha problema com isso, pois o pessoal da CEBRAP pode não permitir a redistribuição de seus dados.



(Alexandre) É interessante destacar que disponibilizar estes dados em um formato mais "amigável" também é uma contribuição do seu trabalho.

\subsection{Análise}

Aqui falarei sobre os algoritmos usados e as métricas escolhidas (coesão parlamentar, número de destaques, etc.). Também é bom frisar que a arquitetura permite adicionar novos algoritmos, talvez até de uma forma reativa. Por exemplo, considerando que a acusação que o PP trancou a pauta para pressionar o Lula a nomear uma pessoa na Petrobrás, poderíamos criar um algoritmo que detectasse trancamentos de pauta por um partido qualquer, para no futuro descobrirmos movimentações parecidas.

\subsection{Notificações/Relatórios}

Falar sobre como os usuários consomem esses dados. Minha ideia atual é gerar um relatório semanal/mensal/quando-aparecer-algo-interessante e enviá-lo por email pra quem se cadastrar. Nesse caso, devo mostrar exemplos desse relatório.


(Alexandre) Seria legal considerar também a criação de uma conta que twitaria periódicamente os relatórios.


\section{Considerações Finais}

Desenvolvi uma ferramenta que extrai dados de diversas fontes, os consolida, analisa e gera relatórios para os usuários. Cada parte pode ser modificada e expandida, e os dados (e algoritmos) podem ser usados por outras pessoas. Daqui falta validar que os algoritmos na parte de análise geram algo interessante.