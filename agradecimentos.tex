Devo todas as conquistas de minha vida a minha família. Foi o seu
trabalho, amor e dedicação que me ensinaram e me permitiram fazer o que fiz. Se
todos chegamos aonde estamos por nos apoiarmos nos ombros de gigantes, foram
eles e elas os primeiros gigantes nos quais me apoiei.

Agradeço em especial a minha esposa, melhor amiga e co-orientadora não-oficial
Samara. Sem sua ajuda, esse trabalho não seria possível e minha vida seria
muito mais solitária.

Agradeço também ao meu orientador, Alexandre. Só o conheci ao me inscrever no
mestrado, mas ao longo desses anos tive certeza que não poderia ter tido mais
sorte nessa escolha. Por sua orientação técnica, mas principalmente pelo seu
interesse indiscutível nessa área de pesquisa. Em outra situação, acredito que
ele mesmo teria escrito esse trabalho, o que é prova inegável da nossa
sintonia.

Agradeço as professoras Andréa Freitas e Thaís Gaudencio, que aceitaram
participar da minha banca, emprestando seu tempo e conhecimento para a melhoria
deste trabalho.

Agradeço aos pesquisadores e funcionários do \gls{CEBRAP}, cuja contribuição à
área da Ciência Política é incalculável, no Brasil e no mundo. Em especial,
gostaria de agradecer a Andréa Freitas, ao Samuel Moura e ao Maurício Izumi, que 
me ajudaram muito a validar as ideias que discuti nesse trabalho, e ao Paulo
Hubert, que me auxiliou a acessar o banco de dados legislativos do
\gls{CEBRAP}, do qual extraí a lista de coalizões usada. Além, é claro, a
Argelina Figueiredo e o Fernando Limongi, cuja pesquisa foi um divisor de águas
no pensamento da Ciência Política brasileira.

Ao longo do tempo, o contato com pessoas interessadas na intersecção entre
computação, política e jornalismo foi abrindo meus olhos para essa nova área
que acho extremamente interessante. Isso foi possibilitado, principalmente,
pela criação do grupo Transparência Hacker por, entre tantos outros, Pedro
Markun e Daniela Silva. Através desse grupo, conheci pessoas fenomenais como os
jornalistas Daniel Bramatti, José Roberto de Toledo e Amanda Rossi que, junto
com o Diego Rabatone, formavam o Estadão Dados, onde tive o prazer de trabalhar
por uma semana durante o segundo turno das eleições de 2012.

Agradeço aos amigos criados durante a organização do Encontro de Software Livre
da Paraíba (ENSOL), em especial a Rodrigo Vieira e Anahuac de Paula Gil, os
principais responsáveis no meu amadurecimento com relação a software e cultura
livres.

Trabalhando na ThoughtWorks em Porto Alegre, fiz diversos amigos. Em especial,
Leonardo Tartari e Thiago Bueno, companheiros de vários hackathons, foram quem
despertaram em mim o interesse pela visualização de dados, que foi uma das
razões que me fizeram entrar na Open Knowledge Foundation (OKF).

A OKF é uma ONG inglesa que trabalha com dados abertos. Durante os anos que
trabalhei nela, tive oportunidade de conhecer diversas pessoas que me ajudaram
a me aprofundar nessa área, em especial o time de desenvolvimento do CKAN e os
fundadores da Open Knowledge Foundation Brasil.

Por último, mas de forma alguma menos importante, agradeço aos amigos brutais,
os irmãos e irmãs que encontrei durante a vida. Em especial ao Pedro Guimarães
que, além de amigo e parceiro em diversos projetos, se tornou meu cunhado.

À todas essas pessoas e muitas outras, dedico esse trabalho.
