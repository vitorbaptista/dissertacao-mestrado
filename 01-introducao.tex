\chapter{Introdução} \label{intro}

Neste capítulo, serão descritos o que motivou o desenvolvimento deste trabalho
(seção \ref{sec:motivacao}), juntamente com a definição do problema e objetivos
da pesquisa, a metodologia seguida, publicações relacionadas e, por fim, será
resumida a estrutura dos restante da dissertação.

\section{Motivação}\label{sec:motivacao}

O acompanhamento das atividades dos legisladores é extremamente importante,
pois são eles que alteram as regras do jogo para a vida no Brasil, afetando
todos brasileiros. Desde uma empregada doméstica, que quer entender quais são
seus novos direitos conquistados pela exigência que seus patrões assinem sua
carteira de trabalho, até a presidente de uma grande construtora, que pode
ganhar ou perder milhões de acordo com a aprovação ou não de novas leis. As grandes
empresas, com recurso para investir, reconhecem a importância de acompanhar de
perto a atividade legislativa, seja passiva ou ativamente através de
\emph{lobbying}. Infelizmente, com exceção das leis que são divulgadas na mídia,
como aumentos do salário mínimo ou, recentemente, a redução da maioridade
penal, a maioria dos cidadãos não se interessa por essa área, seja por falta de
tempo, conhecimento ou simplesmente falta de interesse.

Jornalistas políticos exercem um papel fundamental nesse sentido, traduzindo os
termos técnicos e jurídicos usados pelos parlamentares em uma forma que possa
ser mais facilmente compreendida pelo cidadão comum. Entretanto, como essa
análise demanda bastante tempo, ela acaba se restringindo aos temas mais
polêmicos, que atingem um maior número de pessoas. Esses temas são muito
importantes, mas não suficientes: se eu trabalho numa ONG de preservação do
meio ambiente, por exemplo, meu maior interesse é em projetos de lei que versem
nessa área.

Percebendo essa necessidade, empresas como o \gls{ELLO} no Brasil e a
FiscalNote nos Estados Unidos, criaram ferramentas que facilitam esse
monitoramento personalizado. Seus produtos permitem que o usuário defina suas
as áreas de interesse (por exemplo, meio ambiente ou mobilidade urbana),
recebendo resumos periódicos do que afeta essas áreas, inclusive com previsões
da probabilidade de aprovação dos projetos de lei relacionados. Apesar disso,
por serem ferramentas pagas, seu uso ainda é restrito.

% Explica importância do monitoramento do comportamento dos legisladores
Um dos fatores mais importantes no comportamento dos parlamentares é o conflito
governo/oposicao \cite{Leoni2002,Desposato2005b,Freitas2012,Izumi2013}. Dessa
forma, quando um parlamentar muda de lado, seja migrando de partido ou quando
seu próprio partido se une (ou deixa) à coalizão governamental, é de se esperar
que seu comportamento também mude. Como, no geral, os partidos são capazes de
disciplinar seus filiados a votar de certa forma, essa mudança de forças pode
definir a aprovação ou não de um projeto de lei. Dessa forma, é essencial
monitorar essas mudanças para entender as chances que um projeto tem de virar
lei.

Apesar disso, o grande volume de dados gera um desafio. Considerando somente o
nível federal, 513 deputados federais e 81 senadores participam de centenas de
votações a cada legislatura, tornando difícil o entendimento dos seus padrões
de votação. Algumas ferramentas, como o Basômetro e o Radar Parlamentar, foram
criadas para tentar diminuir esse problema \cite{Estadao2012,Trento2013}.

Elas são ferramentas gratuitas que permitem o acompanhamento da taxa de
governismo (no caso do Basômetro) ou das posições relativas dos parlamentares
(no caso do Radar Parlamentar), sendo um grande avanço para auxiliar o cidadão
comum a se aproximar do processo legislativo \cite{Dantas2014}. Entretanto,
mesmo facilitando bastante, não é fácil analisar um gráfico com 513 pontos (no
caso da Câmara) e visualizar o que está mudando ou não. Para diminuir esse
desafio, as análises acabam sendo feitas baseados no comportamento agregado dos
partidos, e não dos parlamentares individualmente.

% FIXME: Acertar os valores da coesão do PT, possivelmente usando o Basômetro.
Os partidos são capazes de influenciar o comportamento dos seus parlamentares,
mantendo taxas de coesão, em sua maioria, acima de 75\%
\cite{Figueiredo2001,Cheibub2009,Zucco2009}. Assim, a análise a nível de partido
é uma forma importantíssima de entender o comportamento parlamentar. Apesar
disso, algumas informações são perdidas ao fazer essa agregação. Para dar um
exemplo concreto, o PT é, historicamente, um dos partidos brasileiros mais
coesos, sendo comum chegar a taxas de coesão acima de 95\%. Em 2003 e 2004, os
dois primeiros anos do primeiro governo Lula, o PT mantinha uma coesão
altíssima (XX \%). Apesar disso, três deputados constantemente votavam
contrários a indicação do líder do partido. Depois de serem alertados sobre seu
comportamento, eles acabaram sendo expulsos do PT.

% FIXME: Citar os fundadores do PSOL e números de deputados do PT
Esses deputados são Babá, Luciana Genro e XXX que, juntamente com a senadora
Heloísa Helena, expulsa pela mesma razão, fundaram o PSOL. Essa movimentação
passaria desapercebida ao analisar o PT como um todo, já que só 3 dos quase 100
deputados do partido tiveram esse comportamento na época.

O objetivo deste trabalho é o desenvolvimento de um modelo estatístico que
determine a chance de um parlamentar ter mudado de posicionamento com base no
seu padrão de voto. Esse modelo foi treinado com os dados da
50\textordfeminine{} até a 54\textordfeminine{} legislaturas, compreendendo o
período de 20 anos de 1995, no início do primeiro governo de Fernando Henrique
Cardoso, até o início de 2015, no final do primeiro governo de Dilma Rousseff.

Com isso, espero criar uma ferramenta para que os cidadãos, jornalistas e
cientistas políticos consigam filtrar quais parlamentares mudaram de
comportamento mais significativamente em cada período, otimizando o uso do seu
tempo ao focar seus esforços em analisar quem está mudando. Esse modelo poderá
ser usado separadamente, ou integrado em ferramentas já existentes, como o
Basômetro ou o Radar Parlamentar, visando aumentar sua utilidade como
ferramentas de monitoramento legislativo.

Como preditores, o modelo usa os pontos ideais dos parlamentares, estimados
peloo algoritmo W-NOMINATE \cite{Poole1985,Poole2005}. Além desses pontos,
também replicamos parte da pesquisa de \citeonline{Freitas2008}, que analisa os
aspectos temporais das migrações partidárias no Brasil, focando nas mudanças de
posicionamento, sejam a partir da migração para partidos de posicionamento
oposto (indo do governo para oposição ou vice-versa), ou na entrada ou saída do
próprio partido na coalizão governamental.

O principal diferencial deste trabalho é democratizar o acesso a técnicas
antes só disponíveis em sistemas pagos de empresas como o \gls{ELLO} no Brasil
e a FiscalNote nos Estados Unidos.

\section{Problema de pesquisa}

Os partidos e as coalizões governamentais são capazes de influenciar o
comportamento dos parlamentares \cite{Figueiredo2001,Santos2003}. Partindo
disso, \citeonline{Izumi2013} mostrou que o comportamento dos senadores muda ao
entrar ou sair da coalizão, mas não sabemos se ele muda antes ou depois da
oficialização dessa mudança. A pergunta que buscamos responder neste trabalho
é:

\emph{É possível detectar a mudança de posicionamento de um deputado federal,
com ele entrando ou saindo da coalizão governamental, a partir de uma mudança
no seu padrão de votação?}

O modelo estatístico desenvolvido neste trabalho só faz sentido caso, além de
ser possível detectar mudanças de posicionamento, essa detecção ocorra antes da
oficialização da mudança. Em outras palavras, que o modelo seja capaz de
detectar uma mudança de posicionamento antes que ela seja de conhecimento
público. Para isto, precisaremos também responder a pergunta:

\emph{Os deputados federais mudam seu padrão de votação antes de mudar de
posicionamento?}

No levantamento bibliográfico feito no capítulo
\ref{cap:trabalhos-relacionados} mostraremos que, apesar de alguns autores
descobrirem que os parlamentares mudam de comportamento ao mudarem de
posicionamento, não foram encontrados trabalhos que discorram sobre se essa
mudança de comportamento ocorre antes ou depois da efetiva mudança de
posicionamento. Dessa forma, responder essa pergunta é uma outra contribuição
deste trabalho.

\section{Objetivos}

O objetivo geral desta dissertação é o desenvolvimento e validação de um modelo
capaz de determinar a chance de um deputado federal ter mudado de
posicionamento em um determinado período.

Para alcançar esse objetivo geral, foram definidos os seguintes objetivos
específicos:

\begin{enumerate}
  \item Determinar um conjunto de características a partir das quais seja
    possível determinar a chance de um parlamentar mudar de posicionamento;
  \item Descobrir se os parlamentares mudam de comportamento antes de mudarem
    de posicionamento;
  \item Analisar diversos modelos estatísticos, buscando qual tem melhor
    performance na detecção da mudança de posicionamento dos deputados
    federais brasileiros.
\end{enumerate}

\section{Metodologia}

Para o desenvolvimento desta pesquisa, foram seguidos os seguintes passos:

\begin{itemize}
  \item Levantamento bibliográfico sobre análise do comportamento parlamentar,
    análise da mudança do comportamento parlamentar, e métodos de aprendizado
    de máquina usados no âmbito da Ciência Política;
  \item Extração dos dados de votos e votações a partir da página da Câmara dos
    Deputados, e da listagem de coalizões a partir do banco de dados
    legislativo do \gls{CEBRAP};
  \item Definição da forma para representação desses dados usando a teoria
    espacial do voto;
  \item Análise dos padrões gerais e temporais dos pontos ideias estimados;
  \item Definição de variáveis independentes capazes de serem usadas para
    diferenciar parlamentares que mudaram de posicionamento dos que não
    mudaram;
  \item Análise de diversos modelos preditivos buscando o que obtem a melhor
    performance nesses dados;
  \item Validação do modelo final baseado na análise feita na etapa anterior.
\end{itemize}

\section{Publicações Relacionadas}

Mencionar o artigo que você publicou no BRASNAM

\section{Estrutura da Dissertação}

Esta dissertação é dividida em cinco capítulos, incluindo este introdutório,
que apresentou a motivação, problema de pesquisa e objetivos deste trabalho.

No capítulo \ref{cap:fundamentacao}, \nameref{cap:fundamentacao}, serão
brevemente apresentados os conceitos básicos necessários para entendimento do
restante do trabalho. Na seção \ref{cap:fundamentacao:ciencia-de-dados}, falarei
sobre a Ciência de Dados, com foco no desenvolvimento de modelos preditivos
para predição de variáveis categóricas e sua validação através de matrizes de
confusão e ferramentas como a curva \gls{ROC}.

Na seção \ref{cap:fundamentacao:teoria-espacial-do-voto}, apresentarei técnicas
baseadas na teoria espacial do voto, que permitem colocar um conjunto de
parlamentares em um plano cartesiano, com suas posições definidas a partir de
seus votos em um conjunto de votações. Existem diversas técnicas para definir
essas posições, mas neste trabalho focarei no W-NOMINATE, uma das mais usadas.

Na seção \ref{cap:fundamentacao:comparando-pontos-ideais-no-tempo}, será
apresentado o problema em comparar pontos ideais ao longo do tempo, descrevendo
técnicas que usam ``pontes'' para diferenciar mudanças causadas por diferenças
na agenda legislativa dos períodos das causadas pela mudança de comportamento
do parlamentar.

Ao final desse capítulo, no subtítulo
\ref{cap:fundamentacao:processo-legislativo}, será resumido o processo
legislativo brasileiro em nível federal com foco nas formas de votação, em
especial no critérios que determinam quando um projeto de lei será posto em
votação nominal, que são a principal fonte dos dados usados para construção do
modelo estatístico.

No capítulo \ref{cap:trabalhos-relacionados},
\nameref{cap:trabalhos-relacionados}, serão resumidos alguns trabalhos,
encontrados durante a revisão bibliográfica feita nessa pesquisa, que versam
sobre a análise do comportamento parlamentar, a da mudança de comportamento e o
uso de técnicas de aprendizagem de máquina na Ciência Política.

Explicadas, até esse momento, as ferramentas usadas no trabalho e a literatura
da área, no capítulo \ref{cap:desenvolvimento}, \nameref{cap:desenvolvimento},
será descrito o processo de criação do modelo, partindo da definição do
universo de estudo, coleta e preparação dos dados, gerando estimativas dos
pontos ideais dos parlamentares, passando para a análise das características
gerais e temporais dos dados para, finalmente, na seção
\ref{cap:desenvolvimento:modelagem}, \nameref{cap:desenvolvimento:modelagem},
descrevermos o desenvolvimento e validação do modelo.

Por fim, o capítulo \ref{cap:conclusao}, \nameref{cap:conclusao}, apresenta as
conclusões da pesquisa, incluindo suas limitações e possíveis trabalhos futuros.
