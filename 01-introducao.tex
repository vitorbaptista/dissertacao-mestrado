\chapter{Introdução} \label{intro}

\section{Motivação}\label{sec:motivacao}

Desde XXXX, a Câmara dos Deputados disponibiliza em seu site o resultado das
votações nominais realizadas nesta Casa. Algumas ferramentas foram criadas para
analizar o comportamento dos parlamentares através desses dados, como o
Basômetro e o Radar Parlamentar. Essas ferramentas são usadas por jornalistas e
cientistas políticos, como CITAR LIVRO BASÔMETRO E MATERIAS DO ESTADAO. Apesar
disso, a análise desses dados ainda se dá de forma manual. Com 513 deputados
federais a cada legislatura, sem contar os suplentes, o volume de dados é muito
grande, então as análises acabam sendo feitas por partidos ou com parlamentares
específicos.

O objetivo desse trabalho é desenvolver um modelo capaz de determinar qual a
chance de um parlamentar ter mudado de posicionamento em um determinado
período. Defino uma mudança de posicionamento como sendo a entrada ou saída da
coalizão. Para isso, o modelo se baseia no padrão de votação, mensurado a
partir do modelo espacial de votação W-NOMINATE, e em padrões temporais. Ele
pode ser executado automaticamente, conforme novas votações forem surgindo, e
alertar algum interessado caso a chance de ter havido uma mudança de
posicionamento seja superior a algum valor pré-definido (e.x.: 80\%).

\section{Objetivos}

O objetivo geral desta dissertação é o desenvolvimento e validação de um modelo
capaz de determinar a chance de um deputado federal ter mudado de
posicionamento em um determinado período.

\subsection{Objetivos Especificos}

\section{Problema de pesquisa}

Os partidos e as coalizões governamentais são capazes de influenciar o
comportamento dos parlamentares \cite{Figueiredo2001,Santos2003}. Partindo
disso, \citeonline{Izumi2013} mostrou que o comportamento dos senadores muda ao
entrar ou sair da coalizão, mas não sabemos se ele muda antes ou depois da
oficialização dessa mudança. A pergunta que buscamos responder neste trabalho
é:

% FIXME: A partir dessa frase, parece que o modelo irá dizer "O deputado X vai
% mudar de posicionamento daqui a 3 meses", o que é errado. Ele vai dizer "Eu
% tenho 85% de certeza que o deputado X mudou de posicionamento entre os
% períodos Y1 e Y2".
\emph{É possível prever se um deputado federal irá mudar de posicionamento,
entrando ou saindo da coalizão do governo, a partir do seu padrão de votação?}

Para tanto, partimos do pressuposto que os parlamentares mudam de comportamento
antes de mudar de posicionamento. Caso isto esteja errado e eles mudem
depois (ou não mudem), não conseguiremos prever essa mudança de posicionamento.

\section{Metodologia}

Não sei qual o nome da metodologia que estou usando: por enquanto é tudo muito ad-hoc. Estudo de caso é um componente, mas não sei se há outra.

Alexandre -  Não é preciso dar uma nome a metodologia. O que você precisa descrever aqui são os passos seguidos até chegar ao resultado final.

Tipo
1 - Levantamento Bibliográfico sobre X e Y
2 - Desenvolviment do mecanismo para extração e filtragem dos dados
3 - ...

\section{Publicações Relacionadas}

Mencionar o artigo que você publicou no BRASNAM

\section{Estrutura da Dissertação}

Descrever a estrutura dos demais capítulos. Melhor fazer isso só no final.
