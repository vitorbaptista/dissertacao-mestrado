\chapter{Introdução} \label{intro}

\section{Motivação}\label{sec:motiva}

A quantidade de dados relacionados a atividade dos parlamentares brasileiros é muito grande. Exemplos simples são as votações, presenças em plenário, projetos de Lei, viagens, candidaturas, etc. Analisar manualmente esse volume de dados é uma tarefa bastante difícil e por isso podem haver situações importantes que não sejam percebidas. Visando auxiliar no processo de acompanhamento e fiscalização da atividade dos parlamentares brasileiros este trabalho propõe uma ferramenta de monitoramento automático e contínuo que gera alertas e relatórios sobre o que acontece nas casas legislativas brasileiras, usando a Câmara dos Deputados como estudo de caso.

\section{Objetivos}

O objetivo geral desta dissertação é o desenvolvimento e validação de um modelo
capaz de prever quando um deputado federal irá deixar ou entrar na coalizão do
governo a partir das votações nominais.

\subsection{Objetivos Especificos}

\begin{itemize}
\item Desenvolver um mecanismo capaz de detectar mudanças no padrão de comportamento dos parlamentares;
\item Desenvolver um mecanismo capaz de detectar mudanças semelhantes entre grupos de parlamentares (ex.: João e Pedro sempre divergiram em seus votos e de repente passaram a concordar);
\item Desenvolver um mecanismo capaz de detectar votações importantes/polêmicas (falta definir o que são importante e polêmica)
\end{itemize}

\section{Metodologia}

Não sei qual o nome da metodologia que estou usando: por enquanto é tudo muito ad-hoc. Estudo de caso é um componente, mas não sei se há outra.

Alexandre -  Não é preciso dar uma nome a metodologia. O que você precisa descrever aqui são os passos seguidos até chegar ao resultado final.

Tipo
1 - Levantamento Bibliográfico sobre X e Y
2 - Desenvolviment do mecanismo para extração e filtragem dos dados
3 - ...

\section{Publicações Relacionadas}

Mencionar o artigo que você publicou no BRASNAM

\section{Estrutura da Dissertação}

Descrever a estrutura dos demais capítulos. Melhor fazer isso só no final.
