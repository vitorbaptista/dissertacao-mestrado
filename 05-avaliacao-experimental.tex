\chapter{Estudos de Caso}\label{cap:avalia}

Aqui irei escolher um (ou mais) relatórios gerados pela ferramenta e analisar seu conteúdo. Por exemplo, se me falaram que um projeto de Lei é interessante (anômalo, etc.), confirmar se ele realmente é.

Vejo duas formas de fazer isso. Uma é considerando só o relatório, que foi gerado a partir de algoritmos que analisam diversos dados diferentes. Por exemplo, um pode ver modificações na coesão, outro o número de destaques em um PL, outro a nomeação de ministros. Ou seja, estou testando a ferramenta como um todo, mas não cada algoritmo/métrica. Pode ser que analisar nomeação de ministros não dê resultados interessantes, mas não conseguirei ver isso a partir do relatório.

Outra forma seria analisar o resultado de cada algoritmo. Assim, o estudo de caso seria executar um algoritmo em um período qualquer e ver que resultados ele me dá. Estaria fazendo um estudo de caso para cada algoritmo. Assim, justifico melhor a escolha dos algoritmos, mas em contrapartida estou fazendo uma análise de um ponto de vista que os usuários do sistema não verão. Não é uma visão "holística" de "o sistema me dá bons resultados", mas sim "esse algoritmo específico me dá bons resultados".

Não sei se consegui explicar direito.

Talvez faça mais sentido estudos de caso se focarem no sistema como um todo, mas eu faria uma validação de cada algoritmo em algum outro lugar, analisando se ele dá resultados interessantes. Não sei onde isso entraria.

(Alexandre) Essa parte da análise dos algoritmos é muito interessante sim mas não seria um estudo de caso. Acho que ela entraria melhor no capítulo anterior como uma análise experimental sobre os algoritmos a serem utilizados na ferramenta.

Os estudos de caso seriam mais amplos e focados na ferramenta como um todo como você mesmo descreveu.



\section{Estudo de Caso}

\subsection{Ferramentas e Tecnologia}
\subsection{Requisitos}
\subsection{Desenvolvimento}
\subsection{Avaliação}

\section{Experimento}

\subsection{Plano do Experimento}
\subsection{Execução do Experimento}
\subsection{Análise do Experimento}

\section{Considerações Finais}


