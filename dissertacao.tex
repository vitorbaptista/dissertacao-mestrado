%%%%%%%%%%%%%%%%%%%%%%%%%%%%%%%%%%%%%%%%%%%%%%%%%%%%%%%%%%%%%%%%%%%%%%%%%%%%%%%%%%%%%%%%%%%
%% Ultimas modificacoes, 06/02/2012 - Alexandre Duarte 
%% Baseado no modelo latex de Isaac Maia (COPIN/UFCG)
%%
%% Para utilizar ese modelo sao necessarios os seguintes arquivos:
%%
%% ppgi.cls
%% ppgi.sty
%% mestre.sty
%%
%%
%% Mais detalhes sobre normas ABNT no latex, consultar http://abntex.codigolivre.org.br
%% Wiki interessante com dicas uteis sobre latex : http://www.tex-br.org
%%
%%
%% Para compilar esse arquivo, e' sempre importante fazer duas passagens com latex
%%%%
%%%%%%%%%%%%%%%%%%%%%%%%%%%%%%%%%%%%%%%%%%%%%%%%%%%%%%%%%%%%%%%%%%%%%%%%%%%%%%%%%%%%%%%%%%%

\documentclass[a4paper,titlepage]{ppgi}
\usepackage[portuguese,ruled,linesnumbered]{algorithm2e}
\usepackage[english,portuges]{babel}
\usepackage{ppgi,mestre,epsfig}
\usepackage{times}
\usepackage[final]{pdfpages}
\usepackage{hyperref}
\hypersetup{
    bookmarks=true,   
    pdftitle={Monitor legislativo},
    pdfauthor={Vitor Márcio Paiva de Sousa Baptista}, 
    pdfsubject={Modelo de Documento Científico},
    pdfkeywords={Dissertação, Mestrado, PPGI, UFPB, modelo}, 
    colorlinks=true,
    linkcolor=black,
    citecolor=black,
    filecolor=black,
    urlcolor=black
 }

% Corrige bug no algorithm2e que usa termos em espanhol ao invés de português
\SetKwFor{Para}{para}{fa\c{c}a}{fim para}
\SetKwFor{ParaPar}{para}{fa\c{c}a em paralelo}{fim para}
\SetKwFor{ParaCada}{para cada}{fa\c{c}a}{fim para cada}
\SetKwFor{ParaTodo}{para todo}{fa\c{c}a}{fim para todo}

%-------------------------- Para usar acentuacaoo em sistemas ISO8859-1 ------------------------------------
% Se estiver usando o Microsoft Windows ou linux com essa codificacao, descomente essas linhas abaixo
% e comente as linhas referentes ao UTF8
%\usepackage[applemac]{inputenc} % Usar acentuacao em sistemas ISO8859-1, comentar a linha com  \usepackage[utf8x {inputenc}
%-----------------------------------------------------------------------------------------------------

%-------------------------- Para usar acentuacao em sistemas UTF8 ------------------------------------
% Para a maior parte das distribuicoes linux, usar essa opcao
\usepackage{ucs}
\usepackage[utf8x]{inputenc}
\usepackage[T1]{fontenc}
%-----------------------------------------------------------------------------------------------------
\usepackage{float}      
\usepackage{fancyvrb}
\usepackage{fancyheadings}
\usepackage{graphicx}
\graphicspath{{figure/}}
\setkeys{Gin}{width=0.5\textwidth}
\usepackage{longtable} %tabelas longas, para tabelas que ultrapassam uma pagina
\usepackage{minted} % Syntax highlighting para código
\usepackage{pdflscape} % landscape figures
\usepackage{glossaries} % Gerar glossário

\makeglossaries
\newacronym{IR}{IR}{Índice de Rice}
\newacronym{PT}{PT}{Partido dos Trabalhadores}
\newacronym{PSOL}{PSOL}{Partido Socialismo e Liberdade}
\newacronym{API}{API}{\emph{Application Programming Interface}}
\newacronym{XML}{XML}{\emph{eXtensible Markup Language}}
\newacronym{CSV}{CSV}{\emph{Comma Separated Values}}
\newacronym{JSON}{JSON}{\emph{JavaScript Object Notation}}
\newacronym{CEBRAP}{CEBRAP}{Centro Brasileiro de An\'alise e Pesquisa}

%\input{psfig.sty}
% ----------------- Para inserir codigo fonte de linguagens de programacao no documento -------------
\usepackage{listings}
\lstset{numbers=left,
stepnumber=1,
firstnumber=1,
numberstyle=\scriptsize,
extendedchars=true,
breaklines=true,
frame=tb,
basicstyle=\scriptsize,
stringstyle=\ttfamily,
showstringspaces=false
}
\renewcommand{\lstlistingname}{C\'odigo Fonte}
\renewcommand{\lstlistlistingname}{Lista de C\'odigos Fonte}

% ---------------------------------------------------------------------------------------------------

\selectlanguage{portuges}
\sloppy

\setcounter{secnumdepth}{4}
\setcounter{tocdepth}{4}
\usepackage{abnt-alf}

% Centraliza todos os floats
\makeatletter
\g@addto@macro\@floatboxreset\centering
\makeatother

\begin{document}


%%%%%%%%%%%%%%%%%%%%%%%%%%%%%%%%%%%%%%%%%%%%%%%%%%%%%%%%%%%%%%%%%%%%%%%%%%%%%%%%
\Titulo{Monitor legislativo}
\Autor{Vitor Márcio Paiva de Sousa Baptista}
\Data{06 de Março de 2012}
\Area{Ciência da Computação}
\Pesquisa{Computação Distribuída | Sinais, Sistemas Digitais e Gráficos}
\Orientadores{Alexandre Nóbrega Duarte\\ (Orientador)}

\newpage
\cleardoublepage
\PaginadeRosto

\newpage
\cleardoublepage

%%%%%%%%%%%%%%%%%%%%%%%%%%%%%%%%%%%%%%%%%%%%%%%%%%%%%%%%%%%%%%%%%%%%%%%%%%%%%%%%
\begin{resumo} 
No Brasil, existem ferramentas para o acompanhamento do comportamento dos
parlamentares em votações nominais, tais como o Basômetro do jornal O Estado de
São Paulo e o Radar Parlamentar. Essas ferramentas são usadas para análises
tanto por jornalistas quanto por cientistas políticos.

Apesar de serem ótimas ferramentas de análise, sua utilidade para monitoramento
é limitada por exigir um acompanhamento manual, o que se torna muito trabalhoso
quando consideramos o volume de dados. Somente na Câmara dos Deputados, 513
parlamentares participam em média de mais de 400 votações nominais por
legislatura. É possível diminuir a quantidade de dados analisando os partidos
como um todo, mas em contrapartida perdemos a capacidade de detectar
movimentações de indivíduos ou grupos intrapartidários como as bancadas.

% FIXME: Estava falando de análise de comportamento em geral, e aqui já pulei
% para mudanças de posicionamento. Falta algo para ligar os dois.

Para diminuir esse problema, desenvolvi neste trabalho um modelo estatístico
que detecta quando um parlamentar muda de posicionamento, entrando ou saindo da
coalizão governamental, através de estimativas de pontos ideais usando o
W-NOMINATE. Ele pode ser usado individualmente ou integrado a ferramentas como
o Basômetro, oferecendo um filtro para os pesquisadores encontrarem os
parlamentares que mudaram mais significativamente de comportamento.

O universo de estudo é composto pelos parlamentares da Câmara dos Deputados no
período da 50\textordfeminine{} até a 54\textordfeminine{} legislaturas,
iniciando no primeiro mandato de Fernando Henrique Cardoso em 1995 até o final
do primeiro mandato de Dilma Rousseff em 2015.
\\
\\
\textbf{Palavras-chave:} Análise legislativa, Ciência política, Ciência de
dados, Modelos preditivos, Aprendizagem de máquina.

% FIXME: Faltou falar um pouco dos resultados.

\end{resumo}
%\newpage
%\cleardoublepage

%%%%%%%%%%%%%%%%%%%%%%%%%%%%%%%%%%%%%%%%%%%%%%%%%%%%%%%%%%%%%%%%%%%%%%%%%%%%%%%%
\begin{summary}
In Brazil, there are tools for monitoring the behaviour of legislators in
rollcalls, such as O Estado de São Paulo's Basômetro and Radar Parlamentar.
These tools are used both by journalists and political scientists for analysis.

Although they are great analysis tools, their usefulness for monitoring is
limited because they require a manual follow-up, which makes it a lot of work
when we consider the volume of data. Only in the Chamber of Deputies, 513
legislators participate on average over than 400 rollcalls by legislature. It
is possible to decrease the amount of data analyzing the parties as a whole,
but in contrast we lose the ability to detect individuals' drives or
intra-party groups such as factions.

In order to mitigate this problem, I developed a statistical model that detects
when a legislator changes his or her position, joining or leaving the
governmental coalition, through ideal points estimates using the W-NOMINATE. It
can be used individually or integrated to tools such as Basômetro, providing a
filter for researchers find the deputies who changed their behaviour most
significantly.

The universe of study is composed of legislators from the Chamber of Deputies
from the 50th to the 54th legislatures, starting in the first term of Fernando
Henrique Cardoso in 1995 until the end of the first term of Dilma Rousseff in
2015.

\textbf{Keywords:} Legislative Analysis, Political Science, Data Science,
Predictive Models, Machine Learning.

\end{summary}

%\newpage
%\cleardoublepage

%%%%%%%%%%%%%%%%%%%%%%%%%%%%%%%%%%%%%%%%%%%%%%%%%%%%%%%%%%%%%%%%%%%%%%%%%%%%%%%%
% TMP: Agradecimentos
\begin{agradecimentos}
Devo todas as conquistas de minha vida a minha família. Foi o seu
trabalho, amor e dedicação que me ensinaram e me permitiram fazer o que fiz. Se
todos chegamos aonde estamos por nos apoiarmos nos ombros de gigantes, foram
eles e elas os primeiros gigantes nos quais me apoiei.

Agradeço em especial a minha esposa, melhor amiga e coorientadora não-oficial
Samara. Sem sua ajuda, esse trabalho não seria possível e minha vida seria
muito mais solitária.

Agradeço também ao meu orientador, Alexandre. Só o conheci ao me inscrever no
mestrado, mas ao longo desses anos tive certeza que não poderia ter tido mais
sorte nessa escolha. Por sua orientação técnica, mas principalmente pelo seu
interesse indiscutível nessa área de pesquisa. Em outra situação, acredito que
ele mesmo teria escrito esse trabalho, o que é prova inegável da nossa
sintonia.

Agradeço as professoras Andréa Freitas e Thaís Gaudêncio, que aceitaram
participar da minha banca, emprestando seu tempo e conhecimento para a melhoria
deste trabalho.

Agradeço aos pesquisadores e funcionários do \gls{CEBRAP}, cuja contribuição à
área da Ciência Política é incalculável, no Brasil e no mundo. Em especial,
gostaria de agradecer a Andréa Freitas, ao Samuel Moura e ao Maurício Izumi, que 
me ajudaram muito a validar as ideias que discuti nesse trabalho, e ao Paulo
Hubert, que me auxiliou a acessar o banco de dados legislativos do
\gls{CEBRAP}, do qual extraí a lista de coalizões usada. Além, é claro, a
Argelina Figueiredo e o Fernando Limongi, cuja pesquisa foi um divisor de águas
no pensamento da Ciência Política brasileira.

Ao longo do tempo, o contato com pessoas interessadas na intersecção entre
computação, política e jornalismo foi abrindo meus olhos para essa nova área
que acho extremamente interessante. Isso foi possibilitado, principalmente,
pela criação do grupo Transparência Hacker por, entre tantos outros, Pedro
Markun e Daniela Silva. Através desse grupo, conheci pessoas fenomenais como os
jornalistas Daniel Bramatti, José Roberto de Toledo e Amanda Rossi que, junto
com o Diego Rabatone, formavam o Estadão Dados, onde tive o prazer de trabalhar
por uma semana durante o segundo turno das eleições de 2012.

Agradeço aos amigos criados durante a organização do Encontro de Software Livre
da Paraíba (ENSOL), em especial a Rodrigo Vieira e Anahuac de Paula Gil, os
principais responsáveis no meu amadurecimento com relação a software e cultura
livres.

Trabalhando na ThoughtWorks em Porto Alegre, fiz diversos amigos. Em especial,
Leonardo Tartari e Thiago Bueno, companheiros de vários hackathons, foram quem
despertaram em mim o interesse pela visualização de dados, que foi uma das
razões que me fizeram entrar na Open Knowledge Foundation (OKF).

A OKF é uma ONG inglesa que trabalha com dados abertos. Durante os anos que
trabalhei nela, tive oportunidade de conhecer diversas pessoas que me ajudaram
a me aprofundar nessa área, em especial o time de desenvolvimento do CKAN e os
fundadores da Open Knowledge Foundation Brasil.

Por último, mas de forma alguma menos importante, agradeço aos amigos brutais,
os irmãos e irmãs que encontrei durante a vida. Em especial ao Pedro Guimarães
que, além de amigo e parceiro em diversos projetos, se tornou meu cunhado.

À todas essas pessoas e muitas outras, dedico esse trabalho.

\end{agradecimentos}

\clearpage

%%%%%%%%%%%%%%%%%%%%%%%%%%%%%%%%%%%%%%%%%%%%%%%%%%%%%%%%%%%%%%%%%%%%%%%%%%%%%%%%
%% Definicao do cabecalho: secao do lado esquerdo e numero da pagina do lado direito
\pagestyle{fancy}
\addtolength{\headwidth}{\marginparsep}\addtolength{\headwidth}{\marginparwidth}\headwidth = \textwidth
\renewcommand{\chaptermark}[1]{\markboth{#1}{}}
\renewcommand{\sectionmark}[1]{\markright{\thesection\ #1}}\lhead[\fancyplain{}{\bfseries\thepage}]%
	     {\fancyplain{}{\emph{\rightmark}}}\rhead[\fancyplain{}{\bfseries\leftmark}]%
             {\fancyplain{}{\bfseries\thepage}}\cfoot{}

%%%%%%%%%%%%%%%%%%%%%%%%%%%%%%%%%%%%%%%%%%%%%%%%%%%%%%%%%%%%%%%%%%%%%%%%%%%%%%%%

\Sumario
\ListadeSiglas
\listoffigures
\listoftables
\lstlistoflistings %lista de codigos fonte - Para inserir a listagem de
% codigos fonte

\newpage
\cleardoublepage
\Introducao


%%%%%%%%%%%%%%%%%%%%%%%%%%%%%%%%%%%%%%%%%%%%%%%%%%%%%%%%%%%%%%%%%%%%%%%%%%%%%%%%
%
% Hifenizacao - Colocar lista de palavras que nao devem ser separadas e que 
% nao estao no dicionario portuges.
% As palavras do dicionario portuges ja sao separadas corretamente pelo lateX
%
\hyphenation{ gLite OurGrid GridDoctor }

%%%%%%%%%%%%%%%%%%%%%%%%%%%%%%%%%%%%%%%%%%%%%%%%%%%%%%%%%%%%%%%%%%%%%%%%%%%%%%%%
%% A partir daqui coloque seus capitulos. Sugere-se que eles sejam inseridos com o comando \input
%% Da seguinte maneira:
%% 



