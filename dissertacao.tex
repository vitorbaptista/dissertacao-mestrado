%%%%%%%%%%%%%%%%%%%%%%%%%%%%%%%%%%%%%%%%%%%%%%%%%%%%%%%%%%%%%%%%%%%%%%%%%%%%%%%%%%%%%%%%%%%
%% Ultimas modificacoes, 06/02/2012 - Alexandre Duarte 
%% Baseado no modelo latex de Isaac Maia (COPIN/UFCG)
%%
%% Para utilizar ese modelo sao necessarios os seguintes arquivos:
%%
%% ppgi.cls
%% ppgi.sty
%% mestre.sty
%%
%%
%% Mais detalhes sobre normas ABNT no latex, consultar http://abntex.codigolivre.org.br
%% Wiki interessante com dicas uteis sobre latex : http://www.tex-br.org
%%
%%
%% Para compilar esse arquivo, e' sempre importante fazer duas passagens com latex
%%%%
%%%%%%%%%%%%%%%%%%%%%%%%%%%%%%%%%%%%%%%%%%%%%%%%%%%%%%%%%%%%%%%%%%%%%%%%%%%%%%%%%%%%%%%%%%%

\documentclass[a4paper,titlepage]{ppgi}\usepackage[]{graphicx}\usepackage[]{color}
%% maxwidth is the original width if it is less than linewidth
%% otherwise use linewidth (to make sure the graphics do not exceed the margin)
\makeatletter
\def\maxwidth{ %
  \ifdim\Gin@nat@width>\linewidth
    \linewidth
  \else
    \Gin@nat@width
  \fi
}
\makeatother

\definecolor{fgcolor}{rgb}{0.345, 0.345, 0.345}
\newcommand{\hlnum}[1]{\textcolor[rgb]{0.686,0.059,0.569}{#1}}%
\newcommand{\hlstr}[1]{\textcolor[rgb]{0.192,0.494,0.8}{#1}}%
\newcommand{\hlcom}[1]{\textcolor[rgb]{0.678,0.584,0.686}{\textit{#1}}}%
\newcommand{\hlopt}[1]{\textcolor[rgb]{0,0,0}{#1}}%
\newcommand{\hlstd}[1]{\textcolor[rgb]{0.345,0.345,0.345}{#1}}%
\newcommand{\hlkwa}[1]{\textcolor[rgb]{0.161,0.373,0.58}{\textbf{#1}}}%
\newcommand{\hlkwb}[1]{\textcolor[rgb]{0.69,0.353,0.396}{#1}}%
\newcommand{\hlkwc}[1]{\textcolor[rgb]{0.333,0.667,0.333}{#1}}%
\newcommand{\hlkwd}[1]{\textcolor[rgb]{0.737,0.353,0.396}{\textbf{#1}}}%

\usepackage{framed}
\makeatletter
\newenvironment{kframe}{%
 \def\at@end@of@kframe{}%
 \ifinner\ifhmode%
  \def\at@end@of@kframe{\end{minipage}}%
  \begin{minipage}{\columnwidth}%
 \fi\fi%
 \def\FrameCommand##1{\hskip\@totalleftmargin \hskip-\fboxsep
 \colorbox{shadecolor}{##1}\hskip-\fboxsep
     % There is no \\@totalrightmargin, so:
     \hskip-\linewidth \hskip-\@totalleftmargin \hskip\columnwidth}%
 \MakeFramed {\advance\hsize-\width
   \@totalleftmargin\z@ \linewidth\hsize
   \@setminipage}}%
 {\par\unskip\endMakeFramed%
 \at@end@of@kframe}
\makeatother

\definecolor{shadecolor}{rgb}{.97, .97, .97}
\definecolor{messagecolor}{rgb}{0, 0, 0}
\definecolor{warningcolor}{rgb}{1, 0, 1}
\definecolor{errorcolor}{rgb}{1, 0, 0}
\newenvironment{knitrout}{}{} % an empty environment to be redefined in TeX

\usepackage{alltt}
\usepackage[portuguese,ruled,linesnumbered]{algorithm2e}
\usepackage[english,portuges]{babel}
\usepackage{ppgi,mestre,epsfig}
\usepackage{times}
\usepackage[final]{pdfpages}
\usepackage{hyperref}
\hypersetup{
    bookmarks=true,   
    pdftitle={Monitor legislativo},
    pdfauthor={Vitor Márcio Paiva de Sousa Baptista}, 
    pdfsubject={Modelo de Documento Científico},
    pdfkeywords={Dissertação, Mestrado, PPGI, UFPB, modelo}, 
    colorlinks=true,
    linkcolor=black,
    citecolor=black,
    filecolor=black,
    urlcolor=black
 }

% Corrige bug no algorithm2e que usa termos em espanhol ao invés de português
\SetKwFor{Para}{para}{fa\c{c}a}{fim para}
\SetKwFor{ParaPar}{para}{fa\c{c}a em paralelo}{fim para}
\SetKwFor{ParaCada}{para cada}{fa\c{c}a}{fim para cada}
\SetKwFor{ParaTodo}{para todo}{fa\c{c}a}{fim para todo}

%-------------------------- Para usar acentuacaoo em sistemas ISO8859-1 ------------------------------------
% Se estiver usando o Microsoft Windows ou linux com essa codificacao, descomente essas linhas abaixo
% e comente as linhas referentes ao UTF8
%\usepackage[applemac]{inputenc} % Usar acentuacao em sistemas ISO8859-1, comentar a linha com  \usepackage[utf8x {inputenc}
%-----------------------------------------------------------------------------------------------------

%-------------------------- Para usar acentuacao em sistemas UTF8 ------------------------------------
% Para a maior parte das distribuicoes linux, usar essa opcao
\usepackage{ucs}
\usepackage[utf8x]{inputenc}
\usepackage[T1]{fontenc}
%-----------------------------------------------------------------------------------------------------
\usepackage{float}      
\usepackage{fancyvrb}
\usepackage{fancyheadings}
\usepackage{graphicx}
\usepackage{longtable} %tabelas longas, para tabelas que ultrapassam uma pagina
\usepackage{minted} % Syntax highlighting para código
\usepackage{pdflscape} % landscape figures
\usepackage{glossaries} % Gerar glossário

\makeglossaries
\newacronym{IR}{IR}{Índice de Rice}
\newacronym{PT}{PT}{Partido dos Trabalhadores}
\newacronym{PSOL}{PSOL}{Partido Socialismo e Liberdade}
\newacronym{API}{API}{\emph{Application Programming Interface}}
\newacronym{XML}{XML}{\emph{eXtensible Markup Language}}
\newacronym{CSV}{CSV}{\emph{Comma Separated Values}}
\newacronym{JSON}{JSON}{\emph{JavaScript Object Notation}}
\newacronym{CEBRAP}{CEBRAP}{Centro Brasileiro de An\'alise e Pesquisa}

%\input{psfig.sty}
% ----------------- Para inserir codigo fonte de linguagens de programacao no documento -------------
\usepackage{listings}
\lstset{numbers=left,
stepnumber=1,
firstnumber=1,
numberstyle=\scriptsize,
extendedchars=true,
breaklines=true,
frame=tb,
basicstyle=\scriptsize,
stringstyle=\ttfamily,
showstringspaces=false
}
\renewcommand{\lstlistingname}{C\'odigo Fonte}
\renewcommand{\lstlistlistingname}{Lista de C\'odigos Fonte}

% ---------------------------------------------------------------------------------------------------

\selectlanguage{portuges}
\sloppy

\setcounter{secnumdepth}{4}
\setcounter{tocdepth}{4}
\usepackage{abnt-alf}

% Centraliza todos os floats
\makeatletter
\g@addto@macro\@floatboxreset\centering
\makeatother
\IfFileExists{upquote.sty}{\usepackage{upquote}}{}
\begin{document}


%%%%%%%%%%%%%%%%%%%%%%%%%%%%%%%%%%%%%%%%%%%%%%%%%%%%%%%%%%%%%%%%%%%%%%%%%%%%%%%%
\Titulo{Monitor legislativo}
\Autor{Vitor Márcio Paiva de Sousa Baptista}
\Data{06 de Março de 2012}
\Area{Ciência da Computação}
\Pesquisa{Computação Distribuída | Sinais, Sistemas Digitais e Gráficos}
\Orientadores{Alexandre Nóbrega Duarte\\ (Orientador)}

\newpage
\cleardoublepage
\PaginadeRosto

\newpage
\cleardoublepage

%%%%%%%%%%%%%%%%%%%%%%%%%%%%%%%%%%%%%%%%%%%%%%%%%%%%%%%%%%%%%%%%%%%%%%%%%%%%%%%%
\begin{resumo} 
No Brasil, existem ferramentas para o acompanhamento do comportamento dos
parlamentares em votações nominais, tais como o Basômetro do jornal O Estado de
São Paulo e o Radar Parlamentar. Essas ferramentas são usadas para análises
tanto por jornalistas quanto por cientistas políticos.

Apesar de serem ótimas ferramentas de análise, sua utilidade para monitoramento
é limitada por exigir um acompanhamento manual, o que se torna muito trabalhoso
quando consideramos o volume de dados. Somente na Câmara dos Deputados, 513
parlamentares participam em média de mais de 400 votações nominais por
legislatura. É possível diminuir a quantidade de dados analisando os partidos
como um todo, mas em contrapartida perdemos a capacidade de detectar
movimentações de indivíduos ou grupos intrapartidários como as bancadas.

% FIXME: Estava falando de análise de comportamento em geral, e aqui já pulei
% para mudanças de posicionamento. Falta algo para ligar os dois.

Para diminuir esse problema, desenvolvi neste trabalho um modelo estatístico
que detecta quando um parlamentar muda de posicionamento, entrando ou saindo da
coalizão governamental, através de estimativas de pontos ideais usando o
W-NOMINATE. Ele pode ser usado individualmente ou integrado a ferramentas como
o Basômetro, oferecendo um filtro para os pesquisadores encontrarem os
parlamentares que mudaram mais significativamente de comportamento.

O universo de estudo é composto pelos parlamentares da Câmara dos Deputados no
período da 50\textordfeminine{} até a 54\textordfeminine{} legislaturas,
iniciando no primeiro mandato de Fernando Henrique Cardoso em 1995 até o final
do primeiro mandato de Dilma Rousseff em 2015.
\\
\\
\textbf{Palavras-chave:} Análise legislativa, Ciência política, Ciência de
dados, Modelos preditivos, Aprendizagem de máquina.

% FIXME: Faltou falar um pouco dos resultados.

\end{resumo}
%\newpage
%\cleardoublepage

%%%%%%%%%%%%%%%%%%%%%%%%%%%%%%%%%%%%%%%%%%%%%%%%%%%%%%%%%%%%%%%%%%%%%%%%%%%%%%%%
\begin{summary}
In Brazil, there are tools for monitoring the behaviour of legislators in
rollcalls, such as O Estado de São Paulo's Basômetro and Radar Parlamentar.
These tools are used both by journalists and political scientists for analysis.

Although they are great analysis tools, their usefulness for monitoring is
limited because they require a manual follow-up, which makes it a lot of work
when we consider the volume of data. Only in the Chamber of Deputies, 513
legislators participate on average over than 400 rollcalls by legislature. It
is possible to decrease the amount of data analyzing the parties as a whole,
but in contrast we lose the ability to detect individuals' drives or
intra-party groups such as factions.

In order to mitigate this problem, I developed a statistical model that detects
when a legislator changes his or her position, joining or leaving the
governmental coalition, through ideal points estimates using the W-NOMINATE. It
can be used individually or integrated to tools such as Basômetro, providing a
filter for researchers find the deputies who changed their behaviour most
significantly.

The universe of study is composed of legislators from the Chamber of Deputies
from the 50th to the 54th legislatures, starting in the first term of Fernando
Henrique Cardoso in 1995 until the end of the first term of Dilma Rousseff in
2015.

\textbf{Keywords:} Legislative Analysis, Political Science, Data Science,
Predictive Models, Machine Learning.

\end{summary}

%\newpage
%\cleardoublepage

%%%%%%%%%%%%%%%%%%%%%%%%%%%%%%%%%%%%%%%%%%%%%%%%%%%%%%%%%%%%%%%%%%%%%%%%%%%%%%%%
% TMP: Agradecimentos
\begin{agradecimentos}
Devo todas as conquistas de minha vida a minha família. Foi o seu
trabalho, amor e dedicação que me ensinaram e me permitiram fazer o que fiz. Se
todos chegamos aonde estamos por nos apoiarmos nos ombros de gigantes, foram
eles e elas os primeiros gigantes nos quais me apoiei.

Agradeço em especial a minha esposa, melhor amiga e coorientadora não-oficial
Samara. Sem sua ajuda, esse trabalho não seria possível e minha vida seria
muito mais solitária.

Agradeço também ao meu orientador, Alexandre. Só o conheci ao me inscrever no
mestrado, mas ao longo desses anos tive certeza que não poderia ter tido mais
sorte nessa escolha. Por sua orientação técnica, mas principalmente pelo seu
interesse indiscutível nessa área de pesquisa. Em outra situação, acredito que
ele mesmo teria escrito esse trabalho, o que é prova inegável da nossa
sintonia.

Agradeço as professoras Andréa Freitas e Thaís Gaudêncio, que aceitaram
participar da minha banca, emprestando seu tempo e conhecimento para a melhoria
deste trabalho.

Agradeço aos pesquisadores e funcionários do \gls{CEBRAP}, cuja contribuição à
área da Ciência Política é incalculável, no Brasil e no mundo. Em especial,
gostaria de agradecer a Andréa Freitas, ao Samuel Moura e ao Maurício Izumi, que 
me ajudaram muito a validar as ideias que discuti nesse trabalho, e ao Paulo
Hubert, que me auxiliou a acessar o banco de dados legislativos do
\gls{CEBRAP}, do qual extraí a lista de coalizões usada. Além, é claro, a
Argelina Figueiredo e o Fernando Limongi, cuja pesquisa foi um divisor de águas
no pensamento da Ciência Política brasileira.

Ao longo do tempo, o contato com pessoas interessadas na intersecção entre
computação, política e jornalismo foi abrindo meus olhos para essa nova área
que acho extremamente interessante. Isso foi possibilitado, principalmente,
pela criação do grupo Transparência Hacker por, entre tantos outros, Pedro
Markun e Daniela Silva. Através desse grupo, conheci pessoas fenomenais como os
jornalistas Daniel Bramatti, José Roberto de Toledo e Amanda Rossi que, junto
com o Diego Rabatone, formavam o Estadão Dados, onde tive o prazer de trabalhar
por uma semana durante o segundo turno das eleições de 2012.

Agradeço aos amigos criados durante a organização do Encontro de Software Livre
da Paraíba (ENSOL), em especial a Rodrigo Vieira e Anahuac de Paula Gil, os
principais responsáveis no meu amadurecimento com relação a software e cultura
livres.

Trabalhando na ThoughtWorks em Porto Alegre, fiz diversos amigos. Em especial,
Leonardo Tartari e Thiago Bueno, companheiros de vários hackathons, foram quem
despertaram em mim o interesse pela visualização de dados, que foi uma das
razões que me fizeram entrar na Open Knowledge Foundation (OKF).

A OKF é uma ONG inglesa que trabalha com dados abertos. Durante os anos que
trabalhei nela, tive oportunidade de conhecer diversas pessoas que me ajudaram
a me aprofundar nessa área, em especial o time de desenvolvimento do CKAN e os
fundadores da Open Knowledge Foundation Brasil.

Por último, mas de forma alguma menos importante, agradeço aos amigos brutais,
os irmãos e irmãs que encontrei durante a vida. Em especial ao Pedro Guimarães
que, além de amigo e parceiro em diversos projetos, se tornou meu cunhado.

À todas essas pessoas e muitas outras, dedico esse trabalho.

\end{agradecimentos}

\clearpage

%%%%%%%%%%%%%%%%%%%%%%%%%%%%%%%%%%%%%%%%%%%%%%%%%%%%%%%%%%%%%%%%%%%%%%%%%%%%%%%%
%% Definicao do cabecalho: secao do lado esquerdo e numero da pagina do lado direito
\pagestyle{fancy}
\addtolength{\headwidth}{\marginparsep}\addtolength{\headwidth}{\marginparwidth}\headwidth = \textwidth
\renewcommand{\chaptermark}[1]{\markboth{#1}{}}
\renewcommand{\sectionmark}[1]{\markright{\thesection\ #1}}\lhead[\fancyplain{}{\bfseries\thepage}]%
	     {\fancyplain{}{\emph{\rightmark}}}\rhead[\fancyplain{}{\bfseries\leftmark}]%
             {\fancyplain{}{\bfseries\thepage}}\cfoot{}

%%%%%%%%%%%%%%%%%%%%%%%%%%%%%%%%%%%%%%%%%%%%%%%%%%%%%%%%%%%%%%%%%%%%%%%%%%%%%%%%

\Sumario
\ListadeSiglas
\listoffigures
\listoftables
\lstlistoflistings %lista de codigos fonte - Para inserir a listagem de
% codigos fonte

\newpage
\cleardoublepage
\Introducao


%%%%%%%%%%%%%%%%%%%%%%%%%%%%%%%%%%%%%%%%%%%%%%%%%%%%%%%%%%%%%%%%%%%%%%%%%%%%%%%%
%
% Hifenizacao - Colocar lista de palavras que nao devem ser separadas e que 
% nao estao no dicionario portuges.
% As palavras do dicionario portuges ja sao separadas corretamente pelo lateX
%
\hyphenation{ gLite OurGrid GridDoctor }

%%%%%%%%%%%%%%%%%%%%%%%%%%%%%%%%%%%%%%%%%%%%%%%%%%%%%%%%%%%%%%%%%%%%%%%%%%%%%%%%
%% A partir daqui coloque seus capitulos. Sugere-se que eles sejam inseridos com o comando \input
%% Da seguinte maneira:
%% 



\chapter{Introdução} \label{intro}

Neste capítulo, serão descritos o que motivou o desenvolvimento deste trabalho
(Seção \ref{sec:motivacao}), juntamente com a definição do problema e objetivos
da pesquisa, a metodologia seguida, publicações relacionadas e, por fim, será
resumida a estrutura do restante da dissertação.

\section{Motivação}\label{sec:motivacao}

O acompanhamento das atividades dos legisladores é extremamente importante,
pois são eles que alteram as leis do Brasil. Isso afeta todos que têm alguma
relação com o país, estando ou não em solo brasileiro. As grandes empresas,
com recurso para investir, reconhecem a importância de acompanhar de perto a
atividade legislativa, seja passiva ou ativamente através de \emph{lobbying}.
Infelizmente, com exceção das leis que são divulgadas na mídia, como aumentos
do salário mínimo ou, recentemente, a redução da maioridade penal, a maioria
dos cidadãos não se interessa por essa área, seja por falta de tempo,
conhecimento ou simplesmente falta de interesse.

Jornalistas políticos exercem um papel fundamental nesse sentido, traduzindo os
termos técnicos e jurídicos usados pelos parlamentares em uma forma que possa
ser mais facilmente compreendida pelo cidadão comum. Entretanto, como essa
análise demanda bastante tempo, ela acaba se restringindo aos temas mais
polêmicos, que atingem um maior número de pessoas. Esses temas são muito
importantes, mas não suficientes: se eu trabalho numa ONG de preservação do
meio ambiente, por exemplo, meu maior interesse é em projetos de lei que versem
sobre essa área.

Percebendo essa necessidade, empresas como o \gls{ELLO} no Brasil e a
FiscalNote nos Estados Unidos, criaram ferramentas que facilitam esse
monitoramento personalizado. Seus produtos permitem que o usuário defina suas
áreas de interesse (por exemplo, meio ambiente ou mobilidade urbana),
recebendo resumos periódicos do que as afeta, inclusive com previsões da
probabilidade de aprovação dos projetos de lei relacionados a elas. Apesar
disso, por serem ferramentas pagas, seu uso ainda é restrito.

% Explica importância do monitoramento do comportamento dos legisladores
Um dos fatores mais importantes no comportamento dos parlamentares é o conflito
governo/oposição \cite{Leoni2002,Desposato2005b,Freitas2012,Izumi2013}. Diante
disso, quando um parlamentar muda de lado, seja migrando de partido ou quando
seu próprio partido se une (ou deixa) à coalizão governamental, é de se esperar
que seu comportamento também mude. Como, no geral, os partidos são capazes de
disciplinar seus filiados a votarem de certa forma, essa mudança de forças pode
definir a aprovação ou não de um projeto de lei. Assim, é essencial monitorar
essas mudanças para entender as chances que um projeto tem de ser aprovado.

Entretanto, o grande volume de dados gera um desafio. Considerando somente o
nível federal, 513 deputados federais e 81 senadores participam de centenas de
votações em cada legislatura, tornando difícil o entendimento dos seus padrões
de votação. Algumas ferramentas, como o Basômetro e o Radar Parlamentar, foram
criadas para tentar diminuir esse problema \cite{Estadao2012,Trento2013}.

Elas são ferramentas gratuitas que permitem o acompanhamento da taxa de
governismo (no caso do Basômetro) ou das posições relativas dos parlamentares
(no caso do Radar Parlamentar), auxiliando o cidadão comum a se aproximar do
processo legislativo \cite{Dantas2014}. Entretanto, mesmo facilitando bastante,
não é fácil analisar um gráfico com 513 pontos (no caso da Câmara) e visualizar
o que está mudando ou não. Para diminuir esse desafio, as análises acabam sendo
feitas baseadas no comportamento agregado dos partidos, e não dos parlamentares
individualmente.

Os partidos são capazes de influenciar o comportamento dos seus parlamentares,
mantendo taxas de disciplina, em sua maioria, acima de 75\%
\cite{Figueiredo2001,Cheibub2009,Zucco2009}. Assim, a análise a nível de partido
é uma forma importantíssima de entender o comportamento parlamentar. Apesar
disso, algumas informações são perdidas ao fazer essa agregação. Para dar um
exemplo concreto, o \gls{PT} é, historicamente, um dos partidos brasileiros mais
disciplinados. Em 2003, o primeiro ano do primeiro governo Lula, o \gls{PT} manteve uma
disciplina altíssima (98,36 \% na Câmara). Apesar disso, três deputados e uma
senadora constantemente votavam contrários a indicação do partido, o que acabou
resultando na sua expulsão no final do mesmo ano \cite{Breve2013}.

Esses parlamentares são os deputados Babá, Luciana Genro e João Fontes e a
senadora Heloísa Helena. Após serem expulsos, fundaram um novo partido: o
\gls{PSOL}. Essa movimentação passaria desapercebida ao analisar o \gls{PT}
como um todo, já que só 4 dos quase 100 deputados e senadores do partido
tiveram esse comportamento.

O objetivo deste trabalho é o desenvolvimento de um modelo estatístico que
determine a chance de um parlamentar ter mudado de posicionamento com base no
seu padrão de voto. Esse modelo foi treinado com os dados da
50\textordfeminine{} até a 54\textordfeminine{} legislaturas, compreendendo o
período de 20 anos de 1995, no início do primeiro governo de Fernando Henrique
Cardoso, até o início de 2015, no início do segundo governo de Dilma Rousseff.

Com isso, espero criar uma ferramenta para que os cidadãos, jornalistas e
cientistas políticos consigam filtrar e ordenar os parlamentares pela
intensidade da sua mudança de comportamento. Dessa forma, eles otimizariam o
uso do seu tempo, focando em quem está mudando. Esse modelo poderá ser usado
separadamente, ou integrado em ferramentas já existentes, como o Basômetro ou o
Radar Parlamentar, visando aumentar sua utilidade como ferramentas de
monitoramento legislativo.

Essa ferramenta poderá ser útil também para os próprios parlamentares,
especialmente os líderes, permitindo que monitorem se seus liderados estão
mudando de lado.

Como preditores, o modelo usa os pontos ideais dos parlamentares, estimados
pelo algoritmo W-NOMINATE \cite{Poole1985,Poole2005}. Além desses pontos,
também replicamos parte da pesquisa de \citeonline{Freitas2008}, que analisa os
aspectos temporais das migrações partidárias no Brasil, focando nas mudanças de
posicionamento, sejam a partir da migração para partidos de posicionamento
oposto (indo do governo para oposição ou vice-versa), ou na entrada ou saída do
próprio partido na coalizão governamental.

Uma das principais contribuições deste trabalho é democratizar o acesso a
técnicas antes só disponíveis em sistemas pagos de empresas como o \gls{ELLO}
e a FiscalNote.

\section{Problema de pesquisa}
\label{cap:introducao:problemas-de-pesquisa}

Os partidos e as coalizões governamentais são capazes de influenciar o
comportamento dos parlamentares \cite{Figueiredo2001,Santos2003}. Partindo
disso, \citeonline{Izumi2013} mostrou que o comportamento dos senadores muda ao
entrar ou sair da coalizão, mas não sabemos se ele muda antes ou depois da
oficialização dessa mudança. A pergunta a que buscamos responder neste trabalho
é:

\emph{É possível detectar a mudança de posicionamento de um deputado federal,
com ele entrando ou saindo da coalizão governamental, a partir de uma mudança
no seu padrão de votação?}

A premissa básica deste trabalho é que, além de ser possível detectar mudanças
de posicionamento, essa detecção ocorra antes da oficialização da mudança. Em
outras palavras, que o modelo seja capaz de detectar uma mudança de
posicionamento antes que ela seja do conhecimento público. Para isto,
precisaremos também responder à pergunta:

\emph{Os deputados federais mudam seu padrão de votação antes de mudarem de
posicionamento?}

Alguns autores mostraram que os parlamentares mudam de comportamento ao mudarem
de posicionamento (ver Capítulo \ref{cap:trabalhos-relacionados}). Apesar
disso, não foram encontrados trabalhos que discorram sobre se tal mudança de
comportamento ocorre antes ou depois da efetiva mudança de posicionamento.
Assim, responder a essa pergunta é uma outra contribuição deste trabalho.

\section{Objetivos}

O objetivo geral desta dissertação é o desenvolvimento e validação de um modelo
capaz de determinar a chance de um deputado federal ter mudado de
posicionamento em um determinado período.

Para alcançar esse objetivo geral, foram definidos os seguintes objetivos
específicos:

\begin{itemize}
  \item Determinar um conjunto de características a partir das quais seja
    possível determinar a chance de um parlamentar mudar de posicionamento;
  \item Descobrir se os parlamentares mudam de comportamento antes de mudarem
    de posicionamento;
  \item Analisar diversos modelos estatísticos, buscando qual tem melhor
    performance na detecção da mudança de posicionamento dos deputados
    federais brasileiros.
\end{itemize}

\section{Metodologia}

Para o desenvolvimento desta pesquisa, foram seguidos os seguintes passos:

\begin{enumerate}
  \item Levantamento bibliográfico sobre análise do comportamento parlamentar,
    análise da mudança do comportamento parlamentar, e métodos de aprendizado
    de máquina usados no âmbito da Ciência Política;
  \item Extração dos dados de votos e votações a partir da página da Câmara dos
    Deputados, e da listagem de coalizões a partir do banco de dados
    legislativo do \gls{CEBRAP};
  \item Definição da forma para representação desses dados, usando a teoria
    espacial do voto;
  \item Análise dos padrões gerais e temporais dos pontos ideais estimados;
  \item Definição de variáveis independentes capazes de serem usadas para
    diferenciar parlamentares que mudaram de posicionamento dos que não
    mudaram;
  \item Análise de diversos modelos preditivos buscando o que obtêm a melhor
    performance nesses dados;
  \item Validação do modelo final baseado na análise feita na etapa anterior.
\end{enumerate}

\section{Publicações Relacionadas}

Resultados iniciais desta pesquisa foram publicados no artigo ``Uma ferramenta
para analisar mudanças na coesão entre parlamentares em votações nominais'',
apresentado no III Brazilian Workshop on Social Network Analysis and Mining
(BRASNAM), ocorrido em 2014 \cite{Baptista2014}.

\section{Estrutura da Dissertação}

Esta dissertação é dividida em cinco capítulos, incluindo este introdutório,
que apresentou a motivação, problema de pesquisa e objetivos deste trabalho.

No Capítulo \ref{cap:fundamentacao}, \nameref{cap:fundamentacao}, serão
brevemente apresentados os conceitos básicos necessários para entendimento do
restante do trabalho. Na Seção \ref{cap:fundamentacao:ciencia-de-dados}, falarei
sobre a Ciência de Dados, com foco no desenvolvimento de modelos preditivos
para predição de variáveis categóricas e sua validação através de matrizes de
confusão e ferramentas como a curva \gls{ROC}.

Na Seção \ref{cap:fundamentacao:teoria-espacial-do-voto}, apresentarei técnicas
baseadas na teoria espacial do voto, que permitem colocar um conjunto de
parlamentares em um plano cartesiano, com suas posições definidas a partir de
seus votos em um conjunto de votações. Existem diversas técnicas para definir
essas posições, mas neste trabalho focarei no W-NOMINATE, uma das mais usadas.

Na Seção \ref{cap:fundamentacao:comparando-pontos-ideais-no-tempo}, será
apresentado o problema em comparar pontos ideais ao longo do tempo, descrevendo
técnicas que usam ``pontes'' para diferenciar mudanças causadas por diferenças
na agenda legislativa dos períodos das causadas pela mudança de comportamento
do parlamentar.

No Capítulo \ref{cap:trabalhos-relacionados},
\nameref{cap:trabalhos-relacionados}, serão resumidos alguns trabalhos,
encontrados durante a revisão bibliográfica feita nesta pesquisa, que versam
sobre a análise do comportamento parlamentar, a mudança de comportamento e o
uso de técnicas de aprendizagem de máquina na Ciência Política.

Explicadas, até este momento, as ferramentas usadas no trabalho e a literatura
da área, no Capítulo \ref{cap:desenvolvimento}, \nameref{cap:desenvolvimento},
será descrito o processo de criação do modelo, partindo da definição do
universo de estudo, coleta e preparação dos dados, gerando estimativas dos
pontos ideais dos parlamentares, passando pela análise das características
gerais e temporais dos dados para, finalmente, descrever o desenvolvimento e
validação do modelo na Seção \ref{cap:desenvolvimento:modelagem},
\nameref{cap:desenvolvimento:modelagem}.

Por fim, o Capítulo \ref{cap:conclusao}, \nameref{cap:conclusao}, apresenta as
conclusões da pesquisa, incluindo suas limitações e possíveis trabalhos futuros.


\chapter{Fundamentação Teórica}\label{cap:fundamentacao}

\section{Índice de Rice}

O \gls{IR} é uma métrica para o cálculo da coesão\footnote{Neste trabalho
entendemos por coesão o grau de unidade do partido nas decisões legislativas
tomadas em votações nominais, independente se isso ocorre porque seus membros
têm uma ideologia semelhante, ou porque seus líderes conseguem discipliná-los.}
entre grupos em uma votação. Desde sua introdução em \cite{Rice1924}, ele vem
sendo amplamente utilizado em estudos legislativos por ser intuitivo, flexível
e facilmente interpretável \cite{Neiva2011}.

Para uma votação $i$, o seu cálculo se dá pela diferença, em números absolutos,
entre a quantidade de votos \verb|Sim| e de votos \verb|Não|, dividida pelo
\verb|Total| de votos, como pode ser visto na fórmula \ref{eq:rice}.

\begin{equation}\label{eq:rice}
  IR_i = \frac{|Sim - N\tilde{a}o|}{Total}
\end{equation}

O índice varia de zero, quando metade votou \verb|Sim| e metade votou
\verb|Não|, até um, quando todos votaram da mesma forma. Para analisar mais de
uma votação, calculamos a média aritmética dos \gls{IR} das votações no
período, como mostra a fórmula \ref{eq:rice-media}.

\begin{equation}\label{eq:rice-media}
  \sum_{i=1}^{n} \frac{IR_i}{n}
\end{equation}

% TODO: Escrever sobre o problema de só considerar votos Sim ou Não e como
% resolvemos nesse trabalho (talvez só explicitar o problema, pra manter
% fundamentação teórica só como fundamentação teórica).
% Figueiredo/Limongi 1995 só consideram votos Sim e Não, então talvez posso
% dizer que estou seguindo o exemplo deles :)
% ---
% O primeiro problema enfrentado em usar o índice de Rice no contexto brasileiro
% é que ele considera só os indivíduos que votaram, ignorando os ausentes, os que
% se abstiveram e os que obstruíram.

Um problema descrito por Desposato (2005) é que índices de coesão em geral
(o de Rice em particular) são fortemente enviesados. Partidos pequenos têm uma
coesão esperada maior do que partidos grandes, o que invalidaria comparações
entre grupos de tamanhos diferentes, pois diferenças na coesão podem ser
resultado de uma característica do índice usado, e não no comportamento dos
parlamentares. Este problema é especialmente grave no Brasil por existir uma
grande variação no tamanho dos partidos.\nocite{Desposato2005}

Ele propôs uma solução simples: diminuir o tamanho dos partidos.
Especificamente, ele sugere trocar o índice de coesão $C_{ij}$ de cada
partido por $E(C_{ijr})$, que é a coesão esperada de uma amostra de $r$ votos
tomados sem reposição do partido $i$ na votação $j$. O tamanho da amostra $r$
pode variar entre 2 e o tamanho do menor partido, mas ele sugere escolhermos $r
= 2$, que funciona em todos os casos.\nocite{Desposato2005}

Com essa escolha, a interpretação de $E(C_{ij2})$ é simples: é a probabilidade
de que dois membros do partido $i$ escolhidos aleatoriamente votaram juntos na
votação $j$. O cálculo também fica simplificado, sendo feito pela fórmula
\ref{eq:rice-adj}, onde $Y$ é o número de votos ``sim'' e $R$ é o tamanho do
partido.\nocite{Desposato2005}

\begin{equation}\label{eq:rice-adj}
  E(C_{ij2}|Y, R) = \frac{Y(Y - 1) + (R - Y)(R - Y - 1)}{R(R - 1)}
\end{equation}

Caso o índice $C$ já seja conhecido, a fórmula pode ser reduzida a
\ref{eq:adj-known-metric}.\nocite{Desposato2005}

\begin{equation}\label{eq:adj-known-metric}
  E(C_2|Y, R) = \frac{RC^2 + R + 2}{2(R - 1)}
\end{equation}

\subsection{Calculando coesão entre partidos}

Encontramos um problema ao calcular a coesão entre partidos. Caso simplesmente
calculássemos o \gls{IR} entre o conjunto de parlamentares dos partidos
que queremos analisar, estaríamos mostrando um resultado errado caso todos os
partidos não tivessem o mesmo número de parlamentares votando em cada votação.

Por exemplo, considere que queremos calcular a coesão entre o \gls{PT} e o \gls{PSOL} na
Câmara dos Deputados na 54a legislatura. Nesse período, o \gls{PT} contava com XXX
deputados federais, enquanto o \gls{PSOL} contava com 3. Se simplesmente calcularmos
a coesão considerando o conjunto de deputados do \gls{PT} com o do \gls{PSOL}, o índice
seria muito influenciado pela diferença de parlamentares.

Para resolver esse problema, não consideramos o voto de cada parlamentar de
cada partido individualmente, mas sim o voto de um ``parlamentar médio'' do
partido. O voto do parlamentar médio é definido como o voto da maioria do
partido. Dessa forma, conseguimos comparar partidos independente de seus
tamanhos.

Poderíamos ter definido o ``parlamentar médio'' como a indicação do líder do
partido, mas neste caso teríamos que desconsiderar as votações em que o líder
libera sua bancada e não conseguiríamos calcular para partidos cujo líder não
pode indicar voto (de acordo como o Regimento Interno). Dado essas
desvantagens, e o fato que há poucas diferenças entre a indicação do líder e o
voto da maioria \cite{Limongi1995}, preferimos usar o voto da maioria.

\section{Modelos espaciais de votação}



A ideia básica de modelos espaciais de votação é que o conjunto de alternativas
políticas de uma votação pode ser tratado como uma dimensão em um espaço
euclidiano, e cada parlamentar tem preferências de valores nessa dimensão. O
voto seria então definido pela escolha da alternativa mais próxima da sua
preferência.

Há diversos modelos para estimar esses valores, como o NOMINATE, W-NOMINATE,
DW-NOMINATE, que são paramétricos; o Optimal Classification, que é
não-paramétrico, e; modelos baseados em estatística Bayesiana, como o IDEAL.
Nosso foco neste trabalho é no W-NOMINATE.
\cite{Poole2000,Poole2005,Poole2014,Jackman2000,Clinton2004}

Considere um exemplo onde 5 parlamentares votam em
dois projetos de Lei: um propondo a redução da maioridade penal de
18 para 16 anos e outra
propondo o aumento do salário-mínimo de R\$ 1.000 para
R\$ 1.200. Suponha que cada um tem uma única preferência
(\emph{single-peakedness}), conhecido como seu ponto ideal, e vota sinceramente
de acordo com ela. Considere também que as preferências são simétricas. Isto é,
dado duas escolhas a uma mesma distância do ponto ideal de um parlamentar, ele
será indiferente a qualquer uma delas.

Graficamente, temos a figura \ref{fig:modelo-espacial-votacao}, onde os pontos
representam as preferências dos legisladores sobre cada votação, representados
como as dimensões nesse espaço euclidiano. As linhas são chamadas linhas de
corte. Elas passam pelo ponto médio entre as duas alternativas em votação: a de
votar sim e a de votar não. Em outras palavras, se as escolhas são entre uma
maioridade penal de 16 ou
18, a linha de corte passa em $\frac{16 + 18}{2}$,
ou seja, $17$. Ela divide os legisladores que irão votar não, que estão a
esquerda da linha, dos que irão votar sim, que estão a direita da mesma. Caso
esteja em cima da linha de corte, ele é indiferente as alternativas. Caso algum
legislador não siga essa previsão, ele é considerado um ``erro'' do modelo.

\begin{knitrout}
\definecolor{shadecolor}{rgb}{0.969, 0.969, 0.969}\color{fgcolor}\begin{figure}
\includegraphics[width=\maxwidth]{figure/modelo-espacial-votacao-1} \caption[Preferências de 5 deputados em 2 votações com suas respectivas linhas de corte]{Preferências de 5 deputados em 2 votações com suas respectivas linhas de corte}\label{fig:modelo-espacial-votacao}
\end{figure}


\end{knitrout}

Formalmente, sendo $O_{jy}$ e $O_{jn}$ os resultados correspondendo
respectivamente a um voto ``sim'' e um voto ``não'' na votação $j$ ($j = 1,
..., q$), definimos $Z_j$ como: 

\begin{equation}\label{eq:cutpoint}
  Z_j = \frac{O_{jy} + O_{jn}}{2}
\end{equation}

O $Z_j$ pode definir um ponto, linha, plano ou hiperplano, dependendo do número
de dimensões que estamos tratando. Ele é conhecido como ponto (ou linha, etc.)
de corte.

% Introdução

%% O que são de modelos de votação?

%% Por que os modelos espaciais de votação foram criados?

%% Por que não usar os votos diretamente?

% Desenvolvimento

% Conclusão

\subsection{Comparando pontos ideais ao longo do tempo}
\label{cap:fundamentacao:comparando-pontos-ideais-no-tempo}

O principal problema em comparar mudanças nos pontos ideais ao longo do tempo é
distinguir alterações causadas por mudanças na agenda legislativa das causadas
por mudanças no posicionamento dos parlamentares \cite{Bailey2007}. Em outras
palavras, se o ponto ideal de um deputado federal passa de 0.3 para -0.2 de um ano
para o outro, como descobrir se isso representa uma mudança real de ideologia ou
é somente reflexo da diferença na agenda legislativa dos dois períodos?

Segundo \citeonline{Shor2010}, todos os esforços para resolver esse problema
usam ``pontes'', que podem ser parlamentares cujo posicionamento assume-se ter
se mantido estável durante o período de interesse, ou; projetos de Lei que
foram votados em mais de um momento (nas duas casas legislativas em um sistema
bicameral, por exemplo). O primeiro é mais usado para comparar as mudanças nos
posicionamentos dos parlamentares, enquanto o segundo permite unir pessoas que
não votaram juntas em um mesmo mapa espacial\footnote{Por exemplo,
\citeonline{Shor2010} colocam todos os legisladores de 11 estados americanos e
do congresso federal em um período que varia entre 7 e 15 anos, dependendo do
estado, em um mesmo mapa espacial}. \citeonline{Poole2005} propõe duas formas
para estimar pontos ideais usando pontes.

Na primeira, batizada de \emph{pooled scaling} por \citeonline{Shor2010},
dividimos os votos dos parlamentares que queremos mensurar em dois
parlamentares ``virtuais'', um com os votos antes e outro com os votos depois
da data de interesse. Unimos esses parlamentares virtuais com os
parlamentares-ponte, que possuem um registro único, em uma tabela individual e
executamos o algoritmo de estimação dos pontos ideais. Ao final, teremos dois
pontos para cada parlamentar de interesse e um ponto para os pontes. Na
segunda, que \citeonline{Shor2010} chamam de \emph{linear mapping}, estimamos
os pontos ideias separadamente em cada período e os conectamos usando regressão
entre os conjuntos de pontos dos parlamentares-ponte. Ambas formas devem gerar
resultados similares, mas a segunda é computacionalmente mais simples, o que
pode ser essencial, dependendo da quantidade de parlamentares e votações votos
em estudo.

\citeonline{Poole2005} ainda descreve uma terceira forma, similar a
\emph{pooled scaling} descrita acima, que usa para testar se os senadores
norte-americanos mudam de comportamento nos últimos dois anos de seus mandatos,
antes de concorrer à reeleição. Neste caso, ele quer calcular a mudança de
comportamento de todos os legisladores em dois momentos numa mesma legislatura.
Se usássemos o \emph{pooled scaling} diretamente, precisaríamos escolher alguns
parlamentares como pontes que, por definição, não teriam mudado de
comportamento. Ao invés disso, ele segue o seguinte processo:

\begin{enumerate}
  \item Para cada parlamentar, faça:
    \begin{enumerate}
      \item Transforme-o em dois parlamentares ``virtuais'', um com o conjunto
de votos antes, e outro com os depois da data de interesse. Os outros
parlamentares não são modificados;
      \item Calcule os pontos ideias;
      \item Guarde os resultados dos dois parlamentares ``virtuais''. A
diferença entre suas posições representa a mudança de comportamento do
parlamentar.
    \end{enumerate}
  \item Calcule medidas de incerteza para as estimativas.
\end{enumerate}

Ao final, ele tem dois pontos para cada senador: um representando sua posição
nos primeiros 4, e o outro nos últimos 2 anos da legislatura. Note que, como as
matrizes de votações usadas para gerar os mapas espaciais são diferentes, eles
não são estritamente comparáveis. Apesar disso, \citeonline{Poole2005}
argumenta que como elas possuem o mesmo conjunto de votos, com a única
diferença de que um dos parlamentares foi dividido em dois, ele considera ser
seguro compará-las.

\section{Processo legislativo}

Como são criadas as Leis no Brasil? Essa seção pode ser gigantesca, então acho melhor focar no essencial e pontuar as características mais importantes para este trabalho, por exemplo o que são destaques, tipos de votação, tipos de votos, etc.

% ALEXANDRE: Acho que senado e câmara podem ser subseções desta seção.
% Essas subseções não precisam ser muito longas. Devem descrevas basicamente as competências e atribuições de cada casa.
%
% Acho que a justificativa da escolha da Câmara pode aparecer mais tarde, na seção do estudo de caso.  
% Assim a seção de fundamentação teórica fica sendo realmente apenas de fundamentação teórica, sem efeitos colaterais.

\subsection{Senado Federal}

Também chamada de Casa Alta, ela é composta por 81 senadores, 3 por cada Estado e Distrito Federal. Os senadores têm mandato de 8 anos e são eleitos pelo sistema majoritário\footnote{No sistema majoritário, o candidato mais votado é eleito. Nos anos em que são eleitos 2 senadores, os dois candidatos mais votados serão eleitos. Ele pressupõe um (ou dois) único candidatos por partido. É o mesmo sistema usado nas eleições para os chefes do Poder Executivo (presidente da República, governadores e prefeitos). Com exceção da eleição de senadores e prefeitos em cidades com menos de 200 mil habitantes, onde há um único turno, as eleições ocorrem em dois turnos. \cite{Netto2013}}. A renovação da casa é parcial, modificando $\frac{1}{3}$ e $\frac{2}{3}$ dos senadores alternadamente de 4 em 4 anos.

\subsection{Câmara dos Deputados}

Também chamada de Casa Baixa ou Casa do Povo, ela é composta por 513 deputados federais, cujo papel é representar o povo. Cada Estado ou Distrito Federal elege entre 8 e 70 deputados, proporcionalmente a sua população. Os deputados têm mandato de 4 anos, eleitos pelo sistema proporcional\footnote{No sistema proporcional os partidos registram vários candidatos para o mesmo cargo. Os votos recebidos por cada candidato são direcionados ao partido, que precisa atingir uma quantidade mínima de votos (chamado quociente eleitoral) para eleger ao menos um de seus candidatos. Nesse sistema, um candidato com menos votos pode se eleger, enquanto outro com mais votos não se elegeu, dependendo do partido de cada um deles. \cite{Netto2013,Bramatti2014}}. Diferente do Senado Federal, a renovação é total de 4 em 4 anos.

\section{Coesão parlamentar}

Definir o que é coesão e disciplina, e explicar algumas métricas (ou só o índice de Rice)


\section{Ciência de dados}

Texto bem geral sobre data science. Pode ser interessante falar sobre algum algoritmo específico que eu vá usar, ou talvez seja melhor explicá-los no miolo da dissertação.

\section{Considerações Finais}





\chapter{Trabalhos Relacionados}\label{cap:trabalhos-relacionados}


\section{Análise do comportamento parlamentar}
\label{cap:trabalhos-relacionados:analise-comportamento}

No final da década de 80 e na de 90, diversos autores viam o sistema político
brasileiro como sendo altamente instável, com um Poder Executivo fraco,
obrigado a negociar com cada parlamentar individualmente para executar seu
plano de governo. Os parlamentares, preocupados somente com seus interesses
individuais e regionalistas, seriam indisciplinados, levando a
ingovernabilidade do país
\cite{Abranches1988,Lamounier1994,Mainwaring2001,Ames2003}.

Contrários a esse diagnóstico, \citeonline{Limongi1995} descreveram um cenário
muito diferente. O Executivo, por ter domínio sobre a agenda legislativa,
seria capaz de executar seu plano de governo negociando diretamente com os partidos
políticos, que seriam capazes de disciplinar seus parlamentares a votar
conforme suas indicações. Esse diagnóstico foi feito a partir da análise das
votações nominais na Câmara dos Deputados no período de 1989 a 1994. Nessa
análise, os autores calcularam a coesão interna de cada partido, usando o
\gls{IR}\footnote{O \glsfirst{IR} varia de 0 a 100, e é calculado subtraindo a
proporção dos votos minoritários dos majoritários. Ele é 0 quando o partido
está dividido, com metade votando sim e a outra metade votando não, e é 100
quando todos os parlamentares votam da mesma forma \cite{Rice1924}.}. Todos os
partidos apresentaram níveis de coesão relativamente altos, com o menos coeso,
o PDS, atingindo um \gls{IR} 75,70, e o mais coeso, o PT, com \gls{IR} 95,96.

Essas duas correntes foram tão distintas que \citeonline{Power2010} as dividiu
em ``pessimistas'' e ``otimistas''.

Figueiredo e Limongi continuaram analisando dados relacionados ao mesmo tempo,
focando em períodos e características distintos, que confirmaram o prognóstico.
A coletânea desses artigos gerou o livro ``Executivo e Legislativo na nova
ordem constitucional'' \cite{Figueiredo2001}.

Também usando métricas de coesão, \citeonline{Cheibub2009} concluem que o
Congresso brasileiro, ao contrário do previsto pelos pessimistas, é altamente
centralizado, com partidos e seus líderes capazes de disciplinar seus
parlamentares, evitando que estes ajam somente em busca de benefícios para si
mesmo e seus redutos eleitorais.

Além de cientistas políticos, jornalistas também têm usado técnicas semelhantes
para análise das votações. O Grupo Estado, dono do jornal Estado de São Paulo
(o Estadão), lançou em maio de 2012 a ferramenta Basômetro\footnote{Disponível
em \url{http://estadaodados.com/basometro/}.}, um site interativo que permite a
análise de como os parlamentares, individualmente e agregados por partido,
votaram, com dados a partir do primeiro governo de Lula em 2003. Além da
análise de cada votação, o Basômetro também calcula a taxa de
governismo\footnote{A taxa de governismo é calculada como o percentual de vezes
que os parlamentares votaram de acordo com a posição do governo.} dos partidos
e parlamentares, mostrando suas flutuações em cada mês \cite{Estadao2012}.

\citeonline{Dantas2014} organizaram uma coletânea de artigos escritos baseados
em análises feitas com o Basômetro gerando o livro ``Análise política \&
jornalismo de dados''.

Apesar de serem de interpretação e cálculo mais simples, métricas de coesão têm
alguns problemas. Em primeiro lugar, pela distância entre valores não ser
uniforme (por exemplo, a distância entre 40\% e 50\% não é necessariamente
igual a entre 50\% e 60\%), eles só ordenam os parlamentares. Há também um
baixo valor de números possíveis. Em $p$ votações, há $p + 1$ valores
possíveis. Por exemplo, em 3 votações, os parlamentares só podem assumir os
valores 0\%, 33\%, 66\% e 100\%. Por fim, o índice é calculado em relação a
algum ponto específico. Por exemplo, a taxa de governismo é calculada com base
no voto do governo, que normalmente é definido pela indicação ou voto do líder
do governo. Assim, ninguém pode ser mais governista do que o próprio líder, o
que é um pressuposto que precisa ser testado
\cite{Poole2005,McCarty2011,Izumi2013}. Para evitar essas limitações, diversos
autores usam modelos espaciais de votação (ver Seção
\ref{cap:fundamentacao:teoria-espacial-do-voto}).

No Brasil, o primeiro autor a usar esses modelos foi \citeonline{Leoni2002},
que analisou o posicionamento dos deputados federais entre 1991 e 1998 usando o
W-NOMINATE. \citeonline{Desposato2005b} usou o mesmo método para analisar o
efeito da mudança partidária no comportamento dos parlamentares.
\citeonline{Zucco2009} também usa o W-NOMINATE, dessa vez para entender os
fatores que influenciam o comportamento dos parlamentares, encontrando
que a ideologia não explica completamente seus comportamentos, e indicativos
que o presidente da República é um importante influenciador, especialmente
através da distribuição de cargos e recursos.

\citeonline{Freitas2012} analisam o significado da primeira dimensão, estimada
usando W-NOMINATE, na Câmara dos Deputados e no Senado Federal, concluindo que
ela está ligada ao conflito governo e oposição. \citeonline{Izumi2013} repete
essa análise no Senado Federal usando outro modelo, o \emph{Optimal
Classification}, chegando a mesma conclusão.

Além de técnicas específicas para gerar mapas espaciais de votação, alguns
autores usaram técnicas de redução de dimensionalidade mais gerais.
\citeonline{Trento2013} analisaram os dados da Câmara dos Deputados, Senado
Federal e Câmara Municipal de São Paulo usando \gls{ACP}\footnote{Do inglês,
\emph{Principal Component Analysis} (PCA).}, uma técnica estatística de redução
de dimensionalidade. Eles mapeiam um voto sim ou não nos valores 1 e 0, e
transformam o conjunto de votos dos parlamentares em pontos em um espaço
n-dimensional, onde cada dimensão representa uma votação. Como o número de
votações é alto, é difícil visualizar esses pontos. Assim, usam o \gls{ACP}
para colocar esses pontos em um espaço bidimensional, mantendo ao máximo suas
posições relativas. Dessa forma, quanto mais próximo estiverem dois
parlamentares (ou partidos), mais vezes eles votaram da mesma forma. O
resultado dessa pesquisa foi o Radar Parlamentar\footnote{Disponível em
\url{http://radarparlamentar.polignu.org}}, um site interativo no qual é
possível visualizar os diversos gráficos gerados, com as posições dos
legisladores e partidos a cada ano.

Usando uma técnica semelhante, a \gls{MCA}, \citeonline{Andrade2015} colocam os
deputados federais em um espaço bidimensional, mostrando aonde estaria o
presidente da Câmara dos Deputados, Eduardo Cunha (PMDB/RJ). Como ele não vota,
consideraram seu posicionamento como sendo igual ao do líder do PMDB na Câmara
no período. Eles também analisaram a similaridade dos deputados com o Eduardo
Cunha através do percentual de votos que deram iguais a ele, batizando essa
métrica de ``Cunhômetro''.

Apesar dessas técnicas estatísticas como a \gls{ACP} e \gls{MCA} serem
similares a técnicas como o W-NOMINATE, não encontramos trabalhos que validem
seu uso em dados de votação, o que pode dificultar a interpretação dos seus
resultados.

% FIXME: Citar o GovTrack.us e https://opengovdata.io/

\subsection{Análise da mudança de comportamento parlamentar}

O problema básico em comparar mudanças de comportamento ao longo do tempo é
distinguir alterações por causa de uma mudança de agenda das causadas por uma
real mudança das preferências dos parlamentares \cite{Bailey2007}. Em outras
palavras, se em um momento dois parlamentares votaram 90\% das vezes da mesma
forma, e em outro momento eles votaram 50\% das vezes, como definir se essa
mudança se deu porque eles mudaram seus posicionamentos, ou simplesmente porque
eles concordavam nas votações do primeiro momento, mas não nas do segundo?

Segundo \citeonline{Shor2010}, todos os esforços para resolver esse problema
usam ``pontes'', que são parlamentares que estiveram presentes em ambos
momentos e cujo posicionamento se assume ter se mantido estável. Podemos citar
como exemplos parlamentares que foram eleitos para mais de uma legislatura, ou
que mudaram de Casa (deputados federais que se tornam senadores). Projetos de
Lei também podem ser usados como ponte, caso eles tenham sido votados nas
instituições ou períodos de interesse.

No artigo, \citeonline{Shor2010} usam três tipos de pontes para colocar
parlamentares que serviram no nível estadual em ambas as casas\footnote{O
sistema legislativo norte-americano, ao contrário do brasileiro, é bicameral
tanto a nível federal quanto estadual (exceto o estado de Nebrasca, que só
possui um Senado estadual)}, federal em ambas as casas, e no tempo.
Parlamentares que serviram por múltiplas legislaturas tanto a nível estadual
quanto federal servem de ponte entre as legislaturas; legisladores que passam
da casa baixa para a casa alta (ou vice-versa) em nível estadual conectam as
respectivas casas, e; parlamentares que passam do nível estadual para atuar no
nível federal conectam o estado com o Congresso.

No Brasil, \citeonline{Leoni2002} analisa as posições dos deputados federais na
49\textordfeminine{} e 50\textordfeminine{} legislaturas não usando pontes, mas
sim a correlação de Pearson entre os pontos ideais. Ele encontrou correlações
altas, em torno de 0,80. Apesar disso, ele reconhece que uma limitação de seu
trabalho é que, nos períodos estudados, o governo sempre foi de direita, o que
poderia causar essa baixa mudança de pontos ideais. Além disso, como ele não
usou nenhuma técnica para diferenciar mudanças de comportamento de mudanças na
agenda legislativa (como as pontes), não é possível distinguir a razão desse
resultado com segurança.

\citeonline{Desposato2005b}, ao analisar o impacto das mudanças de partido no
comportamento dos parlamentares, usou o posicionamento dos legisladores que não
mudaram de partido como pontes. \citeonline{Izumi2013} faz uma pesquisa
semelhante, mas analisando a mudança de comportamento dos senadores que mudaram
de um partido dentro da coalizão governamental para um fora (ou vice-versa),
para entender o efeito da coalizão no comportamento dos parlamentares.

O Basômetro e o Radar Parlamentar não diferenciam mudanças de comportamento
reais das causadas por mudança da agenda legislativa
\cite{Estadao2012,Trento2013}.

Em geral, trabalhos que usam métricas de coesão interpretam as razões de
mudanças de comportamento usando métodos qualitativos, enquanto os que usam
pontos ideais usam métodos quantitativos.

\section{Aprendizagem de máquina na Ciência Política}
\label{ref:trabalhos-relacionados:data-science-polsci}

Os trabalhos encontrados que usam técnicas de aprendizagem de máquina no
âmbito da Ciência Política dividem-se em duas categorias: os que analisam
textos (como discursos) buscando entender seu conteúdo ou o posicionamento do
autor, e os que analisam projetos de lei tentando prever seus resultados.

Dos que fazem análise textual, \citeonline{Thomas2006} criaram um modelo que
classifica os discursos em debates sobre projetos de lei no congresso
estadunidense como sendo contrários ou favoráveis ao mesmo.
\citeonline{Quinn2006} criaram um modelo que extrai tópicos do texto de
discursos, validando-o nos dados do 105\textordmasculine{} ao
107\textordmasculine{} senado dos Estados Unidos. \citeonline{Yu2008}
determinam a filiação partidária dos parlamentares estadunidense a partir dos
seus discursos. \citeonline{Somasundaran2010} classificam autores de textos
publicados na Internet identificando seu posicionamento político a partir de
suas argumentações. Já \citeonline{Conover2011} fazem o mesmo com os usuários do
Twitter\footnote{Disponível em \url{http://twitter.com}.}, mas a partir da
frequência do uso de certas palavras.

Já na previsão do sucesso de projetos de lei, \citeonline{Gerrish2011} e
\citeonline{Goldblatt2012} desenvolveram modelos que acertaram mais de 90\% dos
resultados nos períodos estudados. \citeonline{Yano2012} fizeram uma análise
semelhante, mas focando em prever os projetos de lei que conseguirão passar das
discussões ocorridas em comissões e ser postos em votação.
\citeonline{Wang2012} preveem os resultados através de uma caminhada aleatória
em 3 grafos heterogêneos interconectados, sendo dois representando os
legisladores e um representando projetos de lei. Os legisladores estão
interligados caso tenham escrito ao menos um projeto juntos\footnote{Do inglês,
\emph{co-sponsorship}.}, e os projetos de lei interligados pela análise da
similaridade textual. Quando um legislador vota em um projeto de lei, um
vértice é criado entre seu nó e o do projeto. Ele criou dois grafos para os
legisladores como forma de criar dois tipos de ligação entre os legisladores e
os projetos de lei: uma para representar votos sim, e outra para votos não.

A empresa americana \citeonline{FiscalNote2015} desenvolve ferramentas para
monitoramento legislativo usando técnicas de ciência de dados. Uma delas, a
\emph{Prophecy}\footnote{Disponível em
\url{https://www.fiscalnote.com/prophecy}.} (do inglês, profecia), supostamente
prevê o resultado de projetos de lei com 94\% de acurácia. No Brasil, o
\gls{ELLO}, empresa ligada ao \gls{CEBRAP}, possui um produto similar.

\section{Considerações Finais}

Como visto na Seção \ref{cap:trabalhos-relacionados:analise-comportamento}, o
foco da literatura encontrada é na análise e interpretação do comportamento dos
parlamentares, seja usando métricas de coesão ou modelos espaciais de votação.
Na Seção \ref{ref:trabalhos-relacionados:data-science-polsci}, foram
apresentados trabalhos que aplicam técnicas de aprendizagem de máquina em temas
da ciência política, buscando descobrir o posicionamento de pessoas através dos
seus textos ou prever o resultado de votações.

Nesse levantamento bibliográfico, não foi encontrado nenhum autor que
desenvolva o objetivo deste trabalho: a detecção de mudanças de posicionamento
dos legisladores. O que parece ser mais próximo é a pesquisa feita pela empresa
americana \citeonline{FiscalNote2015} para o desenvolvimento do produto
\emph{Prophecy}, mas não foi possível analisar as semelhanças mais a fundo pois
o processo usado por eles não é divulgado.

Nesse sentido, considero a definição da metodologia para o desenvolvimento de
um modelo de detecção de mudança de posicionamento dos deputados federais
brasileiros como sendo o principal diferencial e contribuição deste trabalho.




\chapter{Miolo da sua dissertação}\label{cap:miolo}

Nessa parte pretendo falar sobre a ferramenta desenvolvida, mostrar sua arquitetura e como cada parte funciona e pode ser modificada.

\section{Os dados}

Para deesenvolver o modelo preditivo, precisamos de dois conjuntos de dados: os
resultados das votações nominais na Câmara dos Deputados e informações sobre as
coalizões governamentais. Nessa seção descreverei o processo de extração,
transformação e carga\footnote{Do inglês, \footnote{extract, transform, load}
(ETL)} desses dados, desde suas fontes até estarem prontos para uso no modelo.

\subsection{Extração}
\label{sec:miolo:extracao}

O \gls{CEBRAP} reúne diversos dados sobre o legislativo brasileiro. Para este
trabalho, extraímos a lista das coalizões governamentais do Brasil.
\citeonline{Figueiredo2007} se inspirou no trabalho de \citeonline{Muller2000}
para definir os critérios de início de uma nova coalizão. Eles são: i) a
mudança da legislatura, e; ii) alterações no conjunto de partidos que controlam
os ministérios. Com base nisso e pesquisando em fontes oficiais e
jornalísticas, \citeauthor{Figueiredo2007} listou as coalizões formadas
durante as duas últimas constituições democráticas, nos períodos de 1946 até
1964, e de 1988 até os dias atuais.

De posse desses dados, também precisamos do resultado das votações nominais
ocorridas na Câmara Federal. O banco de dados do \gls{CEBRAP} possui essas
informações, mas como o modelo desenvolvido foi pensado para ser usado
continuamente, monitorando o comportamento dos deputados federais, preferimos
buscá-los diretamente da fonte.

A Câmara disponibiliza diversos conjuntos de dados em seu sua página na
web\footnote{\url{http://www2.camara.leg.br/transparencia/dados-abertos}}.
Dentre eles está o histórico dos resultados das votações nominais que
precisamos. Infelizmente, até a data de escrita deste trabalho, esses dados só
estavam disponíveis para consulta individualmente através de uma \gls{API}, mas
precisávamos do conjunto completo para análise local.  Isto nos obrigou a
desenvolver um software que busca, uma a uma, as votações ocorridas no nosso
período de interesse.

Esse software\footnote{Disponível em:
\url{https://github.com/vitorbaptista/dados-abertos-camara.gov.br}}, conhecido
como \emph{crawler} (do inglês, ``raspador''), usa os métodos
\emph{ListarProposicoesVotadasEmPlenario}, que a partir de um ano retorna a
lista de proposições votadas, e \emph{ObterVotacaoProposicao}, que retorna
todas as votações ocorridas em uma proposição a partir do seu tipo, número e
ano. Ele foi desenvolvido em Python usando a biblioteca Scrapy
\cite{Python276,Scrapy}. Além de baixar os dados, ele também converte a lista
de proposições votadas em um \gls{CSV} e as votações por proposição em um
\gls{JSON}, a partir do original em \gls{XML}, pois mais ferramentas suportam
esses formatos.

\subsection{Transformação}
\label{sec:miolo:transformacao}

Usando outro programa também desenvolvido em Python\footnote{Disponível em:
\url{https://github.com/vitorbaptista/codigo-mestrado}}, dividimos a
transformação em duas etapas. Primeiramente, a partir dos arquivos gerados na
fase anterior, fazemos:

\begin{itemize}
  \item Retiramos espaços em branco no começo ou final dos campos;
  \item Normalizamos os nomes dos partidos seguindo o padrão adotado pelo
\gls{CEBRAP}, onde mudanças de nome são identificados pela primeira e última
sigla (ex.: PJ se torna PRN que se torna PJC, então temos PJ>PTC) e fusões são
identificados pelo maior partido na data da fusão e o novo nome (ex.: PL e
PRONA se tornam PL>PR) \cite{Freitas2008}. A lista completa está no apêndice
\ref{apendice:lista-partidos}.
\end{itemize}

O resultado dessa etapa é guardado em um banco de dados SQLite com o auxílio da
biblioteca SQLAlchemy
\cite{SQLite3,SQLAlchemy}. A partir daí, geramos dois arquivos \gls{CSV} por
legislatura: um contendo os dados dos parlamentares e seus votos (tabela
\ref{table:votes}) e outro contendo os detalhes das votações (tabela
\ref{table:votes-metadata}).

\begin{table}
\centering
\begin{knitrout}
\definecolor{shadecolor}{rgb}{0.969, 0.969, 0.969}\color{fgcolor}
\begin{tabular}{r|l|l|l|r|r|r|r|r}
\hline
id & name & party & state & 76 & 273 & 271 & 272 & 485\\
\hline
3151 & Jairo Ataíde & PFL>DEM & MG & NA & 0 & NA & 0 & NA\\
\hline
4929 & Joseph Bandeira & PT & BA & NA & NA & NA & NA & NA\\
\hline
4930 & Silvio Costa & PSC & PE & 1 & 0 & 0 & 0 & 0\\
\hline
4931 & Izalci & PL+PRONA>PR & DF & 1 & 1 & 0 & 0 & 0\\
\hline
62881 & Danilo Forte & PMDB & CE & 1 & 1 & 0 & 0 & NA\\
\hline
67138 & Iracema Portella & PDS>PP & PI & 1 & 1 & 0 & 0 & 0\\
\hline
\end{tabular}


\end{knitrout}
\caption{Exemplo da tabela com detalhes dos parlamentares e seus respectivos votos.}
\label{table:votes}
\end{table}

\begin{landscape}
\begin{table}
\centering
\begin{knitrout}
\definecolor{shadecolor}{rgb}{0.969, 0.969, 0.969}\color{fgcolor}
\begin{tabular}{r|r|r|l|l|l}
\hline
id & id\_sessao & proposicao\_id & data & resumo & obj\_votacao\\
\hline
76 & 4212 & 491974 & 2011-02-15 21:09:00 & Aprovado o Requerime... & URGÊNCIA PARA O PL 3...\\
\hline
273 & 4215 & 491872 & 2011-02-16 00:17:00 & Mantida a expressão.... & DVS - Bloco PvPps - ...\\
\hline
271 & 4214 & 491872 & 2011-02-16 22:38:00 & Rejeitada a Emenda d... & DVS - PSDB - EMENDA ...\\
\hline
272 & 4214 & 491872 & 2011-02-16 23:27:00 & Rejeitada a Emenda d... & DVS - DEM - EMENDA N...\\
\hline
485 & 4219 & 485274 & 2011-02-22 17:36:00 & Rejeitado o Requerim... & REQUERIMENTO DE RETI...\\
\hline
486 & 4219 & 485274 & 2011-02-22 19:20:00 & Rejeitado o Requerim... & REQUERIMENTO DE ADIA...\\
\hline
487 & 4220 & 485274 & 2011-02-22 20:45:00 & Rejeitado o Requerim... & REQUERIMENTO DE RETI...\\
\hline
955 & 4221 & 485275 & 2011-02-23 17:01:00 & Rejeitado o Requerim... & REQUERIMENTO DE RETI...\\
\hline
1042 & 4251 & 485758 & 2011-04-05 17:02:00 & Rejeitado o Requerim... & REQUERIMENTO DE RETI...\\
\hline
1043 & 4251 & 485758 & 2011-04-05 19:11:00 & Rejeitado o Requerim... & REQUERIMENTO DE CONC...\\
\hline
\end{tabular}


\end{knitrout}
\caption{Exemplo da tabela com detalhes das votações. Os campos ``resumo'' e
``obj\_votacao'' foram limitados a 20 caracteres por
questões de formatação da página.}
\label{table:votes-metadata}
\end{table}
\end{landscape}

Esses arquivos, por sua vez, são usados juntamente com os dados das coalizões
advindos do banco de dados legislativos do \gls{CEBRAP} em um programa escrito
em R que faz 

\begin{algorithm}[H]
 \Entrada{Votações, votos e lista de parlamentares que entraram ou saíram da coalizão}
 \Saida{Variáveis dependentes e independentes}
 initialization\;
 $resultado \longleftarrow ()$\;
 $legislaturas \longleftarrow (50, 51, 52, 53, 54)$\;
 \ParaCada{$legislatura$ em $legislaturas$}{
   \Para{$i \longleftarrow 100, 110, ..., numero\_de\_votacoes_{\text{legislatura}}$}{
     $votos \longleftarrow votos_{\text{legislatura}}(1..i)$\;
     $indice\_votacao\_media \longleftarrow floor(i / 2)$\;
     \ParaCada{$parlamentar$ em $votos$}{
       divida os votos do $parlamentar$ de $1..indice\_votacao\_media$ e $indice\_votacao\_media..i$\;
       execute o $wnominate$ usando $votos$ 10 vezes para gerar os erros-padrão\;
       guarde os dois resultados do $parlamentar$\;
     }
   }
 }
 \caption{How to write algorithms}
\end{algorithm}

% \begin{listing}[ht]
% \begin{minted}{xml}
% <proposicoes>
%   <proposicao>
%     <codProposicao>19319</codProposicao>
%     <nomeProposicao>PL 3232/1992</nomeProposicao>
%     <dataVotacao>10/06/2014</dataVotacao>
%   </proposicao>
%   <proposicao>
%     <codProposicao>43617</codProposicao>
%     <nomeProposicao>PLP 275/2001</nomeProposicao>
%     <dataVotacao>22/04/2014</dataVotacao>
%   </proposicao>
%   <!-- Continua... -->
% </proposicoes>
% \end{minted}
% \caption{Resultado da API \emph{ListarProposicoesVotadasEmPlenario} usando o ano 2014 como parâmetro}
% \end{listing}

Restringimos a analisar da 50\textordfeminine{} até a 54\textordfeminine{}
legislatura, que compreende um período de 20 anos, entre o início do primeiro
governo de Fernando Henrique Cardoso em 1995 até o final do primeiro governo de
Dilma Rousseff em 2015. Esse recorte foi feito pois é quando, de acordo com
\cite{Freitas2008}, termina a fase de acomodação dos parlamentares as regras
definidas pela Constituição de 1988.

\begin{table}
\centering
\begin{knitrout}
\definecolor{shadecolor}{rgb}{0.969, 0.969, 0.969}\color{fgcolor}
\begin{tabular}{r|r|r|r}
\hline
Legislatura & Deputados Federais & Votações & Votos\\
\hline
50 & 631 & 468 & 178603\\
\hline
51 & 624 & 419 & 155737\\
\hline
52 & 614 & 451 & 134461\\
\hline
53 & 606 & 619 & 192879\\
\hline
54 & 644 & 432 & 131552\\
\hline
\end{tabular}


\end{knitrout}
\caption{Número de deputados federais e votações por legislatura}
\label{table:estatisticas-legislaturas}
\end{table}

Na tabela \ref{table:estatisticas-legislaturas}, mostramos o número de
deputados federais e votações por legislatura. Consideramos deputados federais
todos parlamentares que participaram de alguma votação na Câmara dos Deputados
no período, por isso o número é maior do que o número de cadeiras. Por
legislatura, temos em média 623,80 deputados e 477,80
votações com 332,04 votos cada.

\section{Construção do modelo}

% TODO: Preciso falar sobre a variável de coalizão, de onde ela veio, como é
% definida e como a usamos. Também preciso falar do período de estudo (100
% votações).

\subsection{Variável independente}

A variável independente indica se o parlamentar entrou ou saiu da coalizão do
governo no período de estudo. Ela pode tomar os valores \verb|N|, caso ele
tenha se mantido no governo ou oposição, ou \verb|S| caso tenha trocado de
lado. Um favor importante é como definimos o período de estudo. Em outras
palavras, suponha que 

\section{Componentes}

Mostrar em linhas gerais como é organizada a ferramenta, indo de um nível de abstração mais alto (como o usuário vê a ferramenta) até uma arquitetura técnica.


% ALEXANDRE: Apesar de eu não ser muito fã acho importante apresentar algum tipo de diagrama de componentes nesta seção, para permitir uma visualização da arquitetura da solução.


\subsection{Extração dos dados}

Falar sobre as fontes de dados (câmara dos deputados, banco de dados legislativos do Cebrap, etc.), a forma de extração, passos de limpeza, banco de dados, frequência de atualização, se é pull ou push, e como disponibilizamos esses dados para as outras etapas. Ao término dessa seção, o leitor deverá entender nosso processo de data wrangling e saber como ele próprio pode, se quiser, baixar os dados ao final dessa etapa e fazer suas próprias análises. Talvez tenha problema com isso, pois o pessoal da CEBRAP pode não permitir a redistribuição de seus dados.


% ALEXANDRE: É interessante destacar que disponibilizar estes dados em um formato mais "amigável" também é uma contribuição do seu trabalho.

\subsection{Análise}

Aqui falarei sobre os algoritmos usados e as métricas escolhidas (coesão parlamentar, número de destaques, etc.). Também é bom frisar que a arquitetura permite adicionar novos algoritmos, talvez até de uma forma reativa. Por exemplo, considerando que a acusação que o PP trancou a pauta para pressionar o Lula a nomear uma pessoa na Petrobrás, poderíamos criar um algoritmo que detectasse trancamentos de pauta por um partido qualquer, para no futuro descobrirmos movimentações parecidas.

\subsection{Notificações/Relatórios}

Falar sobre como os usuários consomem esses dados. Minha ideia atual é gerar um relatório semanal/mensal/quando-aparecer-algo-interessante e enviá-lo por email pra quem se cadastrar. Nesse caso, devo mostrar exemplos desse relatório.


(Alexandre) Seria legal considerar também a criação de uma conta que twitaria periódicamente os relatórios.


\section{Considerações Finais}

Desenvolvi uma ferramenta que extrai dados de diversas fontes, os consolida, analisa e gera relatórios para os usuários. Cada parte pode ser modificada e expandida, e os dados (e algoritmos) podem ser usados por outras pessoas. Daqui falta validar que os algoritmos na parte de análise geram algo interessante.
\chapter{Estudos de Caso}\label{cap:avalia}

Aqui irei escolher um (ou mais) relatórios gerados pela ferramenta e analisar seu conteúdo. Por exemplo, se me falaram que um projeto de Lei é interessante (anômalo, etc.), confirmar se ele realmente é.

Vejo duas formas de fazer isso. Uma é considerando só o relatório, que foi gerado a partir de algoritmos que analisam diversos dados diferentes. Por exemplo, um pode ver modificações na coesão, outro o número de destaques em um PL, outro a nomeação de ministros. Ou seja, estou testando a ferramenta como um todo, mas não cada algoritmo/métrica. Pode ser que analisar nomeação de ministros não dê resultados interessantes, mas não conseguirei ver isso a partir do relatório.

Outra forma seria analisar o resultado de cada algoritmo. Assim, o estudo de caso seria executar um algoritmo em um período qualquer e ver que resultados ele me dá. Estaria fazendo um estudo de caso para cada algoritmo. Assim, justifico melhor a escolha dos algoritmos, mas em contrapartida estou fazendo uma análise de um ponto de vista que os usuários do sistema não verão. Não é uma visão "holística" de "o sistema me dá bons resultados", mas sim "esse algoritmo específico me dá bons resultados".

Não sei se consegui explicar direito.

Talvez faça mais sentido estudos de caso se focarem no sistema como um todo, mas eu faria uma validação de cada algoritmo em algum outro lugar, analisando se ele dá resultados interessantes. Não sei onde isso entraria.

(Alexandre) Essa parte da análise dos algoritmos é muito interessante sim mas não seria um estudo de caso. Acho que ela entraria melhor no capítulo anterior como uma análise experimental sobre os algoritmos a serem utilizados na ferramenta.

Os estudos de caso seriam mais amplos e focados na ferramenta como um todo como você mesmo descreveu.



\section{Estudo de Caso}

\subsection{Ferramentas e Tecnologia}
\subsection{Requisitos}
\subsection{Desenvolvimento}
\subsection{Avaliação}

\section{Experimento}

\subsection{Plano do Experimento}
\subsection{Execução do Experimento}
\subsection{Análise do Experimento}

\section{Considerações Finais}



\chapter{Conclusão}\label{cap:conclusao}



%%%%%%%%%%%%%%%%%%%%%%%%%%%%%%%%%%%%%%%%%%%%%%%%%%%%%%%%%%%%%%%%%%%%%%%%%%%%%%%%
%% BIbliografia
%% Coloque suas referencias no arquivo ref.bib

\bibliographystyle{abnt-alf} % estilo de bibliografia   plain,unsrt,alpha,abbrv.
\bibliography{ref} % arquivos com as entradas bib.

%Faz aparecer a referencia bibliografica no indice
\addcontentsline{toc}{section}{\numberline{}Referências Bibliográficas}

%%%%%%%%%%%%%%%%%%%%%%%%%%%%%%%%%%%%%%%%%%%%%%%%%%%%%%%%%%%%%%%%%%%%%%%%%%%%%%%%
%% Apendice
% Caso seja necessario algum apendice

\appendix
% TODO: Modificar o texto para não ficar igual a dissertação da Andréa e checar
% se faltam alguns partidos.

\chapter{Lista dos partidos}
\label{apendice:lista-partidos}

\begin{table}
\centering
\begin{tabular}{l l l l}
  Partido & Sigla agregada & Sigla atual & Nome atual \\
  \hline
  PFL>DEM & PFL>DEM & DEM & Democratas \\
  PCdoB & PCdoB & PCdoB & Partido Comunista do Brasil \\
  PDT & PDT & PDT & Partido Democrático Trabalhista \\
  PHS & PHS & PHS & Partido Humanista da Solidariedade \\
  PMDB & PMDB & PMDB & Partido do Movimento Democrático Brasileiro \\
  PMN & PMN & PMN & Partido da Mobilização Nacional \\
  PDS>PP\footnote{O PDS fundiu-se com o PDC em 1993, passando a chamar-se
Partido Progressista Reformador (PPR). Nova fusão com o PP em 21/09/1995 e nova
nomeclatura: Partido Progressista Brasileiro (PPB). Passa a denominar-se
Partido Progressista (PP) em 2003.} & PDS>PPR>PPB>PP & PP & Partido Progressista \\
  PCB>PPS\footnote{O Partido Comunista Brasileiro (PCB) mudou de nome para
Partido Popular Socialista (PPS) em 1991.} & PCB>PPS & PPS & Partido Popular Socialista \\
  PL>PR\footnote{Fusão do Partido Liberal (PL) com o PRONA em 2006, gestando o
Partido da República (PR).} & PL>PR & PR & Partido da República \\
  PMR>PRB\footnote{O Partido Municipalista Renovador (PMR) mudou a nomeclatura
para Partido Republicano Brasileiro (PRB) em 2006.} & PMR>PRB & PRB & Partido Republicano Brasileiro \\
  PRONA & PRONA & PRONA & Partido da Reedificação da Ordem Nacional \\
  PRTB & PRTB & PRTB & Partido Renovador Trabalhista Brasileiro \\
  PSB & PSB & PSB & Partido Socialista Brasileiro \\
  PSC & PSC & PSC & Partido Social Cristão \\
  PSD & PSD & PSD & Partido Social Democrático \\
  PSDB & PSDB & PSDB & Partido da Social Democracia Brasileira \\
  PSDC & PSDC & PSDC & Partido Social Democrata Cristão \\
  PSL & PSL & PSL & Partido Social Trabalhista \\
  PSOL & PSOL & PSOL & Partido Socialismo e Liberdade \\
  PST & PST & PST & Partido Social Trabalhista \\
  PSTU & PSTU & PSTU & Partido Socialista dos Trabalhadores Unificado \\
  PT & PT & PT & Partido dos Trabalhadores \\
  PTB & PTB & PTB & Partido Trabalhista Brasileiro \\
  PJ>PTC\footnote{O Partido da Juventude (PJ), fundado em 1985, mudou de nome
em 1989 para Partido da Reconstrução Nacional (PRN). Em 2001 passou a chamar-se
Partido Trabalhista Cristão (PTC)} & PJ>PRN>PTC & PTC & Partido Trabalhista Cristão \\
  PTN & PTN & PTN & Partido Trabalhista Nacional \\
  PV & PV & PV & Partido Verde \\
  SD & SD & SD & Solidariedade \\
\end{tabular}
\end{table}




\chapter{A saída do PSB da coalizão em Dilma I}\label{cap:analise-saida-psb}

\section{Introdução}

O PSB esteve na base de apoio do governo do PT desde que o partido conquistou a
Presidência da República com Lula em 2003. Dez anos depois, em 2013, penúltimo
ano do primeiro governo de Dilma, o PSB anuncia a candidatura de Eduardo Campos
para a presidência, rompendo a aliança com o PT.

Para entender a repercussão dessa mudança na Câmara dos Deputados, analisaremos
as votações nominais ocorridas naquela legislatura (54\textordfeminine, que
durou de 2011 até 2015) seguindo a metodologia proposta por Keith Poole em
\emph{Spatial models of parliamentary voting} \cite{Poole2005}.
% Não estamos entendendo a repercussão da mudança na CD como um todo, mas sim
% só no comportamento de votação dos parlamentares.

\section{Metodologia}{\label{sec:analise-saida-psb:metodologia}

Para entender a influência da saída do PSB no comportamento de votação dos
parlamentares, precisamos de uma forma de comparar o comportamento deles antes
e depois dessa mudança. Este é um problema complicado, estudado por diversos
autores (CITATION NEEDED). Nesta análise seguimos a metodologia de
\cite{Poole2005}.

Considerando um parlamentar por vez, o substituímos por dois parlamentares
``virtuais'' na tabela de votações, um com os votos antes da mudança e outro
com os votos depois dela, e executamos o algoritmo W-NOMINATE nessa nova
tabela. Esse procedimento é repetido para cada parlamentar separadamente,
enquanto mantém todos os outros sem modificações. Ao final, teremos dois pontos
para cada parlamentar: um para antes e outro para depois da mudança. A
diferença entre esses pontos representa o nível da mudança de comportamento no
período de análise. 

Por exemplo, considere a tabela \ref{table:exemplo-mudanca-de-comportamento}
contendo 3 parlamentares e 4 votações. Para definir qual foi a mudança de
comportamento de Samara entre as votações 2 e 3, dividimos seus votos em dois
parlamentares ``virtuais'', Samara 1 e Samara 2 (ver tabela
\ref{table:exemplo-parlamentar-virtual}), e executamos o algoritmo W-NOMINATE.
Guardamos os resultados de Samara e repetimos os mesmos passos para Pedro e
depois Maria. Ao final, teremos dois pontos para cada parlamentar: um relativo
a sua posição antes, e outro depois da mudança (ver figura
\ref{fig:exemplo-mudanca-de-comportamento}). A distância desses dois pontos é
uma medida da intensidade da mudança do parlamentar.  Comparando as distâncias
de cada parlamentar, podemos determinar se a mudança é significante ou não.

% TODO: Falar sobre o parametric bootstrap
% TODO: Falar sobre os problemas em comparar esses valores.
% \begin{quote}
%   ``(...) all matrices have the same number of nonmissing entries, and they
%   differ only in the fact that each has a different senator divided into two
%   records. Consequently, I believe it's safe to capture the 72 distances with
%   each other, because the differences between the configurations will be
%   trivial.''\cite{Poole2005}
% \end{quote}

\begin{table}
  \begin{minipage}{\textwidth}
    \centering
    \begin{tabular}{ l l l l l }
      nome & votação1 & votação2 & votação3 & votação4 \\
      \hline
      Samara & Sim & Sim & Não & Não \\
      Pedro & Sim & Sim & Sim & Sim \\
      Maria & Não & Não & Não & Não \\
    \end{tabular}
    \caption{Dados originais de votação}
    \label{table:exemplo-mudanca-de-comportamento}
  \end{minipage}
  \begin{minipage}{\textwidth}
    \centering
    \begin{tabular}{ l l l l l }
      nome & votação1 & votação2 & votação3 & votação4 \\
      \hline
      Samara 1 & Sim & Sim & & \\
      Samara 2 & & & Não & Não \\
      Pedro & Sim & Sim & Sim & Sim \\
      Maria & Não & Não & Não & Não \\
    \end{tabular}
    \caption{Dados de votação com Romário dividido no meio}
    \label{table:exemplo-parlamentar-virtual}
  \end{minipage}
\end{table}

\begin{knitrout}
\definecolor{shadecolor}{rgb}{0.969, 0.969, 0.969}\color{fgcolor}\begin{figure}
\includegraphics[width=\maxwidth]{figure/exemplo-mudanca-de-comportamento-1} \caption[Posições dos parlamentares antes e depois da data em análise]{Posições dos parlamentares antes e depois da data em análise}\label{fig:exemplo-mudanca-de-comportamento}
\end{figure}


\end{knitrout}

% Pseudocódigo
%
% \begin{enumerate}
%   \item Para cada parlamentar, faça:
%   \begin{enumerate}
%     \item Crie dois parlamentares ``virtuais'', um com os votos do parlamentar
%       antes da mudança, e o outro com os votos depois dela;
%     \item Execute o W-NOMINATE na matriz completa de votações com todos os
%       parlamentares e o parlamentar em análise substituído pelos parlamentares
%       ``virtuais'';
%     \item Guarde o resultado;
%   \end{enumerate}
%   \item Compara a mudança de cada parlamentar de interesse com todos os outros,
%     para descobrir se ela (a mudança) foi significativa.
% \end{enumerate}

\section{Extração dos dados}



Os dados das votações nominais foram extraídos a partir da API disponibilizada
no site da Câmara dos Deputados. Eles compreendem 131.552 votos
proferidos por 644 parlamentares em 432
votações. A figura \ref{fig:votacoes-por-mes} mostra a distribuição das
votações no período e a figura \ref{fig:deputados-por-partido} mostra o número
de deputados federais por partido. Só consideramos votos \verb|Sim| ou
\verb|Não|.  

\begin{knitrout}
\definecolor{shadecolor}{rgb}{0.969, 0.969, 0.969}\color{fgcolor}\begin{figure}
\includegraphics[width=\maxwidth]{figure/votacoes-por-mes-1} \caption[Distribuição de votações por mês na 54\textordfeminine{} legislatura]{Distribuição de votações por mês na 54\textordfeminine{} legislatura}\label{fig:votacoes-por-mes}
\end{figure}


\end{knitrout}

\begin{knitrout}
\definecolor{shadecolor}{rgb}{0.969, 0.969, 0.969}\color{fgcolor}\begin{figure}
\includegraphics[width=\maxwidth]{figure/deputados-por-partido-1} \caption[Distribuição de Deputados Federais por partido na 54\textordfeminine{} legislatura]{Distribuição de Deputados Federais por partido na 54\textordfeminine{} legislatura}\label{fig:deputados-por-partido}
\end{figure}


\end{knitrout}

\section{Análise}



Seguindo a metodologia descrita na sessão
\ref{sec:analise-saida-psb:metodologia}, dividimos a
54\textordfeminine{} legislatura na votação mais próxima da data de interesse,
no nosso caso, a data em que o PSB deixa a coalizão do governo. De acordo com
os dados disponíveis no banco de dados do CEBRAP, isso ocorreu em 03/10/2013. A
primeira votação depois dessa data ocorreu em
08/10/2013 (identificador
367), relacionada ao MPV 621/2013. Houve
303 votações antes da ruptura e
133 votações após. Dos 643
parlamentares que votaram ao menos uma vez nesse período,
470 passaram pelo nosso filtro de
participação em ao menos 20 votações cuja minoria teve 2,5\% ou mais dos votos.

A figura \ref{fig:boxplot-diff-coalizao} mostra a distribuição da diferença
entre as posições dos parlamentares antes e depois da saída do PSB da base do
governo, agrupado pelos que entraram ou saíram da coalizão e os que mantiveram
sua posição. Nela podemos ver que, no geral, quem entra na coalizão se torna
mais governista, quem sai se torna mais oposicionista, e os outros mantém sua
posição, mas existem diversos \emph{outliers}.

\begin{knitrout}
\definecolor{shadecolor}{rgb}{0.969, 0.969, 0.969}\color{fgcolor}\begin{figure}
\includegraphics[width=\maxwidth]{figure/boxplot-diff-coalizao-1} \caption[Distribuição do nível de mudança da posição dos deputados federais antes e depois da saída do PSB da coalizão do governo]{Distribuição do nível de mudança da posição dos deputados federais antes e depois da saída do PSB da coalizão do governo}\label{fig:boxplot-diff-coalizao}
\end{figure}


\end{knitrout}

Na tabela \ref{table:top-10-diffs} estão os dez parlamentares com maior mudança
de posição em valores absolutos. Desses, analisemos os que não entraram nem
saíram da coalizão: 

\begin{description}
\item[Dudimar Paxiuba] Apesar de continuar fora da coalizão, ele saiu do PSDB
para o PROS em 2013, o que explicaria essa aproximação ao
governo\footnote{\url{http://www2.camara.leg.br/deputados/pesquisa/layouts_deputados_biografia?pk=193033&tipo=0}}.
\item[Rebecca Garcia] Não consegui explicar. A única coisa que encontrei foi
que ela ia se candidatar a prefeitura de Manaus em 2012, mas desistiu porque
vazou um áudio de uma conversa dela com seu amante na época, o ex-vereador Ari
Moutinho. Não me parece ter relação.
\item[Vicente Candido]Também não encontrei uma razão.
\item[Dr. Grilo]Saiu do PSL para o Solidariedade, que pode explicar o
afastamento do governo.
\end{description}

\begin{table}
\centering
\begin{knitrout}
\definecolor{shadecolor}{rgb}{0.969, 0.969, 0.969}\color{fgcolor}
\begin{tabular}{l|l|l|r|r|r}
\hline
Nome & Partido & Coalizão & Antes & Depois & Diferença\\
\hline
Wladimir Costa & PMDB & saiu & 0.32 & -0.60 & -0.92\\
\hline
Dudimar Paxiuba & PROS & nao\_mudou & -0.71 & 0.17 & 0.87\\
\hline
Luiz Nishimori & PL+PRONA>PR & entrou & -0.86 & -0.08 & 0.77\\
\hline
Domingos Dutra & PT & saiu & 0.52 & -0.16 & -0.68\\
\hline
Romário & PSB & saiu & 0.28 & -0.37 & -0.65\\
\hline
Genecias Noronha & PMDB & saiu & -0.01 & -0.64 & -0.63\\
\hline
Rebecca Garcia & PDS>PP & nao\_mudou & 0.39 & -0.23 & -0.62\\
\hline
Antonio Balhmann & PROS & saiu & 0.08 & 0.69 & 0.61\\
\hline
Vicente Candido & PT & nao\_mudou & 0.99 & 0.39 & -0.60\\
\hline
Dr. Grilo & PSL & nao\_mudou & 0.05 & -0.52 & -0.57\\
\hline
\end{tabular}


\end{knitrout}
\caption{Os dez parlamentares com maior diferença absoluta entre suas posições
antes e depois da saída do PSB da base do governo.}
\label{table:top-10-diffs}
\end{table}

\subsection{Os deputados do PSB mudaram de comportamento?}



De acordo com a figura \ref{fig:mudanca-psb}, no geral os deputados do PSB se
afastaram do governo, enquanto a maioria dos parlamentares dos outros partidos
mantiveram suas posições. Há um \emph{outlier} nos deputados do PSB que, ao
contrário da maioria, se aproximou do governo: Alexandre Toledo, que
passou de -0,61 para -0,10 (diferença
de 0,51). Analisando este caso, a razão se torna clara: ele
era do PSDB e passou para o PSB em 2013. Apesar do PSB ter se tornado oposição,
passando de uma posição mediana -0,01
para -0,33, o PSDB é um opositor mais
ferrenho ao governo, com posição mediana -0,87.

Com base nesses dados, podemos concluir que o que já esperávamos: o PSB se
afastou do governo.

\begin{knitrout}
\definecolor{shadecolor}{rgb}{0.969, 0.969, 0.969}\color{fgcolor}\begin{figure}
\includegraphics[width=\maxwidth]{figure/mudanca-psb-1} \caption[Distribuição do nível de mudança da posição dos deputados federais antes e depois da saída do PSB da coalizão do governo]{Distribuição do nível de mudança da posição dos deputados federais antes e depois da saída do PSB da coalizão do governo}\label{fig:mudanca-psb}
\end{figure}


\end{knitrout}

\section{Modelagem}

Nosso objetivo agora é determinar se, baseado somente nas posições dos
deputados federais de antes e depois da saída do PSB da coalizão, conseguimos
inferir se um parlamentar mudou de lado (se era governo, se tornou oposição, ou
vice-versa). Para isso, treinaremos um modelo \emph{Random Forest}.

O primeiro passo é alterar a coluna ``Coalizão'' para somente dois valores: sim
ou não. Nesse momento, nosso objetivo é determinar se o parlamentar mudou de
lado, e não pra que lado ele foi\footnote{Sabendo se o parlamentar mudou de
lado e qual era sua posição anterior, podemos inferir qual é sua nova posição
(e.x.: se ele era oposição e mudou de lado, agora virou governo).}. Também
retiraremos informações como nome e partido e parlamentares cuja posição não
conseguimos estimar pois não houve votos o suficiente, e adicionamos o
desvio-padrão de cada valor\footnote{Calculados usando parametric bootstrap.
TODO: Explicar melhor}. No final, teremos uma tabela como a
\ref{table:clean-data}.

\begin{table}
\centering
\begin{knitrout}
\definecolor{shadecolor}{rgb}{0.969, 0.969, 0.969}\color{fgcolor}
\begin{tabular}{r|r|r|r|r|l}
\hline
Antes & Antes SD & Depois & Depois SD & Diferença & Mudou de posição\\
\hline
0.32 & 0.17 & -0.60 & 0.16 & -0.92 & Sim\\
\hline
-0.71 & 0.12 & 0.17 & 0.08 & 0.87 & Não\\
\hline
-0.86 & 0.10 & -0.08 & 0.19 & 0.77 & Sim\\
\hline
0.52 & 0.13 & -0.16 & 0.13 & -0.68 & Sim\\
\hline
0.28 & 0.10 & -0.37 & 0.20 & -0.65 & Sim\\
\hline
-0.01 & 0.12 & -0.64 & 0.15 & -0.63 & Sim\\
\hline
\end{tabular}


\end{knitrout}
\caption{Os dez parlamentares com maior diferença absoluta entre suas posições
antes e depois da saída do PSB da base do governo.}
\label{table:clean-data}
\end{table}

Com os dados nesse formato, agora precisamos dividí-lo em dois grupos: um para
treino e um para teste. Precisamos fazer isso para que as estimativas de
precisão do nosso modelo sejam confiáveis. Escolhemos manter 80\% para treino e
20\% para testes. Também usamos o \emph{10-fold stratified cross validation}.

\begin{knitrout}
\definecolor{shadecolor}{rgb}{0.969, 0.969, 0.969}\color{fgcolor}\begin{kframe}
\begin{verbatim}
## note: only 30 unique complexity parameters in default grid. Truncating the grid to 30 .
\end{verbatim}
\end{kframe}
\end{knitrout}

O modelo foi treinado com todos os atributos na tabela \ref{table:clean-data} e
a interação entre eles. Analisamos diversos números de preditores e, como pode
ser visto na figura \ref{fig:roc-vs-mtry}, o melhor modelo com relação ao área
sob a curva ROC usou 20 preditores e chegou a área de
0,86 com desvio-padrão de 0,06.

\begin{knitrout}
\definecolor{shadecolor}{rgb}{0.969, 0.969, 0.969}\color{fgcolor}\begin{figure}
\includegraphics[width=\maxwidth]{figure/roc-vs-mtry-1} \caption[ROC como função do número de preditores]{ROC como função do número de preditores.}\label{fig:roc-vs-mtry}
\end{figure}


\end{knitrout}

Usando os dados de teste, podemos avaliar nosso modelo. A figura \ref{fig:roc}
mostra a curva ROC. Ela mostra a sensibilidade e especificidade do nosso modelo
para diversos pontos de corte. O valor marcado é o ponto ``ideal'', definido
como o ponto mais próximo do canto superior esquerdo do gráfico (sensibilidade
e especificidade iguais a 1). Na tabela \ref{table:roc} podemos ver todos
máximos locais dos pontos de corte e as respectivas sensibilidades e
especificidades, com mínimos e máximos dentro de margem de confiança de
95\%\footnote{Calculado a partir de 2.000 stratified bootstrap replications.
TODO: explicar melhor}.

\begin{knitrout}
\definecolor{shadecolor}{rgb}{0.969, 0.969, 0.969}\color{fgcolor}\begin{figure}
\includegraphics[width=\maxwidth]{figure/roc-1} \caption[Curva ROC com ponto ``ideal'' marcado]{Curva ROC com ponto ``ideal'' marcado.}\label{fig:roc}
\end{figure}


\end{knitrout}

\begin{table}
\centering
\begin{knitrout}
\definecolor{shadecolor}{rgb}{0.969, 0.969, 0.969}\color{fgcolor}
\begin{tabular}{l|r|r|r|r|r|r}
\hline
cutpoint & sens.low & sens.median & sens.high & spec.low & spec.median & spec.high\\
\hline
0.755 & 0.00 & 0.20 & 0.40 & 0.91 & 0.96 & 1.00\\
\hline
0.592 & 0.07 & 0.27 & 0.53 & 0.87 & 0.94 & 0.99\\
\hline
0.449 & 0.13 & 0.33 & 0.60 & 0.83 & 0.90 & 0.96\\
\hline
0.362 & 0.20 & 0.47 & 0.73 & 0.77 & 0.85 & 0.91\\
\hline
0.321 & 0.27 & 0.53 & 0.80 & 0.76 & 0.83 & 0.91\\
\hline
0.281 & 0.33 & 0.60 & 0.87 & 0.73 & 0.82 & 0.90\\
\hline
0.256 & 0.40 & 0.67 & 0.87 & 0.72 & 0.81 & 0.88\\
\hline
0.237 & 0.53 & 0.73 & 0.93 & 0.68 & 0.78 & 0.87\\
\hline
0.165 & 0.60 & 0.80 & 1.00 & 0.62 & 0.72 & 0.81\\
\hline
0.143 & 0.67 & 0.87 & 1.00 & 0.55 & 0.65 & 0.76\\
\hline
0.093 & 0.80 & 0.93 & 1.00 & 0.47 & 0.59 & 0.69\\
\hline
0.057 & 1.00 & 1.00 & 1.00 & 0.41 & 0.53 & 0.64\\
\hline
\end{tabular}


\end{knitrout}
\caption{Especificidade e sensibilidade nos máximos locais da curva ROC}
\label{table:roc}
\end{table}

\chapter{Versões dos softwares utilizados}

\begin{itemize}\raggedright
  \item R version 3.2.1 (2015-06-18), \verb|x86_64-pc-linux-gnu|
  \item Locale: \verb|LC_CTYPE=en_US.UTF-8|, \verb|LC_NUMERIC=C|, \verb|LC_TIME=pt_BR.UTF-8|, \verb|LC_COLLATE=en_US.UTF-8|, \verb|LC_MONETARY=pt_BR.UTF-8|, \verb|LC_MESSAGES=en_US.UTF-8|, \verb|LC_PAPER=pt_BR.UTF-8|, \verb|LC_NAME=C|, \verb|LC_ADDRESS=C|, \verb|LC_TELEPHONE=C|, \verb|LC_MEASUREMENT=pt_BR.UTF-8|, \verb|LC_IDENTIFICATION=C|
  \item Base packages: base, datasets, graphics, grDevices,
    methods, parallel, stats, utils
  \item Other packages: caret~6.0-47, data.table~1.9.4,
    doParallel~1.0.8, foreach~1.4.2, ggplot2~1.0.1,
    iterators~1.0.7, knitr~1.9, lattice~0.20-31, plyr~1.8.2,
    pROC~1.8, randomForest~4.6-10
  \item Loaded via a namespace (and not attached):
    BradleyTerry2~1.0-6, brglm~0.5-9, car~2.0-25, chron~2.3-45,
    codetools~0.2-11, colorspace~1.2-4, digest~0.6.8,
    evaluate~0.5.5, formatR~1.2, grid~3.2.1, gtable~0.1.2,
    gtools~3.4.2, highr~0.5, labeling~0.3, lme4~1.1-7,
    MASS~7.3-42, Matrix~1.2-2, mgcv~1.8-6, minqa~1.2.4,
    munsell~0.4.2, nlme~3.1-121, nloptr~1.0.4, nnet~7.3-10,
    pbkrtest~0.4-2, proto~0.3-10, quantreg~5.11, Rcpp~0.11.4,
    reshape2~1.4.1, scales~0.2.4, SparseM~1.6, splines~3.2.1,
    stringr~0.6.2, tools~3.2.1
\end{itemize}


%\chapter{Apêndice B}}\label{apendice-b}

%\chapter{Apêndice C}}\label{apendice-c}

%\chapter{Apêndice D}}\label{apendice-d}

%%%%%%%%%%%%%%%%%%%%%%%%%%%%%%%%%%%%%%%%%%%%%%%%%%%%%%%%%%%%%%%%%%%%%%%%%%%%%%%%

\end{document}
