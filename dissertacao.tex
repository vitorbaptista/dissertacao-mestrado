%%%%%%%%%%%%%%%%%%%%%%%%%%%%%%%%%%%%%%%%%%%%%%%%%%%%%%%%%%%%%%%%%%%%%%%%%%%%%%%%%%%%%%%%%%%
%% Ultimas modificacoes, 06/02/2012 - Alexandre Duarte 
%% Baseado no modelo latex de Isaac Maia (COPIN/UFCG)
%%
%% Para utilizar ese modelo sao necessarios os seguintes arquivos:
%%
%% ppgi.cls
%% ppgi.sty
%% mestre.sty
%%
%%
%% Mais detalhes sobre normas ABNT no latex, consultar http://abntex.codigolivre.org.br
%% Wiki interessante com dicas uteis sobre latex : http://www.tex-br.org
%%
%%
%% Para compilar esse arquivo, e' sempre importante fazer duas passagens com latex
%%%%
%%%%%%%%%%%%%%%%%%%%%%%%%%%%%%%%%%%%%%%%%%%%%%%%%%%%%%%%%%%%%%%%%%%%%%%%%%%%%%%%%%%%%%%%%%%

\documentclass[a4paper,titlepage]{ppgi}
\usepackage[brazilian,english]{babel}
\usepackage{ppgi,mestre,epsfig}
\usepackage{times}
\usepackage[final]{pdfpages}
\usepackage{hyperref}
\hypersetup{
    bookmarks=true,   
    pdftitle={Monitor legislativo},
    pdfauthor={Vitor Márcio Paiva de Sousa Baptista}, 
    pdfsubject={Modelo de Documento Científico},
    pdfkeywords={Dissertação, Mestrado, PPGI, UFPB, modelo}, 
    colorlinks=true,
    linkcolor=black,
    citecolor=black,
    filecolor=black,
    urlcolor=black
 }


%-------------------------- Para usar acentuacaoo em sistemas ISO8859-1 ------------------------------------
% Se estiver usando o Microsoft Windows ou linux com essa codificacao, descomente essas linhas abaixo
% e comente as linhas referentes ao UTF8
%\usepackage[applemac]{inputenc} % Usar acentuacao em sistemas ISO8859-1, comentar a linha com  \usepackage[utf8x {inputenc}
%-----------------------------------------------------------------------------------------------------

%-------------------------- Para usar acentuacao em sistemas UTF8 ------------------------------------
% Para a maior parte das distribuicoes linux, usar essa opcao
\usepackage{ucs}
\usepackage[utf8x]{inputenc}
\usepackage[T1]{fontenc}
%-----------------------------------------------------------------------------------------------------
\usepackage{float}      
\usepackage{fancyvrb}
\usepackage{fancyheadings}
\usepackage{graphicx}
\usepackage{longtable} %tabelas longas, para tabelas que ultrapassam uma pagina
%\input{psfig.sty}
% ----------------- Para inserir codigo fonte de linguagens de programacao no documento -------------
\usepackage{listings}
\lstset{numbers=left,
stepnumber=1,
firstnumber=1,
numberstyle=\scriptsize,
extendedchars=true,
breaklines=true,
frame=tb,
basicstyle=\scriptsize,
stringstyle=\ttfamily,
showstringspaces=false
}
\renewcommand{\lstlistingname}{C\'odigo Fonte}
\renewcommand{\lstlistlistingname}{Lista de C\'odigos Fonte}

% ---------------------------------------------------------------------------------------------------

\selectlanguage{brazilian}
\sloppy



\begin{document}


%%%%%%%%%%%%%%%%%%%%%%%%%%%%%%%%%%%%%%%%%%%%%%%%%%%%%%%%%%%%%%%%%%%%%%%%%%%%%%%%
\Titulo{Monitor legislativo}
\Autor{Vitor Márcio Paiva de Sousa Baptista}
\Data{06 de Março de 2012}
\Area{Ciência da Computação}
\Pesquisa{Computação Distribuída | Sinais, Sistemas Digitais e Gráficos}
\Orientadores{Alexandre Nóbrega Duarte\\ (Orientador)}

\newpage
\cleardoublepage
\PaginadeRosto

\newpage
\cleardoublepage

%%%%%%%%%%%%%%%%%%%%%%%%%%%%%%%%%%%%%%%%%%%%%%%%%%%%%%%%%%%%%%%%%%%%%%%%%%%%%%%%
\begin{resumo} 
Vestibulum varius accumsan odio malesuada gravida. Duis a erat et arcu tincidunt semper sed et quam. Sed mattis semper quam vel imperdiet. Etiam tortor orci, ullamcorper ac aliquam eu, interdum quis justo. Morbi lacinia ligula ac nibh imperdiet semper. Aliquam varius tristique nisl, in blandit tellus ultrices et. Nullam est nisl, pretium sit amet vehicula quis, cursus at enim.
\\
\\
\textbf{Palavras-chave:} Palavras, chave, para, seu, trabalho.

\end{resumo}
%\newpage
%\cleardoublepage

%%%%%%%%%%%%%%%%%%%%%%%%%%%%%%%%%%%%%%%%%%%%%%%%%%%%%%%%%%%%%%%%%%%%%%%%%%%%%%%%
\begin{summary}
In Brazil, there are tools for monitoring the behaviour of legislators in
rollcalls, such as O Estado de São Paulo's Basômetro and Radar Parlamentar.
These tools are used both by journalists and political scientists for analysis.

Although they are great analysis tools, their usefulness for monitoring is
limited because they require a manual follow-up, which makes it a lot of work
when we consider the volume of data. Only in the Chamber of Deputies, 513
legislators participate on average over than 400 rollcalls by legislature. It
is possible to decrease the amount of data analyzing the parties as a whole,
but in contrast we lose the ability to detect individuals' drives or
intra-party groups such as factions.

In order to mitigate this problem, I developed a statistical model that detects
when a legislator changes his or her position, joining or leaving the
governmental coalition, through ideal points estimates using the W-NOMINATE. It
can be used individually or integrated to tools such as Basômetro, providing a
filter for researchers find the deputies who changed their behaviour most
significantly.

The universe of study is composed of legislators from the Chamber of Deputies
from the 50th to the 54th legislatures, starting in the first term of Fernando
Henrique Cardoso in 1995 until the end of the first term of Dilma Rousseff in
2015.

\textbf{Keywords:} Legislative Analysis, Political Science, Data Science,
Predictive Models, Machine Learning.

\end{summary}

%\newpage
%\cleardoublepage

%%%%%%%%%%%%%%%%%%%%%%%%%%%%%%%%%%%%%%%%%%%%%%%%%%%%%%%%%%%%%%%%%%%%%%%%%%%%%%%%
% TMP: Agradecimentos
\begin{agradecimentos}
Donec ultricies elit a quam ornare posuere. Pellentesque eu tortor massa. Aliquam erat volutpat. In vitae justo dolor, ac fringilla nisl. In hac habitasse platea dictumst. Pellentesque placerat eleifend sem, in tempor nisl elementum fermentum. Ut in metus vitae magna volutpat viverra. Suspendisse ac dolor velit, in volutpat magna. Cras blandit urna quis diam feugiat volutpat. Nunc mattis lobortis libero varius posuere. Integer sem augue, aliquet fringilla porta nec, adipiscing sed ante. Aenean feugiat, eros non vehicula pretium, neque purus vehicula diam, eu vulputate leo neque nec velit. Vestibulum at orci quam, et mattis tortor. Donec iaculis orci enim.

\end{agradecimentos}

\clearpage

%%%%%%%%%%%%%%%%%%%%%%%%%%%%%%%%%%%%%%%%%%%%%%%%%%%%%%%%%%%%%%%%%%%%%%%%%%%%%%%%
%% Definicao do cabecalho: secao do lado esquerdo e numero da pagina do lado direito
\pagestyle{fancy}
\addtolength{\headwidth}{\marginparsep}\addtolength{\headwidth}{\marginparwidth}\headwidth = \textwidth
\renewcommand{\chaptermark}[1]{\markboth{#1}{}}
\renewcommand{\sectionmark}[1]{\markright{\thesection\ #1}}\lhead[\fancyplain{}{\bfseries\thepage}]%
	     {\fancyplain{}{\emph{\rightmark}}}\rhead[\fancyplain{}{\bfseries\leftmark}]%
             {\fancyplain{}{\bfseries\thepage}}\cfoot{}

%%%%%%%%%%%%%%%%%%%%%%%%%%%%%%%%%%%%%%%%%%%%%%%%%%%%%%%%%%%%%%%%%%%%%%%%%%%%%%%%
\selectlanguage{brazilian}

\Sumario
\ListadeSimbolos
\listoffigures
\listoftables
\lstlistoflistings %lista de codigos fonte - Para inserir a listagem de
% codigos fonte

\newpage
\cleardoublepage
\Introducao


%%%%%%%%%%%%%%%%%%%%%%%%%%%%%%%%%%%%%%%%%%%%%%%%%%%%%%%%%%%%%%%%%%%%%%%%%%%%%%%%
%
% Hifenizacao - Colocar lista de palavras que nao devem ser separadas e que 
% nao estao no dicionario brazilian.
% As palavras do dicionario brazilian ja sao separadas corretamente pelo lateX
%
\hyphenation{ gLite OurGrid GridDoctor }


%%%%%%%%%%%%%%%%%%%%%%%%%%%%%%%%%%%%%%%%%%%%%%%%%%%%%%%%%%%%%%%%%%%%%%%%%%%%%%%%
%% A partir daqui coloque seus capitulos. Sugere-se que eles sejam inseridos com o comando \input
%% Da seguinte maneira:
%% 
\chapter{Introdução} \label{intro}

\section{Motivação}\label{sec:motiva}

A quantidade de dados relacionados a atividade dos parlamentares brasileiros é muito grande. Exemplos simples são as votações, presenças em plenário, projetos de Lei, viagens, candidaturas, etc. Analisar manualmente esse volume de dados é uma tarefa bastante difícil e por isso podem haver situações importantes que não sejam percebidas. Visando auxiliar no processo de acompanhamento e fiscalização da atividade dos parlamentares brasileiros este trabalho propõe uma ferramenta de monitoramento automático e contínuo que gera alertas e relatórios sobre o que acontece nas casas legislativas brasileiras, usando a Câmara dos Deputados como estudo de caso.

\section{Objetivos}

O objetivo geral desta dissertação é o desenvolvimento e validação de um modelo
capaz de prever quando um deputado federal irá deixar ou entrar na coalizão do
governo a partir das votações nominais.

\subsection{Objetivos Especificos}

\begin{itemize}
\item Desenvolver um mecanismo capaz de detectar mudanças no padrão de comportamento dos parlamentares;
\item Desenvolver um mecanismo capaz de detectar mudanças semelhantes entre grupos de parlamentares (ex.: João e Pedro sempre divergiram em seus votos e de repente passaram a concordar);
\item Desenvolver um mecanismo capaz de detectar votações importantes/polêmicas (falta definir o que são importante e polêmica)
\end{itemize}

\section{Metodologia}

Não sei qual o nome da metodologia que estou usando: por enquanto é tudo muito ad-hoc. Estudo de caso é um componente, mas não sei se há outra.

Alexandre -  Não é preciso dar uma nome a metodologia. O que você precisa descrever aqui são os passos seguidos até chegar ao resultado final.

Tipo
1 - Levantamento Bibliográfico sobre X e Y
2 - Desenvolviment do mecanismo para extração e filtragem dos dados
3 - ...

\section{Publicações Relacionadas}

Mencionar o artigo que você publicou no BRASNAM

\section{Estrutura da Dissertação}

Descrever a estrutura dos demais capítulos. Melhor fazer isso só no final.

\chapter{Fundamentação Teórica}\label{cap:fundamentacao}

\section{Processo legislativo}

Como são criadas as Leis no Brasil? Essa seção pode ser gigantesca, então acho melhor focar no essencial e pontuar as características mais importantes para este trabalho, por exemplo o que são destaques, tipos de votação, tipos de votos, etc.



(Alexandre) Acho que senado e câmara podem ser subseções desta seção.
Essas subseções não precisam ser muito longas. Devem descrevas basicamente as competências e atribuições de cada casa.

Acho que a justificativa da escolha da Câmara pode aparecer mais tarde, na seção do estudo de caso.  
Assim a seção de fundamentação teórica fica sendo realmente apenas de fundamentação teórica, sem efeitos colaterais.

\subsection{Senado Federal}


\subsection{Câmara dos Deputados}

Escrever sobre as competências da Câmara dos Deputados e justificar porque usamos ela no estudo de caso (talvez essa justificativa caiba em outro lugar?)


\section{Coesão parlamentar}

Definir o que é coesão e disciplina, e explicar algumas métricas (ou só o índice de Rice)


\section{Ciência de dados}

Texto bem geral sobre data science. Pode ser interessante falar sobre algum algoritmo específico que eu vá usar, ou talvez seja melhor explicá-los no miolo da dissertação.

\section{Considerações Finais}






\chapter{Trabalhos Relacionados}\label{cap:relacionados}


\section{Análise de mudança de comportamento de parlamentares}

O problema básico em comparar mudanças de comportamento ao longo do tempo é
distinguir alterações por causa de uma mudança de agenda das causadas por uma
real mudança das preferências dos parlamentares \cite{Bailey2007}. Em outras
palavras, se em um momento dois parlamentares \emph{A} e \emph{B} votaram 90\%
das vezes da mesma forma, e em outro momento eles votaram 50\% das vezes, como
definir se essa mudança se deu porque eles mudaram seus posicionamentos, ou
simplesmente porque eles concordavam nas votações do primeiro momento, mas não
nas do segundo?

Segundo \citeonline{Shor2010}, todos os esforços para resolver esse problema
usam ``pontes'', que são parlamentares que estiveram presentes em ambos
momentos e cujo posicionamento se assume ter se mantido estável. Podemos citar
como exemplos parlamentares que foram eleitos para mais de uma legislatura, ou
que mudaram de Casa (deputados federais que se tornam senadores). Projetos de
Lei também podem ser usados como ponte, caso eles tenham sido votados nas
instituições ou períodos de interesse.

No artigo, \citeonline{Shor2010} usa três tipos de pontes para colocar
parlamentares que serviram no nível estadual em ambas as casas\footnote{O
sistema legislativo norte-americano, ao contrário do brasileiro, é bicameral
tanto a nível federal quanto estadual (exceto o estado de Nebrasca, que só
possui um Senado estadual)}, federal em ambas as casas, e no tempo.
Parlamentares que serviram por múltiplas legislaturas tanto a nível estadual
quanto federal servem de ponte entre as legislaturas; legisladores que passam
da casa baixa para a casa alta (ou vice-versa) em nível estadual conectam as
respectivas casas, e; parlamentares que passam do nível estadual para atuar no
nível federal conectam o estado com o Congresso.

Os trabalhos se dividem em dois grandes grupos: os que usam medidas de coesão e
os que usam modelos espaciais de votação.

% Dividem-se em dois grupos: os que usam métodos espaciais e os que usam
% métodos "normais" (qual o nome?).
% Radar Parlamentar
% Basômetro
% TODO: O livro do Poole "Congress: A Political-Economic History of Roll Call
% Voting" parece importante demais pra estar de fora.

\citeonline{Desposato2005b} analisou os efeitos da mudança de partidos no
comportamento dos senadores e deputados federais brasileiros durante a
49\textordfeminine{} e 50\textordfeminine{} legislaturas. Ele usou o W-NOMINATE
para estimar os pontos ideais de cara par parlamentar-partido. Ou seja, se o
parlamentar \emph{A} mudar do PT para o DEM, ele terá dois pontos ideais: um
para cada período.

\citeonline{Leoni2002} analisou o comportamento dos partidos políticos na
Câmara dos Deputados entre 1991 e 1998 usando a correlação de Pearson. Ele
estava buscando entender se os deputados federais mantinham suas posições
relativas ao longo do tempo, e qual influência o presidente da República tinha
em alterar essas posições. Nesse período, ele não encontrou uma influência
estatisticamente relevante dos presidentes na composição da Câmara, mas há o
porém que todos os presidentes estudados eram de direita. Isso pode ter
influenciado seu resultado.

\section{Coesão parlamentar}

A coesão entre parlamentares é estudada há décadas pela Ciência Política. Em 1924, Stewart Rice propôs a primeira métrica para medir coesão: o Índice de Rice \cite{Rice1924}. Depois dele, vários outros pesquisadores proporam novas métricas, como XXX, YYY, ZZZ. Apesar disso, o IR continua sendo muito usado.

No Brasil, podemos citar \citeonline{Figueiredo1995} que analisa o padrão de
votação dos parlamentares da Câmara dos Deputados entre 1989 e 1994;
\citeonline{Neto1997}, que analisa a mesma casa mas entre 1946 e 1964, e;
\citeonline{Neiva2011}, que analisa votações do Senado entre 1989 a 2009. Todos utilizam o Índice de Rice.

\subsection{Discussão}

Explicar que apesar de existirem várias métricas de coesão, a mais usada ainda é a de Rice, mas que usarei ela com algumas modificações como normalizá-lo para manter o mesmo índice independente do número de parlamentares considerados no cálculo. Existe um artigo que explica isso (só preciso encontrá-lo)

Explicar também que, apesar das limitações de analisar a coesão através das votações nominais (muito limitado, ignora o trabalho dos bastidores), ela é uma forma simples e direta de analisar um grande volume de dados, e ainda é muito usada.

\section{Ciência de dados}

Listarei trabalhos sobre ciência de dados, especialmente métodos de detecção de anomalias. Acho que não cabe aqui, mas a parte de sistemas de monitoramento de servidores e a sacada de usá-los nessa outra área é interessante. Também a criação de um pipeline, possivelmente usando o Luigi (\url{https://github.com/spotify/luigi}).


(Alexandre) Essa parte sobre a parte de sistemas de monitoramento entra em outro capítulo e concordo que deve ser mencionada sim na dissertação. Acho que isso pode aparecer no próximo capítulo.


\subsection{Discussão}

\section{Considerações Finais}

Esse trabalho não trás muitas novidades especificamente nas áreas relacionadas, mas a contribuição está na junção dessas ferramentas já existentes. Nas minhas pesquisas, só encontrei o a startup \url{www.fiscalnote.com} que faz um trabalho parecido, mas tenho certeza que existem outras empresas, ou sistemas internos. O que não encontrei é um monitorador legislativo que não foque em uma área ou projeto de Lei específico, mas busque padrões gerais. O que encontrei mais parecido é o monitoramento de servidores, que segue o padrão de analisar o maior número possível de dados buscando anomalias e padrões.


(Alexandre) Então esse deve ser o principal diferencial do seu trabalho em relação aos demais. Essa diferença deve ser reforçada ao apresentar os trabalhos relacionados.

Um negócio que fica muito legal nesta seção é criar uma tabela com um conjunto de características presentes e desejadas e marcar quais trabalhos relacionados atendem a cada uma dessas características.

Assim fica mais fácil constatar onde está o grande diferencial do seu trabalho, que seria justamente ser uma solução com aplicação mais ampla e não focada em áreas específicas.

\chapter{Miolo da sua dissertação}\label{cap:miolo}

Nessa parte pretendo falar sobre a ferramenta desenvolvida, mostrar sua arquitetura e como cada parte funciona e pode ser modificada.

\section{Componentes}

Mostrar em linhas gerais como é organizada a ferramenta, indo de um nível de abstração mais alto (como o usuário vê a ferramenta) até uma arquitetura técnica.


(Alexandre) Apesar de eu não ser muito fã acho importante apresentar algum tipo de diagrama de componentes nesta seção, para permitir uma visualização da arquitetura da solução.


\subsection{Extração dos dados}

Falar sobre as fontes de dados (câmara dos deputados, banco de dados legislativos do Cebrap, etc.), a forma de extração, passos de limpeza, banco de dados, frequência de atualização, se é pull ou push, e como disponibilizamos esses dados para as outras etapas. Ao término dessa seção, o leitor deverá entender nosso processo de data wrangling e saber como ele próprio pode, se quiser, baixar os dados ao final dessa etapa e fazer suas próprias análises. Talvez tenha problema com isso, pois o pessoal da CEBRAP pode não permitir a redistribuição de seus dados.



(Alexandre) É interessante destacar que disponibilizar estes dados em um formato mais "amigável" também é uma contribuição do seu trabalho.

\subsection{Análise}

Aqui falarei sobre os algoritmos usados e as métricas escolhidas (coesão parlamentar, número de destaques, etc.). Também é bom frisar que a arquitetura permite adicionar novos algoritmos, talvez até de uma forma reativa. Por exemplo, considerando que a acusação que o PP trancou a pauta para pressionar o Lula a nomear uma pessoa na Petrobrás, poderíamos criar um algoritmo que detectasse trancamentos de pauta por um partido qualquer, para no futuro descobrirmos movimentações parecidas.

\subsection{Notificações/Relatórios}

Falar sobre como os usuários consomem esses dados. Minha ideia atual é gerar um relatório semanal/mensal/quando-aparecer-algo-interessante e enviá-lo por email pra quem se cadastrar. Nesse caso, devo mostrar exemplos desse relatório.


(Alexandre) Seria legal considerar também a criação de uma conta que twitaria periódicamente os relatórios.


\section{Considerações Finais}

Desenvolvi uma ferramenta que extrai dados de diversas fontes, os consolida, analisa e gera relatórios para os usuários. Cada parte pode ser modificada e expandida, e os dados (e algoritmos) podem ser usados por outras pessoas. Daqui falta validar que os algoritmos na parte de análise geram algo interessante.
\chapter{Estudos de Caso}\label{cap:avalia}

Aqui irei escolher um (ou mais) relatórios gerados pela ferramenta e analisar seu conteúdo. Por exemplo, se me falaram que um projeto de Lei é interessante (anômalo, etc.), confirmar se ele realmente é.

Vejo duas formas de fazer isso. Uma é considerando só o relatório, que foi gerado a partir de algoritmos que analisam diversos dados diferentes. Por exemplo, um pode ver modificações na coesão, outro o número de destaques em um PL, outro a nomeação de ministros. Ou seja, estou testando a ferramenta como um todo, mas não cada algoritmo/métrica. Pode ser que analisar nomeação de ministros não dê resultados interessantes, mas não conseguirei ver isso a partir do relatório.

Outra forma seria analisar o resultado de cada algoritmo. Assim, o estudo de caso seria executar um algoritmo em um período qualquer e ver que resultados ele me dá. Estaria fazendo um estudo de caso para cada algoritmo. Assim, justifico melhor a escolha dos algoritmos, mas em contrapartida estou fazendo uma análise de um ponto de vista que os usuários do sistema não verão. Não é uma visão "holística" de "o sistema me dá bons resultados", mas sim "esse algoritmo específico me dá bons resultados".

Não sei se consegui explicar direito.

Talvez faça mais sentido estudos de caso se focarem no sistema como um todo, mas eu faria uma validação de cada algoritmo em algum outro lugar, analisando se ele dá resultados interessantes. Não sei onde isso entraria.

(Alexandre) Essa parte da análise dos algoritmos é muito interessante sim mas não seria um estudo de caso. Acho que ela entraria melhor no capítulo anterior como uma análise experimental sobre os algoritmos a serem utilizados na ferramenta.

Os estudos de caso seriam mais amplos e focados na ferramenta como um todo como você mesmo descreveu.



\section{Estudo de Caso}

\subsection{Ferramentas e Tecnologia}
\subsection{Requisitos}
\subsection{Desenvolvimento}
\subsection{Avaliação}

\section{Experimento}

\subsection{Plano do Experimento}
\subsection{Execução do Experimento}
\subsection{Análise do Experimento}

\section{Considerações Finais}



\chapter{Conclusão}\label{cap:conclusao}



%%%%%%%%%%%%%%%%%%%%%%%%%%%%%%%%%%%%%%%%%%%%%%%%%%%%%%%%%%%%%%%%%%%%%%%%%%%%%%%%
%% BIbliografia
%% Coloque suas referencias no arquivo ref.bib

\bibliographystyle{abnt-alf} % estilo de bibliografia   plain,unsrt,alpha,abbrv.
\bibliography{ref} % arquivos com as entradas bib.

%Faz aparecer a referencia bibliografica no indice
\addcontentsline{toc}{section}{\numberline{}Referências Bibliográficas}

%%%%%%%%%%%%%%%%%%%%%%%%%%%%%%%%%%%%%%%%%%%%%%%%%%%%%%%%%%%%%%%%%%%%%%%%%%%%%%%%
%% Apendice
% Caso seja necessario algum apendice

\appendix
%\chapter{Apêndice A}}\label{apendice-a}

%\input{apendice-b}
%\chapter{Apêndice C}}\label{apendice-c}

%\chapter{Apêndice D}}\label{apendice-d}

%%%%%%%%%%%%%%%%%%%%%%%%%%%%%%%%%%%%%%%%%%%%%%%%%%%%%%%%%%%%%%%%%%%%%%%%%%%%%%%%

\end{document}
