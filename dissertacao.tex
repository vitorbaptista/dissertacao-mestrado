%%%%%%%%%%%%%%%%%%%%%%%%%%%%%%%%%%%%%%%%%%%%%%%%%%%%%%%%%%%%%%%%%%%%%%%%%%%%%%%%%%%%%%%%%%%
%% Ultimas modificacoes, 06/02/2012 - Alexandre Duarte 
%% Baseado no modelo latex de Isaac Maia (COPIN/UFCG)
%%
%% Para utilizar ese modelo sao necessarios os seguintes arquivos:
%%
%% ppgi.cls
%% ppgi.sty
%% mestre.sty
%%
%%
%% Mais detalhes sobre normas ABNT no latex, consultar http://abntex.codigolivre.org.br
%% Wiki interessante com dicas uteis sobre latex : http://www.tex-br.org
%%
%%
%% Para compilar esse arquivo, e' sempre importante fazer duas passagens com latex
%%%%
%%%%%%%%%%%%%%%%%%%%%%%%%%%%%%%%%%%%%%%%%%%%%%%%%%%%%%%%%%%%%%%%%%%%%%%%%%%%%%%%%%%%%%%%%%%

\documentclass[a4paper,titlepage]{ppgi}\usepackage[]{graphicx}\usepackage[]{color}
%% maxwidth is the original width if it is less than linewidth
%% otherwise use linewidth (to make sure the graphics do not exceed the margin)
\makeatletter
\def\maxwidth{ %
  \ifdim\Gin@nat@width>\linewidth
    \linewidth
  \else
    \Gin@nat@width
  \fi
}
\makeatother

\definecolor{fgcolor}{rgb}{0.345, 0.345, 0.345}
\newcommand{\hlnum}[1]{\textcolor[rgb]{0.686,0.059,0.569}{#1}}%
\newcommand{\hlstr}[1]{\textcolor[rgb]{0.192,0.494,0.8}{#1}}%
\newcommand{\hlcom}[1]{\textcolor[rgb]{0.678,0.584,0.686}{\textit{#1}}}%
\newcommand{\hlopt}[1]{\textcolor[rgb]{0,0,0}{#1}}%
\newcommand{\hlstd}[1]{\textcolor[rgb]{0.345,0.345,0.345}{#1}}%
\newcommand{\hlkwa}[1]{\textcolor[rgb]{0.161,0.373,0.58}{\textbf{#1}}}%
\newcommand{\hlkwb}[1]{\textcolor[rgb]{0.69,0.353,0.396}{#1}}%
\newcommand{\hlkwc}[1]{\textcolor[rgb]{0.333,0.667,0.333}{#1}}%
\newcommand{\hlkwd}[1]{\textcolor[rgb]{0.737,0.353,0.396}{\textbf{#1}}}%

\usepackage{framed}
\makeatletter
\newenvironment{kframe}{%
 \def\at@end@of@kframe{}%
 \ifinner\ifhmode%
  \def\at@end@of@kframe{\end{minipage}}%
  \begin{minipage}{\columnwidth}%
 \fi\fi%
 \def\FrameCommand##1{\hskip\@totalleftmargin \hskip-\fboxsep
 \colorbox{shadecolor}{##1}\hskip-\fboxsep
     % There is no \\@totalrightmargin, so:
     \hskip-\linewidth \hskip-\@totalleftmargin \hskip\columnwidth}%
 \MakeFramed {\advance\hsize-\width
   \@totalleftmargin\z@ \linewidth\hsize
   \@setminipage}}%
 {\par\unskip\endMakeFramed%
 \at@end@of@kframe}
\makeatother

\definecolor{shadecolor}{rgb}{.97, .97, .97}
\definecolor{messagecolor}{rgb}{0, 0, 0}
\definecolor{warningcolor}{rgb}{1, 0, 1}
\definecolor{errorcolor}{rgb}{1, 0, 0}
\newenvironment{knitrout}{}{} % an empty environment to be redefined in TeX

\usepackage{alltt}
\usepackage[english,portuges]{babel}
\usepackage{ppgi,mestre,epsfig}
\usepackage{times}
\usepackage[final]{pdfpages}
\usepackage{hyperref}
\hypersetup{
    bookmarks=true,   
    pdftitle={Monitor legislativo},
    pdfauthor={Vitor Márcio Paiva de Sousa Baptista}, 
    pdfsubject={Modelo de Documento Científico},
    pdfkeywords={Dissertação, Mestrado, PPGI, UFPB, modelo}, 
    colorlinks=true,
    linkcolor=black,
    citecolor=black,
    filecolor=black,
    urlcolor=black
 }


%-------------------------- Para usar acentuacaoo em sistemas ISO8859-1 ------------------------------------
% Se estiver usando o Microsoft Windows ou linux com essa codificacao, descomente essas linhas abaixo
% e comente as linhas referentes ao UTF8
%\usepackage[applemac]{inputenc} % Usar acentuacao em sistemas ISO8859-1, comentar a linha com  \usepackage[utf8x {inputenc}
%-----------------------------------------------------------------------------------------------------

%-------------------------- Para usar acentuacao em sistemas UTF8 ------------------------------------
% Para a maior parte das distribuicoes linux, usar essa opcao
\usepackage{ucs}
\usepackage[utf8x]{inputenc}
\usepackage[T1]{fontenc}
%-----------------------------------------------------------------------------------------------------
\usepackage{float}      
\usepackage{fancyvrb}
\usepackage{fancyheadings}
\usepackage{graphicx}
\usepackage{longtable} %tabelas longas, para tabelas que ultrapassam uma pagina
\usepackage{minted} % Syntax highlighting para código
\usepackage{pdflscape} % landscape figures
\usepackage{glossaries} % Gerar glossário

\makeglossaries
\newacronym{IR}{IR}{Índice de Rice}
\newacronym{PT}{PT}{Partido dos Trabalhadores}
\newacronym{PSOL}{PSOL}{Partido Socialismo e Liberdade}
\newacronym{API}{API}{Application Programming Interface}
\newacronym{XML}{XML}{eXtensible Markup Language}
\newacronym{CEBRAP}{CEBRAP}{Centro Brasileiro de An\'alise e Pesquisa}

%\input{psfig.sty}
% ----------------- Para inserir codigo fonte de linguagens de programacao no documento -------------
\usepackage{listings}
\lstset{numbers=left,
stepnumber=1,
firstnumber=1,
numberstyle=\scriptsize,
extendedchars=true,
breaklines=true,
frame=tb,
basicstyle=\scriptsize,
stringstyle=\ttfamily,
showstringspaces=false
}
\renewcommand{\lstlistingname}{C\'odigo Fonte}
\renewcommand{\lstlistlistingname}{Lista de C\'odigos Fonte}

% ---------------------------------------------------------------------------------------------------

\selectlanguage{portuges}
\sloppy

\setcounter{secnumdepth}{4}
\setcounter{tocdepth}{4}
\usepackage{abnt-alf}

% Centraliza todos os floats
\makeatletter
\g@addto@macro\@floatboxreset\centering
\makeatother
\IfFileExists{upquote.sty}{\usepackage{upquote}}{}
\begin{document}


%%%%%%%%%%%%%%%%%%%%%%%%%%%%%%%%%%%%%%%%%%%%%%%%%%%%%%%%%%%%%%%%%%%%%%%%%%%%%%%%
\Titulo{Monitor legislativo}
\Autor{Vitor Márcio Paiva de Sousa Baptista}
\Data{06 de Março de 2012}
\Area{Ciência da Computação}
\Pesquisa{Computação Distribuída | Sinais, Sistemas Digitais e Gráficos}
\Orientadores{Alexandre Nóbrega Duarte\\ (Orientador)}

\newpage
\cleardoublepage
\PaginadeRosto

\newpage
\cleardoublepage

%%%%%%%%%%%%%%%%%%%%%%%%%%%%%%%%%%%%%%%%%%%%%%%%%%%%%%%%%%%%%%%%%%%%%%%%%%%%%%%%
\begin{resumo} 
Vestibulum varius accumsan odio malesuada gravida. Duis a erat et arcu tincidunt semper sed et quam. Sed mattis semper quam vel imperdiet. Etiam tortor orci, ullamcorper ac aliquam eu, interdum quis justo. Morbi lacinia ligula ac nibh imperdiet semper. Aliquam varius tristique nisl, in blandit tellus ultrices et. Nullam est nisl, pretium sit amet vehicula quis, cursus at enim.
\\
\\
\textbf{Palavras-chave:} Palavras, chave, para, seu, trabalho.

\end{resumo}
%\newpage
%\cleardoublepage

%%%%%%%%%%%%%%%%%%%%%%%%%%%%%%%%%%%%%%%%%%%%%%%%%%%%%%%%%%%%%%%%%%%%%%%%%%%%%%%%
\begin{summary}
In Brazil, there are tools for monitoring the behaviour of legislators in
rollcalls, such as O Estado de São Paulo's Basômetro and Radar Parlamentar.
These tools are used both by journalists and political scientists for analysis.

Although they are great analysis tools, their usefulness for monitoring is
limited because they require a manual follow-up, which makes it a lot of work
when we consider the volume of data. Only in the Chamber of Deputies, 513
legislators participate on average over than 400 rollcalls by legislature. It
is possible to decrease the amount of data analyzing the parties as a whole,
but in contrast we lose the ability to detect individuals' drives or
intra-party groups such as factions.

In order to mitigate this problem, I developed a statistical model that detects
when a legislator changes his or her position, joining or leaving the
governmental coalition, through ideal points estimates using the W-NOMINATE. It
can be used individually or integrated to tools such as Basômetro, providing a
filter for researchers find the deputies who changed their behaviour most
significantly.

The universe of study is composed of legislators from the Chamber of Deputies
from the 50th to the 54th legislatures, starting in the first term of Fernando
Henrique Cardoso in 1995 until the end of the first term of Dilma Rousseff in
2015.

\textbf{Keywords:} Legislative Analysis, Political Science, Data Science,
Predictive Models, Machine Learning.

\end{summary}

%\newpage
%\cleardoublepage

%%%%%%%%%%%%%%%%%%%%%%%%%%%%%%%%%%%%%%%%%%%%%%%%%%%%%%%%%%%%%%%%%%%%%%%%%%%%%%%%
% TMP: Agradecimentos
\begin{agradecimentos}
Donec ultricies elit a quam ornare posuere. Pellentesque eu tortor massa. Aliquam erat volutpat. In vitae justo dolor, ac fringilla nisl. In hac habitasse platea dictumst. Pellentesque placerat eleifend sem, in tempor nisl elementum fermentum. Ut in metus vitae magna volutpat viverra. Suspendisse ac dolor velit, in volutpat magna. Cras blandit urna quis diam feugiat volutpat. Nunc mattis lobortis libero varius posuere. Integer sem augue, aliquet fringilla porta nec, adipiscing sed ante. Aenean feugiat, eros non vehicula pretium, neque purus vehicula diam, eu vulputate leo neque nec velit. Vestibulum at orci quam, et mattis tortor. Donec iaculis orci enim.

\end{agradecimentos}

\clearpage

%%%%%%%%%%%%%%%%%%%%%%%%%%%%%%%%%%%%%%%%%%%%%%%%%%%%%%%%%%%%%%%%%%%%%%%%%%%%%%%%
%% Definicao do cabecalho: secao do lado esquerdo e numero da pagina do lado direito
\pagestyle{fancy}
\addtolength{\headwidth}{\marginparsep}\addtolength{\headwidth}{\marginparwidth}\headwidth = \textwidth
\renewcommand{\chaptermark}[1]{\markboth{#1}{}}
\renewcommand{\sectionmark}[1]{\markright{\thesection\ #1}}\lhead[\fancyplain{}{\bfseries\thepage}]%
	     {\fancyplain{}{\emph{\rightmark}}}\rhead[\fancyplain{}{\bfseries\leftmark}]%
             {\fancyplain{}{\bfseries\thepage}}\cfoot{}

%%%%%%%%%%%%%%%%%%%%%%%%%%%%%%%%%%%%%%%%%%%%%%%%%%%%%%%%%%%%%%%%%%%%%%%%%%%%%%%%

\Sumario
\ListadeSiglas
\listoffigures
\listoftables
\lstlistoflistings %lista de codigos fonte - Para inserir a listagem de
% codigos fonte

\newpage
\cleardoublepage
\Introducao


%%%%%%%%%%%%%%%%%%%%%%%%%%%%%%%%%%%%%%%%%%%%%%%%%%%%%%%%%%%%%%%%%%%%%%%%%%%%%%%%
%
% Hifenizacao - Colocar lista de palavras que nao devem ser separadas e que 
% nao estao no dicionario portuges.
% As palavras do dicionario portuges ja sao separadas corretamente pelo lateX
%
\hyphenation{ gLite OurGrid GridDoctor }

%%%%%%%%%%%%%%%%%%%%%%%%%%%%%%%%%%%%%%%%%%%%%%%%%%%%%%%%%%%%%%%%%%%%%%%%%%%%%%%%
%% A partir daqui coloque seus capitulos. Sugere-se que eles sejam inseridos com o comando \input
%% Da seguinte maneira:
%% 



\chapter{Introdução} \label{intro}

\section{Motivação}\label{sec:motiva}

A quantidade de dados relacionados a atividade dos parlamentares brasileiros é muito grande. Exemplos simples são as votações, presenças em plenário, projetos de Lei, viagens, candidaturas, etc. Analisar manualmente esse volume de dados é uma tarefa bastante difícil e por isso podem haver situações importantes que não sejam percebidas. Visando auxiliar no processo de acompanhamento e fiscalização da atividade dos parlamentares brasileiros este trabalho propõe uma ferramenta de monitoramento automático e contínuo que gera alertas e relatórios sobre o que acontece nas casas legislativas brasileiras, usando a Câmara dos Deputados como estudo de caso.

\section{Objetivos}

O objetivo geral desta dissertação é o desenvolvimento e validação de um modelo
capaz de prever quando um deputado federal irá deixar ou entrar na coalizão do
governo a partir das votações nominais.

\subsection{Objetivos Especificos}

\begin{itemize}
\item Desenvolver um mecanismo capaz de detectar mudanças no padrão de comportamento dos parlamentares;
\item Desenvolver um mecanismo capaz de detectar mudanças semelhantes entre grupos de parlamentares (ex.: João e Pedro sempre divergiram em seus votos e de repente passaram a concordar);
\item Desenvolver um mecanismo capaz de detectar votações importantes/polêmicas (falta definir o que são importante e polêmica)
\end{itemize}

\section{Metodologia}

Não sei qual o nome da metodologia que estou usando: por enquanto é tudo muito ad-hoc. Estudo de caso é um componente, mas não sei se há outra.

Alexandre -  Não é preciso dar uma nome a metodologia. O que você precisa descrever aqui são os passos seguidos até chegar ao resultado final.

Tipo
1 - Levantamento Bibliográfico sobre X e Y
2 - Desenvolviment do mecanismo para extração e filtragem dos dados
3 - ...

\section{Publicações Relacionadas}

Mencionar o artigo que você publicou no BRASNAM

\section{Estrutura da Dissertação}

Descrever a estrutura dos demais capítulos. Melhor fazer isso só no final.


\chapter{Fundamentação Teórica}\label{cap:fundamentacao}

\section{Índice de Rice}

O \gls{IR} é uma métrica para o cálculo da coesão\footnote{Neste trabalho
entendemos por coesão o grau de unidade do partido nas decisões legislativas
tomadas em votações nominais, independente se isso ocorre porque seus membros
têm uma ideologia semelhante, ou porque seus líderes conseguem discipliná-los.}
entre grupos em uma votação. Desde sua introdução em \cite{Rice1924}, ele vem
sendo amplamente utilizado em estudos legislativos por ser intuitivo, flexível
e facilmente interpretável \cite{Neiva2011}.

Para uma votação $i$, o seu cálculo se dá pela diferença, em números absolutos,
entre a quantidade de votos \verb|Sim| e de votos \verb|Não|, dividida pelo
\verb|Total| de votos, como pode ser visto na fórmula \ref{eq:rice}.

\begin{equation}\label{eq:rice}
  IR_i = \frac{|Sim - N\tilde{a}o|}{Total}
\end{equation}

O índice varia de zero, quando metade votou \verb|Sim| e metade votou
\verb|Não|, até um, quando todos votaram da mesma forma. Para analisar mais de
uma votação, calculamos a média aritmética dos \gls{IR} das votações no
período, como mostra a fórmula \ref{eq:rice-media}.

\begin{equation}\label{eq:rice-media}
  \sum_{i=1}^{n} \frac{IR_i}{n}
\end{equation}

% TODO: Escrever sobre o problema de só considerar votos Sim ou Não e como
% resolvemos nesse trabalho (talvez só explicitar o problema, pra manter
% fundamentação teórica só como fundamentação teórica).
% Figueiredo/Limongi 1995 só consideram votos Sim e Não, então talvez posso
% dizer que estou seguindo o exemplo deles :)
% ---
% O primeiro problema enfrentado em usar o índice de Rice no contexto brasileiro
% é que ele considera só os indivíduos que votaram, ignorando os ausentes, os que
% se abstiveram e os que obstruíram.

Um problema descrito por Desposato (2005) é que índices de coesão em geral
(o de Rice em particular) são fortemente enviesados. Partidos pequenos têm uma
coesão esperada maior do que partidos grandes, o que invalidaria comparações
entre grupos de tamanhos diferentes, pois diferenças na coesão podem ser
resultado de uma característica do índice usado, e não no comportamento dos
parlamentares. Este problema é especialmente grave no Brasil por existir uma
grande variação no tamanho dos partidos.\nocite{Desposato2005}

Ele propôs uma solução simples: diminuir o tamanho dos partidos.
Especificamente, ele sugere trocar o índice de coesão $C_{ij}$ de cada
partido por $E(C_{ijr})$, que é a coesão esperada de uma amostra de $r$ votos
tomados sem reposição do partido $i$ na votação $j$. O tamanho da amostra $r$
pode variar entre 2 e o tamanho do menor partido, mas ele sugere escolhermos $r
= 2$, que funciona em todos os casos.\nocite{Desposato2005}

Com essa escolha, a interpretação de $E(C_{ij2})$ é simples: é a probabilidade
de que dois membros do partido $i$ escolhidos aleatoriamente votaram juntos na
votação $j$. O cálculo também fica simplificado, sendo feito pela fórmula
\ref{eq:rice-adj}, onde $Y$ é o número de votos ``sim'' e $R$ é o tamanho do
partido.\nocite{Desposato2005}

\begin{equation}\label{eq:rice-adj}
  E(C_{ij2}|Y, R) = \frac{Y(Y - 1) + (R - Y)(R - Y - 1)}{R(R - 1)}
\end{equation}

Caso o índice $C$ já seja conhecido, a fórmula pode ser reduzida a
\ref{eq:adj-known-metric}.\nocite{Desposato2005}

\begin{equation}\label{eq:adj-known-metric}
  E(C_2|Y, R) = \frac{RC^2 + R + 2}{2(R - 1)}
\end{equation}

\subsection{Calculando coesão entre partidos}

Encontramos um problema ao calcular a coesão entre partidos. Caso simplesmente
calculássemos o \gls{IR} entre o conjunto de parlamentares dos partidos
que queremos analisar, estaríamos mostrando um resultado errado caso todos os
partidos não tivessem o mesmo número de parlamentares votando em cada votação.

Por exemplo, considere que queremos calcular a coesão entre o \gls{PT} e o \gls{PSOL} na
Câmara dos Deputados na 54a legislatura. Nesse período, o \gls{PT} contava com XXX
deputados federais, enquanto o \gls{PSOL} contava com 3. Se simplesmente calcularmos
a coesão considerando o conjunto de deputados do \gls{PT} com o do \gls{PSOL}, o índice
seria muito influenciado pela diferença de parlamentares.

Para resolver esse problema, não consideramos o voto de cada parlamentar de
cada partido individualmente, mas sim o voto de um ``parlamentar médio'' do
partido. O voto do parlamentar médio é definido como o voto da maioria do
partido. Dessa forma, conseguimos comparar partidos independente de seus
tamanhos.

Poderíamos ter definido o ``parlamentar médio'' como a indicação do líder do
partido, mas neste caso teríamos que desconsiderar as votações em que o líder
libera sua bancada e não conseguiríamos calcular para partidos cujo líder não
pode indicar voto (de acordo como o Regimento Interno). Dado essas
desvantagens, e o fato que há poucas diferenças entre a indicação do líder e o
voto da maioria \cite{Figueiredo1995}, preferimos usar o voto da maioria.

\section{Modelos espaciais de votação}



A ideia básica de modelos espaciais de votação é que o conjunto de alternativas
políticas de uma votação pode ser tratado como uma dimensão em um espaço
euclidiano, e cada parlamentar tem preferências de valores nessa dimensão. O
voto seria então definido pela escolha da alternativa mais próxima da sua
preferência.

Há diversos modelos para estimar esses valores, como o NOMINATE, W-NOMINATE,
DW-NOMINATE, que são paramétricos; o Optimal Classification, que é
não-paramétrico, e; modelos baseados em estatística Bayesiana, como o IDEAL.
Nosso foco neste trabalho é no W-NOMINATE.
\cite{Poole2000,Poole2005,Poole2014,Jackman2000,Clinton2004}

Considere um exemplo onde 5 parlamentares votam em
dois projetos de Lei: um propondo a redução da maioridade penal de
18 para 16 anos e outra
propondo o aumento do salário-mínimo de R\$ 1.000 para
R\$ 1.200. Suponha que cada um tem uma única preferência
(\emph{single-peakedness}), conhecido como seu ponto ideal, e vota sinceramente
de acordo com ela. Considere também que as preferências são simétricas. Isto é,
dado duas escolhas a uma mesma distância do ponto ideal de um parlamentar, ele
será indiferente a qualquer uma delas.

Graficamente, temos a figura \ref{fig:modelo-espacial-votacao}, onde os pontos
representam as preferências dos legisladores sobre cada votação, representados
como as dimensões nesse espaço euclidiano. As linhas são chamadas linhas de
corte. Elas passam pelo ponto médio entre as duas alternativas em votação: a de
votar sim e a de votar não. Em outras palavras, se as escolhas são entre uma
maioridade penal de 16 ou
18, a linha de corte passa em $\frac{16 + 18}{2}$,
ou seja, $17$. Ela divide os legisladores que irão votar não, que estão a
esquerda da linha, dos que irão votar sim, que estão a direita da mesma. Caso
esteja em cima da linha de corte, ele é indiferente as alternativas. Caso algum
legislador não siga essa previsão, ele é considerado um ``erro'' do modelo.

\begin{knitrout}
\definecolor{shadecolor}{rgb}{0.969, 0.969, 0.969}\color{fgcolor}\begin{figure}
\includegraphics[width=\maxwidth]{figure/modelo-espacial-votacao-1} \caption[Preferências de 5 deputados em 2 votações com suas respectivas linhas de corte]{Preferências de 5 deputados em 2 votações com suas respectivas linhas de corte}\label{fig:modelo-espacial-votacao}
\end{figure}


\end{knitrout}

Formalmente, sendo $O_{jy}$ e $O_{jn}$ os resultados correspondendo
respectivamente a um voto ``sim'' e um voto ``não'' na votação $j$ ($j = 1,
..., q$), definimos $Z_j$ como: 

\begin{equation}\label{eq:cutpoint}
  Z_j = \frac{O_{jy} + O_{jn}}{2}
\end{equation}

O $Z_j$ pode definir um ponto, linha, plano ou hiperplano, dependendo do número
de dimensões que estamos tratando. Ele é conhecido como ponto (ou linha, etc.)
de corte.

% Introdução

%% O que são de modelos de votação?

%% Por que os modelos espaciais de votação foram criados?

%% Por que não usar os votos diretamente?

% Desenvolvimento

% Conclusão

\subsection{Comparando pontos ideais ao longo do tempo}
\label{cap:fundamentacao:comparando-pontos-ideais-no-tempo}

O principal problema em comparar mudanças nos pontos ideais ao longo do tempo é
distinguir alterações causadas por mudanças na agenda legislativa das causadas
por mudanças no posicionamento dos parlamentares \cite{Bailey2007}. Em outras
palavras, se o ponto ideal de um deputado federal passa de 0.3 para -0.2 de um ano
para o outro, como descobrir se isso representa uma mudança real de ideologia ou
é somente reflexo da diferença na agenda legislativa dos dois períodos?

Segundo \citeonline{Shor2010}, todos os esforços para resolver esse problema
usam ``pontes'', que podem ser parlamentares cujo posicionamento assume-se ter
se mantido estável durante o período de interesse, ou; projetos de Lei que
foram votados em mais de um momento (nas duas casas legislativas em um sistema
bicameral, por exemplo). O primeiro é mais usado para comparar as mudanças nos
posicionamentos dos parlamentares, enquanto o segundo permite unir pessoas que
não votaram juntas em um mesmo mapa espacial\footnote{Por exemplo,
\citeonline{Shor2010} colocam todos os legisladores de 11 estados americanos e
do congresso federal em um período que varia entre 7 e 15 anos, dependendo do
estado, em um mesmo mapa espacial}. \citeonline{Poole2005} propõe duas formas
para estimar pontos ideais usando pontes.

Na primeira, batizada de \emph{pooled scaling} por \citeonline{Shor2010},
dividimos os votos dos parlamentares que queremos mensurar em dois
parlamentares ``virtuais'', um com os votos antes e outro com os votos depois
da data de interesse. Unimos esses parlamentares virtuais com os
parlamentares-ponte, que possuem um registro único, em uma tabela individual e
executamos o algoritmo de estimação dos pontos ideais. Ao final, teremos dois
pontos para cada parlamentar de interesse e um ponto para os pontes. Na
segunda, que \citeonline{Shor2010} chamam de \emph{linear mapping}, estimamos
os pontos ideias separadamente em cada período e os conectamos usando regressão
entre os conjuntos de pontos dos parlamentares-ponte. Ambas formas devem gerar
resultados similares, mas a segunda é computacionalmente mais simples, o que
pode ser essencial, dependendo da quantidade de parlamentares e votações votos
em estudo.

\citeonline{Poole2005} ainda descreve uma terceira forma, similar a
\emph{pooled scaling} descrita acima, que usa para testar se os senadores
norte-americanos mudam de comportamento nos últimos dois anos de seus mandatos,
antes de concorrer à reeleição. Neste caso, ele quer calcular a mudança de
comportamento de todos os legisladores em dois momentos numa mesma legislatura.
Se usássemos o \emph{pooled scaling} diretamente, precisaríamos escolher alguns
parlamentares como pontes que, por definição, não teriam mudado de
comportamento. Ao invés disso, ele segue o seguinte processo:

\begin{enumerate}
  \item Para cada parlamentar, faça:
    \begin{enumerate}
      \item Transforme-o em dois parlamentares ``virtuais'', um com o conjunto
de votos antes, e outro com os depois da data de interesse. Os outros
parlamentares não são modificados;
      \item Calcule os pontos ideias;
      \item Guarde os resultados dos dois parlamentares ``virtuais''. A
diferença entre suas posições representa a mudança de comportamento do
parlamentar.
    \end{enumerate}
  \item Calcule medidas de incerteza para as estimativas.
\end{enumerate}

Ao final, ele tem dois pontos para cada senador: um representando sua posição
nos primeiros 4, e o outro nos últimos 2 anos da legislatura. Note que, como as
matrizes de votações usadas para gerar os mapas espaciais são diferentes, eles
não são estritamente comparáveis. Apesar disso, \citeonline{Poole2005}
argumenta que como elas possuem o mesmo conjunto de votos, com a única
diferença de que um dos parlamentares foi dividido em dois, ele considera ser
seguro compará-las.

\section{Processo legislativo}

Como são criadas as Leis no Brasil? Essa seção pode ser gigantesca, então acho melhor focar no essencial e pontuar as características mais importantes para este trabalho, por exemplo o que são destaques, tipos de votação, tipos de votos, etc.

% ALEXANDRE: Acho que senado e câmara podem ser subseções desta seção.
% Essas subseções não precisam ser muito longas. Devem descrevas basicamente as competências e atribuições de cada casa.
%
% Acho que a justificativa da escolha da Câmara pode aparecer mais tarde, na seção do estudo de caso.  
% Assim a seção de fundamentação teórica fica sendo realmente apenas de fundamentação teórica, sem efeitos colaterais.

\subsection{Senado Federal}

Também chamada de Casa Alta, ela é composta por 81 senadores, 3 por cada Estado e Distrito Federal. Os senadores têm mandato de 8 anos e são eleitos pelo sistema majoritário\footnote{No sistema majoritário, o candidato mais votado é eleito. Nos anos em que são eleitos 2 senadores, os dois candidatos mais votados serão eleitos. Ele pressupõe um (ou dois) único candidatos por partido. É o mesmo sistema usado nas eleições para os chefes do Poder Executivo (presidente da República, governadores e prefeitos). Com exceção da eleição de senadores e prefeitos em cidades com menos de 200 mil habitantes, onde há um único turno, as eleições ocorrem em dois turnos. \cite{Netto2013}}. A renovação da casa é parcial, modificando $\frac{1}{3}$ e $\frac{2}{3}$ dos senadores alternadamente de 4 em 4 anos.

\subsection{Câmara dos Deputados}

Também chamada de Casa Baixa ou Casa do Povo, ela é composta por 513 deputados federais, cujo papel é representar o povo. Cada Estado ou Distrito Federal elege entre 8 e 70 deputados, proporcionalmente a sua população. Os deputados têm mandato de 4 anos, eleitos pelo sistema proporcional\footnote{No sistema proporcional os partidos registram vários candidatos para o mesmo cargo. Os votos recebidos por cada candidato são direcionados ao partido, que precisa atingir uma quantidade mínima de votos (chamado quociente eleitoral) para eleger ao menos um de seus candidatos. Nesse sistema, um candidato com menos votos pode se eleger, enquanto outro com mais votos não se elegeu, dependendo do partido de cada um deles. \cite{Netto2013,Bramatti2014}}. Diferente do Senado Federal, a renovação é total de 4 em 4 anos.

\section{Coesão parlamentar}

Definir o que é coesão e disciplina, e explicar algumas métricas (ou só o índice de Rice)


\section{Ciência de dados}

Texto bem geral sobre data science. Pode ser interessante falar sobre algum algoritmo específico que eu vá usar, ou talvez seja melhor explicá-los no miolo da dissertação.

\section{Considerações Finais}





\chapter{Trabalhos Relacionados}\label{cap:relacionados}


\section{Análise de mudança de comportamento de parlamentares}

O problema básico em comparar mudanças de comportamento ao longo do tempo é
distinguir alterações por causa de uma mudança de agenda das causadas por uma
real mudança das preferências dos parlamentares \cite{Bailey2007}. Em outras
palavras, se em um momento dois parlamentares \emph{A} e \emph{B} votaram 90\%
das vezes da mesma forma, e em outro momento eles votaram 50\% das vezes, como
definir se essa mudança se deu porque eles mudaram seus posicionamentos, ou
simplesmente porque eles concordavam nas votações do primeiro momento, mas não
nas do segundo?

Segundo \citeonline{Shor2010}, todos os esforços para resolver esse problema
usam ``pontes'', que são parlamentares que estiveram presentes em ambos
momentos e cujo posicionamento se assume ter se mantido estável. Podemos citar
como exemplos parlamentares que foram eleitos para mais de uma legislatura, ou
que mudaram de Casa (deputados federais que se tornam senadores). Projetos de
Lei também podem ser usados como ponte, caso eles tenham sido votados nas
instituições ou períodos de interesse.

No artigo, \citeonline{Shor2010} usa três tipos de pontes para colocar
parlamentares que serviram no nível estadual em ambas as casas\footnote{O
sistema legislativo norte-americano, ao contrário do brasileiro, é bicameral
tanto a nível federal quanto estadual (exceto o estado de Nebrasca, que só
possui um Senado estadual)}, federal em ambas as casas, e no tempo.
Parlamentares que serviram por múltiplas legislaturas tanto a nível estadual
quanto federal servem de ponte entre as legislaturas; legisladores que passam
da casa baixa para a casa alta (ou vice-versa) em nível estadual conectam as
respectivas casas, e; parlamentares que passam do nível estadual para atuar no
nível federal conectam o estado com o Congresso.

Os trabalhos se dividem em dois grandes grupos: os que usam medidas de coesão e
os que usam modelos espaciais de votação.

% Dividem-se em dois grupos: os que usam métodos espaciais e os que usam
% métodos "normais" (qual o nome?).
% Radar Parlamentar
% Basômetro
% TODO: O livro do Poole "Congress: A Political-Economic History of Roll Call
% Voting" parece importante demais pra estar de fora.

\citeonline{Desposato2005b} analisou os efeitos da mudança de partidos no
comportamento dos senadores e deputados federais brasileiros durante a
49\textordfeminine{} e 50\textordfeminine{} legislaturas. Ele usou o W-NOMINATE
para estimar os pontos ideais de cara par parlamentar-partido. Ou seja, se o
parlamentar \emph{A} mudar do PT para o DEM, ele terá dois pontos ideais: um
para cada período.

\citeonline{Leoni2002} analisou o comportamento dos partidos políticos na
Câmara dos Deputados entre 1991 e 1998 usando a correlação de Pearson. Ele
estava buscando entender se os deputados federais mantinham suas posições
relativas ao longo do tempo, e qual influência o presidente da República tinha
em alterar essas posições. Nesse período, ele não encontrou uma influência
estatisticamente relevante dos presidentes na composição da Câmara, mas há o
porém que todos os presidentes estudados eram de direita. Isso pode ter
influenciado seu resultado.

\section{Coesão parlamentar}

A coesão entre parlamentares é estudada há décadas pela Ciência Política. Em 1924, Stewart Rice propôs a primeira métrica para medir coesão: o Índice de Rice \cite{Rice1924}. Depois dele, vários outros pesquisadores proporam novas métricas, como XXX, YYY, ZZZ. Apesar disso, o IR continua sendo muito usado.

No Brasil, podemos citar \citeonline{Figueiredo1995} que analisa o padrão de
votação dos parlamentares da Câmara dos Deputados entre 1989 e 1994;
\citeonline{Neto1997}, que analisa a mesma casa mas entre 1946 e 1964, e;
\citeonline{Neiva2011}, que analisa votações do Senado entre 1989 a 2009. Todos utilizam o Índice de Rice.

\subsection{Discussão}

Explicar que apesar de existirem várias métricas de coesão, a mais usada ainda é a de Rice, mas que usarei ela com algumas modificações como normalizá-lo para manter o mesmo índice independente do número de parlamentares considerados no cálculo. Existe um artigo que explica isso (só preciso encontrá-lo)

Explicar também que, apesar das limitações de analisar a coesão através das votações nominais (muito limitado, ignora o trabalho dos bastidores), ela é uma forma simples e direta de analisar um grande volume de dados, e ainda é muito usada.

\section{Ciência de dados}

Listarei trabalhos sobre ciência de dados, especialmente métodos de detecção de anomalias. Acho que não cabe aqui, mas a parte de sistemas de monitoramento de servidores e a sacada de usá-los nessa outra área é interessante. Também a criação de um pipeline, possivelmente usando o Luigi (\url{https://github.com/spotify/luigi}).


(Alexandre) Essa parte sobre a parte de sistemas de monitoramento entra em outro capítulo e concordo que deve ser mencionada sim na dissertação. Acho que isso pode aparecer no próximo capítulo.


\subsection{Discussão}

\section{Considerações Finais}

Esse trabalho não trás muitas novidades especificamente nas áreas relacionadas, mas a contribuição está na junção dessas ferramentas já existentes. Nas minhas pesquisas, só encontrei o a startup \url{www.fiscalnote.com} que faz um trabalho parecido, mas tenho certeza que existem outras empresas, ou sistemas internos. O que não encontrei é um monitorador legislativo que não foque em uma área ou projeto de Lei específico, mas busque padrões gerais. O que encontrei mais parecido é o monitoramento de servidores, que segue o padrão de analisar o maior número possível de dados buscando anomalias e padrões.


(Alexandre) Então esse deve ser o principal diferencial do seu trabalho em relação aos demais. Essa diferença deve ser reforçada ao apresentar os trabalhos relacionados.

Um negócio que fica muito legal nesta seção é criar uma tabela com um conjunto de características presentes e desejadas e marcar quais trabalhos relacionados atendem a cada uma dessas características.

Assim fica mais fácil constatar onde está o grande diferencial do seu trabalho, que seria justamente ser uma solução com aplicação mais ampla e não focada em áreas específicas.




\chapter{Miolo da sua dissertação}\label{cap:miolo}

Nessa parte pretendo falar sobre a ferramenta desenvolvida, mostrar sua arquitetura e como cada parte funciona e pode ser modificada.

\section{Coleta dos dados}

A Câmara dos Deputados disponibiliza em seu site na
web\footnote{\url{http://www2.camara.leg.br/transparencia/dados-abertos}} uma
\gls{API} para acesso a diversos dados. Para extrair os resultados das
votações, usamos os métodos
\emph{ListarProposicoesVotadasEmPlenario}\footnote{\url{http://www2.camara.leg.br/transparencia/dados-abertos/dados-abertos-legislativo/webservices/proposicoes-1/ProposicoesVotadasEmPlenario}}
que a partir de um ano nos retorna a lista de proposições votadas; e o
\emph{ObterVotacaoProposicao}\footnote{\url{http://www2.camara.leg.br/transparencia/dados-abertos/dados-abertos-legislativo/webservices/proposicoes-1/obtervotacaoproposicao}}
que a partir do tipo, número e ano de uma proposição retorna os resultados das
votações ocorridas.  

Os dados são retornados no formato \gls{XML}. De posse deles, desenvolvemos um
programa em
Python\footnote{\url{https://github.com/vitorbaptista/codigo-mestrado}} para
transformá-los e importá-los em um banco de dados
SQLite\footnote{\url{https://www.sqlite.org/}}. A partir desse banco, geramos
dois arquivos por legislatura: um com os dados dos parlamentares e seus votos
(tabela \ref{table:votes}) e outro com os detalhes das votações (tabela
\ref{table:votes-metadata}).

Os nomes dos partidos foram normalizados para evitar que consideremos fusões ou
mudanças de nomes de partidos como um novo partido. Seguimos o padrão adotado
pelo \gls{CEBRAP}, onde mudanças de nome são identificados pela primeira e
última sigla (ex.: PJ se torna PRN que se torna PJC, então temos PJ>PTC) e
fusões são identificados pelo maior partido na data da fusão e o novo nome
(ex.: PL e PRONA se tornam PL>PR) \cite{Freitas2008}. A lista completa está no
apêndice \ref{apendice:lista-partidos}.

\begin{table}
\centering
\begin{knitrout}
\definecolor{shadecolor}{rgb}{0.969, 0.969, 0.969}\color{fgcolor}
\begin{tabular}{r|l|l|l|r|r|r|r|r}
\hline
id & name & party & state & 76 & 273 & 271 & 272 & 485\\
\hline
3151 & Jairo Ataíde & PFL>DEM & MG & NA & 0 & NA & 0 & NA\\
\hline
4929 & Joseph Bandeira & PT & BA & NA & NA & NA & NA & NA\\
\hline
4930 & Silvio Costa & PSC & PE & 1 & 0 & 0 & 0 & 0\\
\hline
4931 & Izalci & PL+PRONA>PR & DF & 1 & 1 & 0 & 0 & 0\\
\hline
62881 & Danilo Forte & PMDB & CE & 1 & 1 & 0 & 0 & NA\\
\hline
67138 & Iracema Portella & PDS>PP & PI & 1 & 1 & 0 & 0 & 0\\
\hline
\end{tabular}


\end{knitrout}
\caption{Dados com detalhes dos parlamentares e seus respectivos votos. Não
mostramos todos os votos por limitações de espaço.}
\label{table:votes}
\end{table}

\begin{landscape}
\begin{table}
\centering
\begin{knitrout}
\definecolor{shadecolor}{rgb}{0.969, 0.969, 0.969}\color{fgcolor}
\begin{tabular}{r|r|r|l|l|l}
\hline
id & id\_sessao & proposicao\_id & data & resumo & obj\_votacao\\
\hline
76 & 4212 & 491974 & 2011-02-15 21:09:00 & Aprovado o Requerime... & URGÊNCIA PARA O PL 3...\\
\hline
273 & 4215 & 491872 & 2011-02-16 00:17:00 & Mantida a expressão.... & DVS - Bloco PvPps - ...\\
\hline
271 & 4214 & 491872 & 2011-02-16 22:38:00 & Rejeitada a Emenda d... & DVS - PSDB - EMENDA ...\\
\hline
272 & 4214 & 491872 & 2011-02-16 23:27:00 & Rejeitada a Emenda d... & DVS - DEM - EMENDA N...\\
\hline
485 & 4219 & 485274 & 2011-02-22 17:36:00 & Rejeitado o Requerim... & REQUERIMENTO DE RETI...\\
\hline
486 & 4219 & 485274 & 2011-02-22 19:20:00 & Rejeitado o Requerim... & REQUERIMENTO DE ADIA...\\
\hline
487 & 4220 & 485274 & 2011-02-22 20:45:00 & Rejeitado o Requerim... & REQUERIMENTO DE RETI...\\
\hline
955 & 4221 & 485275 & 2011-02-23 17:01:00 & Rejeitado o Requerim... & REQUERIMENTO DE RETI...\\
\hline
1042 & 4251 & 485758 & 2011-04-05 17:02:00 & Rejeitado o Requerim... & REQUERIMENTO DE RETI...\\
\hline
1043 & 4251 & 485758 & 2011-04-05 19:11:00 & Rejeitado o Requerim... & REQUERIMENTO DE CONC...\\
\hline
\end{tabular}


\end{knitrout}
\caption{Dados com detalhes sobre cada votação. Os campos ``resumo'' e
``obj\_votacao'' foram limitados a 20 caracteres por
questões de formatação da página.}
\label{table:votes-metadata}
\end{table}
\end{landscape}

% \begin{listing}[ht]
% \begin{minted}{xml}
% <proposicoes>
%   <proposicao>
%     <codProposicao>19319</codProposicao>
%     <nomeProposicao>PL 3232/1992</nomeProposicao>
%     <dataVotacao>10/06/2014</dataVotacao>
%   </proposicao>
%   <proposicao>
%     <codProposicao>43617</codProposicao>
%     <nomeProposicao>PLP 275/2001</nomeProposicao>
%     <dataVotacao>22/04/2014</dataVotacao>
%   </proposicao>
%   <!-- Continua... -->
% </proposicoes>
% \end{minted}
% \caption{Resultado da API \emph{ListarProposicoesVotadasEmPlenario} usando o ano 2014 como parâmetro}
% \end{listing}

Restringimos a analisar da 50\textordfeminine{} até a 54\textordfeminine{}
legislatura, que compreende um período de 20 anos, entre o início do primeiro
governo de Fernando Henrique Cardoso em 1995 até o final do primeiro governo de
Dilma Rousseff em 2015. Esse recorte foi feito pois é quando, de acordo com
\cite{Freitas2008}, termina a fase de acomodação dos parlamentares as regras
definidas pela Constituição de 1988.

\begin{table}
\centering
\begin{knitrout}
\definecolor{shadecolor}{rgb}{0.969, 0.969, 0.969}\color{fgcolor}
\begin{tabular}{r|r|r|r}
\hline
Legislatura & Deputados Federais & Votações & Votos\\
\hline
50 & 631 & 468 & 178603\\
\hline
51 & 624 & 419 & 155737\\
\hline
52 & 614 & 451 & 134461\\
\hline
53 & 606 & 619 & 192879\\
\hline
54 & 644 & 432 & 131552\\
\hline
\end{tabular}


\end{knitrout}
\caption{Número de deputados federais e votações por legislatura}
\label{table:estatisticas-legislaturas}
\end{table}

Na tabela \ref{table:estatisticas-legislaturas}, mostramos o número de
deputados federais e votações por legislatura. Consideramos deputados federais
todos parlamentares que participaram de alguma votação na Câmara dos Deputados
no período, por isso o número é maior do que o número de cadeiras. Por
legislatura, temos em média 623,80 deputados e 477,80
votações com 332,04 votos cada.

\section{Componentes}

Mostrar em linhas gerais como é organizada a ferramenta, indo de um nível de abstração mais alto (como o usuário vê a ferramenta) até uma arquitetura técnica.


% ALEXANDRE: Apesar de eu não ser muito fã acho importante apresentar algum tipo de diagrama de componentes nesta seção, para permitir uma visualização da arquitetura da solução.


\subsection{Extração dos dados}

Falar sobre as fontes de dados (câmara dos deputados, banco de dados legislativos do Cebrap, etc.), a forma de extração, passos de limpeza, banco de dados, frequência de atualização, se é pull ou push, e como disponibilizamos esses dados para as outras etapas. Ao término dessa seção, o leitor deverá entender nosso processo de data wrangling e saber como ele próprio pode, se quiser, baixar os dados ao final dessa etapa e fazer suas próprias análises. Talvez tenha problema com isso, pois o pessoal da CEBRAP pode não permitir a redistribuição de seus dados.


% ALEXANDRE: É interessante destacar que disponibilizar estes dados em um formato mais "amigável" também é uma contribuição do seu trabalho.

\subsection{Análise}

Aqui falarei sobre os algoritmos usados e as métricas escolhidas (coesão parlamentar, número de destaques, etc.). Também é bom frisar que a arquitetura permite adicionar novos algoritmos, talvez até de uma forma reativa. Por exemplo, considerando que a acusação que o PP trancou a pauta para pressionar o Lula a nomear uma pessoa na Petrobrás, poderíamos criar um algoritmo que detectasse trancamentos de pauta por um partido qualquer, para no futuro descobrirmos movimentações parecidas.

\subsection{Notificações/Relatórios}

Falar sobre como os usuários consomem esses dados. Minha ideia atual é gerar um relatório semanal/mensal/quando-aparecer-algo-interessante e enviá-lo por email pra quem se cadastrar. Nesse caso, devo mostrar exemplos desse relatório.


(Alexandre) Seria legal considerar também a criação de uma conta que twitaria periódicamente os relatórios.


\section{Considerações Finais}

Desenvolvi uma ferramenta que extrai dados de diversas fontes, os consolida, analisa e gera relatórios para os usuários. Cada parte pode ser modificada e expandida, e os dados (e algoritmos) podem ser usados por outras pessoas. Daqui falta validar que os algoritmos na parte de análise geram algo interessante.
\chapter{Estudos de Caso}\label{cap:avalia}

Aqui irei escolher um (ou mais) relatórios gerados pela ferramenta e analisar seu conteúdo. Por exemplo, se me falaram que um projeto de Lei é interessante (anômalo, etc.), confirmar se ele realmente é.

Vejo duas formas de fazer isso. Uma é considerando só o relatório, que foi gerado a partir de algoritmos que analisam diversos dados diferentes. Por exemplo, um pode ver modificações na coesão, outro o número de destaques em um PL, outro a nomeação de ministros. Ou seja, estou testando a ferramenta como um todo, mas não cada algoritmo/métrica. Pode ser que analisar nomeação de ministros não dê resultados interessantes, mas não conseguirei ver isso a partir do relatório.

Outra forma seria analisar o resultado de cada algoritmo. Assim, o estudo de caso seria executar um algoritmo em um período qualquer e ver que resultados ele me dá. Estaria fazendo um estudo de caso para cada algoritmo. Assim, justifico melhor a escolha dos algoritmos, mas em contrapartida estou fazendo uma análise de um ponto de vista que os usuários do sistema não verão. Não é uma visão "holística" de "o sistema me dá bons resultados", mas sim "esse algoritmo específico me dá bons resultados".

Não sei se consegui explicar direito.

Talvez faça mais sentido estudos de caso se focarem no sistema como um todo, mas eu faria uma validação de cada algoritmo em algum outro lugar, analisando se ele dá resultados interessantes. Não sei onde isso entraria.

(Alexandre) Essa parte da análise dos algoritmos é muito interessante sim mas não seria um estudo de caso. Acho que ela entraria melhor no capítulo anterior como uma análise experimental sobre os algoritmos a serem utilizados na ferramenta.

Os estudos de caso seriam mais amplos e focados na ferramenta como um todo como você mesmo descreveu.



\section{Estudo de Caso}

\subsection{Ferramentas e Tecnologia}
\subsection{Requisitos}
\subsection{Desenvolvimento}
\subsection{Avaliação}

\section{Experimento}

\subsection{Plano do Experimento}
\subsection{Execução do Experimento}
\subsection{Análise do Experimento}

\section{Considerações Finais}



\chapter{Conclusão}\label{cap:conclusao}



%%%%%%%%%%%%%%%%%%%%%%%%%%%%%%%%%%%%%%%%%%%%%%%%%%%%%%%%%%%%%%%%%%%%%%%%%%%%%%%%
%% BIbliografia
%% Coloque suas referencias no arquivo ref.bib

\bibliographystyle{abnt-alf} % estilo de bibliografia   plain,unsrt,alpha,abbrv.
\bibliography{ref} % arquivos com as entradas bib.

%Faz aparecer a referencia bibliografica no indice
\addcontentsline{toc}{section}{\numberline{}Referências Bibliográficas}

%%%%%%%%%%%%%%%%%%%%%%%%%%%%%%%%%%%%%%%%%%%%%%%%%%%%%%%%%%%%%%%%%%%%%%%%%%%%%%%%
%% Apendice
% Caso seja necessario algum apendice

\appendix
% TODO: Modificar o texto para não ficar igual a dissertação da Andréa e checar
% se faltam alguns partidos.

\chapter{Lista dos partidos}
\label{apendice:lista-partidos}

\begin{table}
\centering
\begin{tabular}{l l l l}
  Partido & Sigla agregada & Sigla atual & Nome atual \\
  \hline
  PFL>DEM & PFL>DEM & DEM & Democratas \\
  PCdoB & PCdoB & PCdoB & Partido Comunista do Brasil \\
  PDT & PDT & PDT & Partido Democrático Trabalhista \\
  PHS & PHS & PHS & Partido Humanista da Solidariedade \\
  PMDB & PMDB & PMDB & Partido do Movimento Democrático Brasileiro \\
  PMN & PMN & PMN & Partido da Mobilização Nacional \\
  PDS>PP\footnote{O PDS fundiu-se com o PDC em 1993, passando a chamar-se
Partido Progressista Reformador (PPR). Nova fusão com o PP em 21/09/1995 e nova
nomeclatura: Partido Progressista Brasileiro (PPB). Passa a denominar-se
Partido Progressista (PP) em 2003.} & PDS>PPR>PPB>PP & PP & Partido Progressista \\
  PCB>PPS\footnote{O Partido Comunista Brasileiro (PCB) mudou de nome para
Partido Popular Socialista (PPS) em 1991.} & PCB>PPS & PPS & Partido Popular Socialista \\
  PL>PR\footnote{Fusão do Partido Liberal (PL) com o PRONA em 2006, gestando o
Partido da República (PR).} & PL>PR & PR & Partido da República \\
  PMR>PRB\footnote{O Partido Municipalista Renovador (PMR) mudou a nomeclatura
para Partido Republicano Brasileiro (PRB) em 2006.} & PMR>PRB & PRB & Partido Republicano Brasileiro \\
  PRONA & PRONA & PRONA & Partido da Reedificação da Ordem Nacional \\
  PRTB & PRTB & PRTB & Partido Renovador Trabalhista Brasileiro \\
  PSB & PSB & PSB & Partido Socialista Brasileiro \\
  PSC & PSC & PSC & Partido Social Cristão \\
  PSD & PSD & PSD & Partido Social Democrático \\
  PSDB & PSDB & PSDB & Partido da Social Democracia Brasileira \\
  PSDC & PSDC & PSDC & Partido Social Democrata Cristão \\
  PSL & PSL & PSL & Partido Social Trabalhista \\
  PSOL & PSOL & PSOL & Partido Socialismo e Liberdade \\
  PST & PST & PST & Partido Social Trabalhista \\
  PSTU & PSTU & PSTU & Partido Socialista dos Trabalhadores Unificado \\
  PT & PT & PT & Partido dos Trabalhadores \\
  PTB & PTB & PTB & Partido Trabalhista Brasileiro \\
  PJ>PTC\footnote{O Partido da Juventude (PJ), fundado em 1985, mudou de nome
em 1989 para Partido da Reconstrução Nacional (PRN). Em 2001 passou a chamar-se
Partido Trabalhista Cristão (PTC)} & PJ>PRN>PTC & PTC & Partido Trabalhista Cristão \\
  PTN & PTN & PTN & Partido Trabalhista Nacional \\
  PV & PV & PV & Partido Verde \\
  SD & SD & SD & Solidariedade \\
\end{tabular}
\end{table}




\chapter{A saída do PSB da coalizão em Dilma I}\label{cap:analise-saida-psb}

\section{Introdução}

O PSB esteve na base de apoio do governo do PT desde que o partido conquistou a
Presidência da República com Lula em 2003. Dez anos depois, em 2013, penúltimo
ano do primeiro governo de Dilma, o PSB anuncia a candidatura de Eduardo Campos
para a presidência, rompendo a aliança com o PT.

Para entender a repercussão dessa mudança na Câmara dos Deputados, analisaremos
as votações nominais ocorridas naquela legislatura (54\textordfeminine, que
durou de 2011 até 2015) seguindo a metodologia proposta por Keith Poole em
\emph{Spatial models of parliamentary voting} \cite{Poole2005}.
% Não estamos entendendo a repercussão da mudança na CD como um todo, mas sim
% só no comportamento de votação dos parlamentares.

\section{Metodologia}{\label{sec:analise-saida-psb:metodologia}

Para entender a influência da saída do PSB no comportamento de votação dos
parlamentares, precisamos de uma forma de comparar o comportamento deles antes
e depois dessa mudança. Este é um problema complicado, estudado por diversos
autores (CITATION NEEDED). Nesta análise seguimos a metodologia de
\cite{Poole2005}.

Considerando um parlamentar por vez, o substituímos por dois parlamentares
``virtuais'' na tabela de votações, um com os votos antes da mudança e outro
com os votos depois dela, e executamos o algoritmo W-NOMINATE nessa nova
tabela. Esse procedimento é repetido para cada parlamentar separadamente,
enquanto mantém todos os outros sem modificações. Ao final, teremos dois pontos
para cada parlamentar: um para antes e outro para depois da mudança. A
diferença entre esses pontos representa o nível da mudança de comportamento no
período de análise. 

Por exemplo, considere a tabela \ref{table:exemplo-mudanca-de-comportamento}
contendo 3 parlamentares e 4 votações. Para definir qual foi a mudança de
comportamento de Samara entre as votações 2 e 3, dividimos seus votos em dois
parlamentares ``virtuais'', Samara 1 e Samara 2 (ver tabela
\ref{table:exemplo-parlamentar-virtual}), e executamos o algoritmo W-NOMINATE.
Guardamos os resultados de Samara e repetimos os mesmos passos para Pedro e
depois Maria. Ao final, teremos dois pontos para cada parlamentar: um relativo
a sua posição antes, e outro depois da mudança (ver figura
\ref{fig:exemplo-mudanca-de-comportamento}). A distância desses dois pontos é
uma medida da intensidade da mudança do parlamentar.  Comparando as distâncias
de cada parlamentar, podemos determinar se a mudança é significante ou não.

% TODO: Falar sobre o parametric bootstrap
% TODO: Falar sobre os problemas em comparar esses valores.
% \begin{quote}
%   ``(...) all matrices have the same number of nonmissing entries, and they
%   differ only in the fact that each has a different senator divided into two
%   records. Consequently, I believe it's safe to capture the 72 distances with
%   each other, because the differences between the configurations will be
%   trivial.''\cite{Poole2005}
% \end{quote}

\begin{table}
  \begin{minipage}{\textwidth}
    \centering
    \begin{tabular}{ l l l l l }
      nome & votação1 & votação2 & votação3 & votação4 \\
      \hline
      Samara & Sim & Sim & Não & Não \\
      Pedro & Sim & Sim & Sim & Sim \\
      Maria & Não & Não & Não & Não \\
    \end{tabular}
    \caption{Dados originais de votação}
    \label{table:exemplo-mudanca-de-comportamento}
  \end{minipage}
  \begin{minipage}{\textwidth}
    \centering
    \begin{tabular}{ l l l l l }
      nome & votação1 & votação2 & votação3 & votação4 \\
      \hline
      Samara 1 & Sim & Sim & & \\
      Samara 2 & & & Não & Não \\
      Pedro & Sim & Sim & Sim & Sim \\
      Maria & Não & Não & Não & Não \\
    \end{tabular}
    \caption{Dados de votação com Romário dividido no meio}
    \label{table:exemplo-parlamentar-virtual}
  \end{minipage}
\end{table}

\begin{knitrout}
\definecolor{shadecolor}{rgb}{0.969, 0.969, 0.969}\color{fgcolor}\begin{figure}
\includegraphics[width=\maxwidth]{figure/exemplo-mudanca-de-comportamento-1} \caption[Posições dos parlamentares antes e depois da data em análise]{Posições dos parlamentares antes e depois da data em análise}\label{fig:exemplo-mudanca-de-comportamento}
\end{figure}


\end{knitrout}

% Pseudocódigo
%
% \begin{enumerate}
%   \item Para cada parlamentar, faça:
%   \begin{enumerate}
%     \item Crie dois parlamentares ``virtuais'', um com os votos do parlamentar
%       antes da mudança, e o outro com os votos depois dela;
%     \item Execute o W-NOMINATE na matriz completa de votações com todos os
%       parlamentares e o parlamentar em análise substituído pelos parlamentares
%       ``virtuais'';
%     \item Guarde o resultado;
%   \end{enumerate}
%   \item Compara a mudança de cada parlamentar de interesse com todos os outros,
%     para descobrir se ela (a mudança) foi significativa.
% \end{enumerate}

\section{Extração dos dados}



Os dados das votações nominais foram extraídos a partir da API disponibilizada
no site da Câmara dos Deputados. Eles compreendem 131.552 votos
proferidos por 644 parlamentares em 432
votações. A figura \ref{fig:votacoes-por-mes} mostra a distribuição das
votações no período e a figura \ref{fig:deputados-por-partido} mostra o número
de deputados federais por partido. Só consideramos votos \verb|Sim| ou
\verb|Não|.  

\begin{knitrout}
\definecolor{shadecolor}{rgb}{0.969, 0.969, 0.969}\color{fgcolor}\begin{figure}
\includegraphics[width=\maxwidth]{figure/votacoes-por-mes-1} \caption[Distribuição de votações por mês na 54\textordfeminine{} legislatura]{Distribuição de votações por mês na 54\textordfeminine{} legislatura}\label{fig:votacoes-por-mes}
\end{figure}


\end{knitrout}

\begin{knitrout}
\definecolor{shadecolor}{rgb}{0.969, 0.969, 0.969}\color{fgcolor}\begin{figure}
\includegraphics[width=\maxwidth]{figure/deputados-por-partido-1} \caption[Distribuição de Deputados Federais por partido na 54\textordfeminine{} legislatura]{Distribuição de Deputados Federais por partido na 54\textordfeminine{} legislatura}\label{fig:deputados-por-partido}
\end{figure}


\end{knitrout}

\section{Análise}



Seguindo a metodologia descrita na sessão
\ref{sec:analise-saida-psb:metodologia}, dividimos a
54\textordfeminine{} legislatura na votação mais próxima da data de interesse,
no nosso caso, a data em que o PSB deixa a coalizão do governo. De acordo com
os dados disponíveis no banco de dados do CEBRAP, isso ocorreu em 03/10/2013. A
primeira votação depois dessa data ocorreu em
08/10/2013 (identificador
367), relacionada ao MPV 621/2013. Houve
303 votações antes da ruptura e
133 votações após. Dos 643
parlamentares que votaram ao menos uma vez nesse período,
470 passaram pelo nosso filtro de
participação em ao menos 20 votações cuja minoria teve 2,5\% ou mais dos votos.

A figura \ref{fig:boxplot-diff-coalizao} mostra a distribuição da diferença
entre as posições dos parlamentares antes e depois da saída do PSB da base do
governo, agrupado pelos que entraram ou saíram da coalizão e os que mantiveram
sua posição. Nela podemos ver que, no geral, quem entra na coalizão se torna
mais governista, quem sai se torna mais oposicionista, e os outros mantém sua
posição, mas existem diversos \emph{outliers}.

\begin{knitrout}
\definecolor{shadecolor}{rgb}{0.969, 0.969, 0.969}\color{fgcolor}\begin{figure}
\includegraphics[width=\maxwidth]{figure/boxplot-diff-coalizao-1} \caption[Distribuição do nível de mudança da posição dos deputados federais antes e depois da saída do PSB da coalizão do governo]{Distribuição do nível de mudança da posição dos deputados federais antes e depois da saída do PSB da coalizão do governo}\label{fig:boxplot-diff-coalizao}
\end{figure}


\end{knitrout}

Na tabela \ref{table:top-10-diffs} estão os dez parlamentares com maior mudança
de posição em valores absolutos. Desses, analisemos os que não entraram nem
saíram da coalizão: 

\begin{description}
\item[Dudimar Paxiuba] Apesar de continuar fora da coalizão, ele saiu do PSDB
para o PROS em 2013, o que explicaria essa aproximação ao
governo\footnote{\url{http://www2.camara.leg.br/deputados/pesquisa/layouts_deputados_biografia?pk=193033&tipo=0}}.
\item[Rebecca Garcia] Não consegui explicar. A única coisa que encontrei foi
que ela ia se candidatar a prefeitura de Manaus em 2012, mas desistiu porque
vazou um áudio de uma conversa dela com seu amante na época, o ex-vereador Ari
Moutinho. Não me parece ter relação.
\item[Vicente Candido]Também não encontrei uma razão.
\item[Dr. Grilo]Saiu do PSL para o Solidariedade, que pode explicar o
afastamento do governo.
\end{description}

\begin{table}
\centering
\begin{knitrout}
\definecolor{shadecolor}{rgb}{0.969, 0.969, 0.969}\color{fgcolor}
\begin{tabular}{l|l|l|r|r|r}
\hline
Nome & Partido & Coalizão & Antes & Depois & Diferença\\
\hline
Wladimir Costa & PMDB & saiu & 0.32 & -0.60 & -0.92\\
\hline
Dudimar Paxiuba & PROS & nao\_mudou & -0.71 & 0.17 & 0.87\\
\hline
Luiz Nishimori & PL+PRONA>PR & entrou & -0.86 & -0.08 & 0.77\\
\hline
Domingos Dutra & PT & saiu & 0.52 & -0.16 & -0.68\\
\hline
Romário & PSB & saiu & 0.28 & -0.37 & -0.65\\
\hline
Genecias Noronha & PMDB & saiu & -0.01 & -0.64 & -0.63\\
\hline
Rebecca Garcia & PDS>PP & nao\_mudou & 0.39 & -0.23 & -0.62\\
\hline
Antonio Balhmann & PROS & saiu & 0.08 & 0.69 & 0.61\\
\hline
Vicente Candido & PT & nao\_mudou & 0.99 & 0.39 & -0.60\\
\hline
Dr. Grilo & PSL & nao\_mudou & 0.05 & -0.52 & -0.57\\
\hline
\end{tabular}


\end{knitrout}
\caption{Os dez parlamentares com maior diferença absoluta entre suas posições
antes e depois da saída do PSB da base do governo.}
\label{table:top-10-diffs}
\end{table}

\subsection{Os deputados do PSB mudaram de comportamento?}



De acordo com a figura \ref{fig:mudanca-psb}, no geral os deputados do PSB se
afastaram do governo, enquanto a maioria dos parlamentares dos outros partidos
mantiveram suas posições. Há um \emph{outlier} nos deputados do PSB que, ao
contrário da maioria, se aproximou do governo: Alexandre Toledo, que
passou de -0,61 para -0,10 (diferença
de 0,51). Analisando este caso, a razão se torna clara: ele
era do PSDB e passou para o PSB em 2013. Apesar do PSB ter se tornado oposição,
passando de uma posição mediana -0,01
para -0,33, o PSDB é um opositor mais
ferrenho ao governo, com posição mediana -0,87.

Com base nesses dados, podemos concluir que o que já esperávamos: o PSB se
afastou do governo.

\begin{knitrout}
\definecolor{shadecolor}{rgb}{0.969, 0.969, 0.969}\color{fgcolor}\begin{figure}
\includegraphics[width=\maxwidth]{figure/mudanca-psb-1} \caption[Distribuição do nível de mudança da posição dos deputados federais antes e depois da saída do PSB da coalizão do governo]{Distribuição do nível de mudança da posição dos deputados federais antes e depois da saída do PSB da coalizão do governo}\label{fig:mudanca-psb}
\end{figure}


\end{knitrout}

\section{Modelagem}

Nosso objetivo agora é determinar se, baseado somente nas posições dos
deputados federais de antes e depois da saída do PSB da coalizão, conseguimos
inferir se um parlamentar mudou de lado (se era governo, se tornou oposição, ou
vice-versa). Para isso, treinaremos um modelo \emph{Random Forest}.

O primeiro passo é alterar a coluna ``Coalizão'' para somente dois valores: sim
ou não. Nesse momento, nosso objetivo é determinar se o parlamentar mudou de
lado, e não pra que lado ele foi\footnote{Sabendo se o parlamentar mudou de
lado e qual era sua posição anterior, podemos inferir qual é sua nova posição
(e.x.: se ele era oposição e mudou de lado, agora virou governo).}. Também
retiraremos informações como nome e partido e parlamentares cuja posição não
conseguimos estimar pois não houve votos o suficiente, e adicionamos o
desvio-padrão de cada valor\footnote{Calculados usando parametric bootstrap.
TODO: Explicar melhor}. No final, teremos uma tabela como a
\ref{table:clean-data}.

\begin{table}
\centering
\begin{knitrout}
\definecolor{shadecolor}{rgb}{0.969, 0.969, 0.969}\color{fgcolor}
\begin{tabular}{r|r|r|r|r|l}
\hline
Antes & Antes SD & Depois & Depois SD & Diferença & Mudou de posição\\
\hline
0.32 & 0.17 & -0.60 & 0.16 & -0.92 & Sim\\
\hline
-0.71 & 0.12 & 0.17 & 0.08 & 0.87 & Não\\
\hline
-0.86 & 0.10 & -0.08 & 0.19 & 0.77 & Sim\\
\hline
0.52 & 0.13 & -0.16 & 0.13 & -0.68 & Sim\\
\hline
0.28 & 0.10 & -0.37 & 0.20 & -0.65 & Sim\\
\hline
-0.01 & 0.12 & -0.64 & 0.15 & -0.63 & Sim\\
\hline
\end{tabular}


\end{knitrout}
\caption{Os dez parlamentares com maior diferença absoluta entre suas posições
antes e depois da saída do PSB da base do governo.}
\label{table:clean-data}
\end{table}

Com os dados nesse formato, agora precisamos dividí-lo em dois grupos: um para
treino e um para teste. Precisamos fazer isso para que as estimativas de
precisão do nosso modelo sejam confiáveis. Escolhemos manter 80\% para treino e
20\% para testes. Também usamos o \emph{10-fold stratified cross validation}.

\begin{knitrout}
\definecolor{shadecolor}{rgb}{0.969, 0.969, 0.969}\color{fgcolor}\begin{kframe}
\begin{verbatim}
## note: only 30 unique complexity parameters in default grid. Truncating the grid to 30 .
\end{verbatim}
\end{kframe}
\end{knitrout}

O modelo foi treinado com todos os atributos na tabela \ref{table:clean-data} e
a interação entre eles. Analisamos diversos números de preditores e, como pode
ser visto na figura \ref{fig:roc-vs-mtry}, o melhor modelo com relação ao área
sob a curva ROC usou 20 preditores e chegou a área de
0,86 com desvio-padrão de 0,06.

\begin{knitrout}
\definecolor{shadecolor}{rgb}{0.969, 0.969, 0.969}\color{fgcolor}\begin{figure}
\includegraphics[width=\maxwidth]{figure/roc-vs-mtry-1} \caption[ROC como função do número de preditores]{ROC como função do número de preditores.}\label{fig:roc-vs-mtry}
\end{figure}


\end{knitrout}

Usando os dados de teste, podemos avaliar nosso modelo. A figura \ref{fig:roc}
mostra a curva ROC. Ela mostra a sensibilidade e especificidade do nosso modelo
para diversos pontos de corte. O valor marcado é o ponto ``ideal'', definido
como o ponto mais próximo do canto superior esquerdo do gráfico (sensibilidade
e especificidade iguais a 1). Na tabela \ref{table:roc} podemos ver todos
máximos locais dos pontos de corte e as respectivas sensibilidades e
especificidades, com mínimos e máximos dentro de margem de confiança de
95\%\footnote{Calculado a partir de 2.000 stratified bootstrap replications.
TODO: explicar melhor}.

\begin{knitrout}
\definecolor{shadecolor}{rgb}{0.969, 0.969, 0.969}\color{fgcolor}\begin{figure}
\includegraphics[width=\maxwidth]{figure/roc-1} \caption[Curva ROC com ponto ``ideal'' marcado]{Curva ROC com ponto ``ideal'' marcado.}\label{fig:roc}
\end{figure}


\end{knitrout}

\begin{table}
\centering
\begin{knitrout}
\definecolor{shadecolor}{rgb}{0.969, 0.969, 0.969}\color{fgcolor}
\begin{tabular}{l|r|r|r|r|r|r}
\hline
cutpoint & sens.low & sens.median & sens.high & spec.low & spec.median & spec.high\\
\hline
0.755 & 0.00 & 0.20 & 0.40 & 0.91 & 0.96 & 1.00\\
\hline
0.592 & 0.07 & 0.27 & 0.53 & 0.87 & 0.94 & 0.99\\
\hline
0.449 & 0.13 & 0.33 & 0.60 & 0.83 & 0.90 & 0.96\\
\hline
0.362 & 0.20 & 0.47 & 0.73 & 0.77 & 0.85 & 0.91\\
\hline
0.321 & 0.27 & 0.53 & 0.80 & 0.76 & 0.83 & 0.91\\
\hline
0.281 & 0.33 & 0.60 & 0.87 & 0.73 & 0.82 & 0.90\\
\hline
0.256 & 0.40 & 0.67 & 0.87 & 0.72 & 0.81 & 0.88\\
\hline
0.237 & 0.53 & 0.73 & 0.93 & 0.68 & 0.78 & 0.87\\
\hline
0.165 & 0.60 & 0.80 & 1.00 & 0.62 & 0.72 & 0.81\\
\hline
0.143 & 0.67 & 0.87 & 1.00 & 0.55 & 0.65 & 0.76\\
\hline
0.093 & 0.80 & 0.93 & 1.00 & 0.47 & 0.59 & 0.69\\
\hline
0.057 & 1.00 & 1.00 & 1.00 & 0.41 & 0.53 & 0.64\\
\hline
\end{tabular}


\end{knitrout}
\caption{Especificidade e sensibilidade nos máximos locais da curva ROC}
\label{table:roc}
\end{table}

\chapter{Versões dos softwares utilizados}

\begin{itemize}\raggedright
  \item R version 3.2.1 (2015-06-18), \verb|x86_64-pc-linux-gnu|
  \item Locale: \verb|LC_CTYPE=en_US.UTF-8|, \verb|LC_NUMERIC=C|, \verb|LC_TIME=pt_BR.UTF-8|, \verb|LC_COLLATE=en_US.UTF-8|, \verb|LC_MONETARY=pt_BR.UTF-8|, \verb|LC_MESSAGES=en_US.UTF-8|, \verb|LC_PAPER=pt_BR.UTF-8|, \verb|LC_NAME=C|, \verb|LC_ADDRESS=C|, \verb|LC_TELEPHONE=C|, \verb|LC_MEASUREMENT=pt_BR.UTF-8|, \verb|LC_IDENTIFICATION=C|
  \item Base packages: base, datasets, graphics, grDevices,
    methods, parallel, stats, utils
  \item Other packages: caret~6.0-47, data.table~1.9.4,
    doParallel~1.0.8, foreach~1.4.2, ggplot2~1.0.1,
    iterators~1.0.7, knitr~1.9, lattice~0.20-31, plyr~1.8.2,
    pROC~1.8, randomForest~4.6-10
  \item Loaded via a namespace (and not attached):
    BradleyTerry2~1.0-6, brglm~0.5-9, car~2.0-25, chron~2.3-45,
    codetools~0.2-11, colorspace~1.2-4, digest~0.6.8,
    evaluate~0.5.5, formatR~1.2, grid~3.2.1, gtable~0.1.2,
    gtools~3.4.2, highr~0.5, labeling~0.3, lme4~1.1-7,
    MASS~7.3-42, Matrix~1.2-0, mgcv~1.8-6, minqa~1.2.4,
    munsell~0.4.2, nlme~3.1-121, nloptr~1.0.4, nnet~7.3-10,
    pbkrtest~0.4-2, proto~0.3-10, quantreg~5.11, Rcpp~0.11.4,
    reshape2~1.4.1, scales~0.2.4, SparseM~1.6, splines~3.2.1,
    stringr~0.6.2, tools~3.2.1
\end{itemize}


%\input{apendice-b}
%\chapter{Apêndice C}}\label{apendice-c}

%\chapter{Apêndice D}}\label{apendice-d}

%%%%%%%%%%%%%%%%%%%%%%%%%%%%%%%%%%%%%%%%%%%%%%%%%%%%%%%%%%%%%%%%%%%%%%%%%%%%%%%%

\end{document}
