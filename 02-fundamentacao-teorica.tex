\chapter{Fundamentação Teórica}\label{cap:fundamentacao}

\section{Índice de Rice}

O índice de Rice é uma métrica para o cálculo da coesão\footnote{Neste trabalho
entendemos por coesão o grau de unidade do partido nas decisões legislativas
tomadas em votações nominais, independente se isso ocorre porque seus membros
têm uma ideologia semelhante, ou porque seus líderes conseguem discipliná-los.}
entre grupos em uma votação. Desde sua introdução em \cite{Rice1924}, ele vem
sendo amplamente utilizado em estudos legislativos por ser intuitivo, flexível
e facilmente interpretável \cite{Neiva2011}.

Para uma votação $i$, o seu cálculo se dá pela diferença, em números absolutos,
entre a quantidade de votos \verb|Sim| e de votos \verb|Não|, dividida pelo
\verb|Total| de votos, como pode ser visto na fórmula \ref{eq:rice}.

\begin{equation}\label{eq:rice}
  IR_i = \frac{|Sim - N\tilde{a}o|}{Total}
\end{equation}

O índice varia de zero, quando metade votou \verb|Sim| e metade votou
\verb|Não|, até um, quando todos votaram da mesma forma. Para analisar mais de
uma votação, calculamos a média aritmética dos índices de Rice das votações no
período, como mostra a fórmula \ref{eq:rice-media}.

\begin{equation}\label{eq:rice-media}
  \sum_{i=1}^{n} \frac{IR_i}{n}
\end{equation}

% TODO: Escrever sobre o problema de só considerar votos Sim ou Não e como
% resolvemos nesse trabalho (talvez só explicitar o problema, pra manter
% fundamentação teórica só como fundamentação teórica).
% Figueiredo/Limongi 1995 só consideram votos Sim e Não, então talvez posso
% dizer que estou seguindo o exemplo deles :)
% ---
% O primeiro problema enfrentado em usar o índice de Rice no contexto brasileiro
% é que ele considera só os indivíduos que votaram, ignorando os ausentes, os que
% se abstiveram e os que obstruíram.

Um problema descrito por Desposato (2005) é que índices de coesão em geral
(o de Rice em particular) são fortemente enviesados. Partidos pequenos têm uma
coesão esperada maior do que partidos grandes, o que invalidaria comparações
entre grupos de tamanhos diferentes, pois diferenças na coesão podem ser
resultado de uma característica do índice usado, e não no comportamento dos
parlamentares. Este problema é especialmente grave no Brasil por existir uma
grande variação no tamanho dos partidos.\nocite{Desposato2005}

Ele propôs uma solução simples: diminuir o tamanho dos partidos.
Especificamente, ele sugere trocar o índice de coesão $C_{ij}$ de cada
partido por $E(C_{ijr})$, que é a coesão esperada de uma amostra de $r$ votos
tomados sem reposição do partido $i$ na votação $j$. O tamanho da amostra $r$
pode variar entre 2 e o tamanho do menor partido, mas ele sugere escolhermos $r
= 2$, que funciona em todos os casos.\nocite{Desposato2005}

Com essa escolha, a interpretação de $E(C_{ij2})$ é simples: é a probabilidade
de que dois membros do partido $i$ escolhidos aleatoriamente votaram juntos na
votação $j$. O cálculo também fica simplificado, sendo feito pela fórmula
\ref{eq:rice-adj}, onde $Y$ é o número de votos ``sim'' e $R$ é o tamanho do
partido.\nocite{Desposato2005}

\begin{equation}\label{eq:rice-adj}
  E(C_{ij2}|Y, R) = \frac{Y(Y - 1) + (R - Y)(R - Y - 1)}{R(R - 1)}
\end{equation}

Caso o índice $C$ já seja conhecido, a fórmula pode ser reduzida a
\ref{eq:adj-known-metric}.\nocite{Desposato2005}

\begin{equation}\label{eq:adj-known-metric}
  E(C_2|Y, R) = \frac{RC^2 + R + 2}{2(R - 1)}
\end{equation}

\subsection{Calculando coesão entre partidos}

Encontramos um problema ao calcular a coesão entre partidos. Caso simplesmente
calculássemos o índice de Rice entre o conjunto de parlamentares dos partidos
que queremos analisar, estaríamos mostrando um resultado errado caso todos os
partidos não tivessem o mesmo número de parlamentares votando em cada votação.

Por exemplo, considere que queremos calcular a coesão entre o PT e o PSOL na
Câmara dos Deputados na 54a legislatura. Nesse período, o PT contava com XXX
deputados federais, enquanto o PSOL contava com 3. Se simplesmente calcularmos
a coesão considerando o conjunto de deputados do PT com o do PSOL, o índice
seria muito influenciado pela diferença de parlamentares.

Para resolver esse problema, não consideramos o voto de cada parlamentar de
cada partido individualmente, mas sim o voto de um ``parlamentar médio'' do
partido. O voto do parlamentar médio é definido como o voto da maioria do
partido. Dessa forma, conseguimos comparar partidos independente de seus
tamanhos.

Poderíamos ter definido o ``parlamentar médio'' como a indicação do líder do
partido, mas neste caso teríamos que desconsiderar as votações em que o líder
libera sua bancada e não conseguiríamos calcular para partidos cujo líder não
pode indicar voto (de acordo como o Regimento Interno). Dado essas
desvantagens, e o fato que há poucas diferenças entre a indicação do líder e o
voto da maioria \cite{Figueiredo1995}, preferimos usar o voto da maioria.

\section{Processo legislativo}

Como são criadas as Leis no Brasil? Essa seção pode ser gigantesca, então acho melhor focar no essencial e pontuar as características mais importantes para este trabalho, por exemplo o que são destaques, tipos de votação, tipos de votos, etc.



(Alexandre) Acho que senado e câmara podem ser subseções desta seção.
Essas subseções não precisam ser muito longas. Devem descrevas basicamente as competências e atribuições de cada casa.

Acho que a justificativa da escolha da Câmara pode aparecer mais tarde, na seção do estudo de caso.  
Assim a seção de fundamentação teórica fica sendo realmente apenas de fundamentação teórica, sem efeitos colaterais.

\subsection{Senado Federal}


\subsection{Câmara dos Deputados}

Escrever sobre as competências da Câmara dos Deputados e justificar porque usamos ela no estudo de caso (talvez essa justificativa caiba em outro lugar?)


\section{Coesão parlamentar}

Definir o que é coesão e disciplina, e explicar algumas métricas (ou só o índice de Rice)


\section{Ciência de dados}

Texto bem geral sobre data science. Pode ser interessante falar sobre algum algoritmo específico que eu vá usar, ou talvez seja melhor explicá-los no miolo da dissertação.

\section{Considerações Finais}





