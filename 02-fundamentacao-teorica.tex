\chapter{Fundamentação Teórica}\label{cap:fundamentacao}

\section{Processo legislativo}

Como são criadas as Leis no Brasil? Essa seção pode ser gigantesca, então acho melhor focar no essencial e pontuar as características mais importantes para este trabalho, por exemplo o que são destaques, tipos de votação, tipos de votos, etc.



(Alexandre) Acho que senado e câmara podem ser subseções desta seção.
Essas subseções não precisam ser muito longas. Devem descrevas basicamente as competências e atribuições de cada casa.

Acho que a justificativa da escolha da Câmara pode aparecer mais tarde, na seção do estudo de caso.  
Assim a seção de fundamentação teórica fica sendo realmente apenas de fundamentação teórica, sem efeitos colaterais.

\subsection{Senado Federal}


\subsection{Câmara dos Deputados}

Escrever sobre as competências da Câmara dos Deputados e justificar porque usamos ela no estudo de caso (talvez essa justificativa caiba em outro lugar?)


\section{Coesão parlamentar}

Definir o que é coesão e disciplina, e explicar algumas métricas (ou só o índice de Rice)


\section{Ciência de dados}

Texto bem geral sobre data science. Pode ser interessante falar sobre algum algoritmo específico que eu vá usar, ou talvez seja melhor explicá-los no miolo da dissertação.

\section{Considerações Finais}





