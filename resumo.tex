No Brasil, existem ferramentas para o acompanhamento do comportamento dos
parlamentares em votações nominais, tais como o Basômetro do jornal O Estado de
São Paulo e o Radar Parlamentar. Essas ferramentas são usadas para análises
tanto por jornalistas quanto por cientistas políticos.

Apesar de serem ótimas ferramentas de análise, sua utilidade para monitoramento
é limitada por exigir um acompanhamento manual, o que se torna muito trabalhoso
quando consideramos o volume de dados. Somente na Câmara dos Deputados, 513
parlamentares participam em média de mais de 400 votações nominais por
legislatura. É possível diminuir a quantidade de dados analisando os partidos
como um todo, mas em contrapartida perdemos a capacidade de detectar
movimentações de indivíduos ou grupos intrapartidários como as bancadas.

% FIXME: Estava falando de análise de comportamento em geral, e aqui já pulei
% para mudanças de posicionamento. Falta algo para ligar os dois.

Para diminuir esse problema, desenvolvi neste trabalho um modelo estatístico
que prevê quando um parlamentar vai entrar ou sair da coalizão governamental
através de estimativas de pontos ideais usando o W-NOMINATE. Ele pode ser usado
individualmente ou integrado a ferramentas como o Basômetro, oferecendo um
filtro para os pesquisadores encontrarem os parlamentares que mudaram mais
significativamente de comportamento.

O universo de estudo é composto pelos parlamentares da Câmara dos Deputados no
período da 50\textordfeminine{} até a 54\textordfeminine{} legislaturas,
iniciando no primeiro mandato de Fernando Henrique Cardoso em 1995 até o final
do primeiro mandato de Dilma Rousseff em 2015.
\\
\\
\textbf{Palavras-chave:} Análise legislativa, Ciência política, Ciência de
dados, Modelos preditivos, Aprendizado de máquina.

% FIXME: Faltou falar um pouco dos resultados.
