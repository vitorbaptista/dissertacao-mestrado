\chapter{Trabalhos Relacionados}\label{cap:relacionados}


\section{Coesão parlamentar}

A coesão entre parlamentares é estudada há décadas pela Ciência Política. Em 1924, Stewart Rice propôs a primeira métrica para medir coesão: o Índice de Rice \cite{Rice1924}. Depois dele, vários outros pesquisadores proporam novas métricas, como XXX, YYY, ZZZ. Apesar disso, o IR continua sendo muito usado.

No Brasil, podemos citar \citeonline{Figueiredo1995} que analisa o padrão de
votação dos parlamentares da Câmara dos Deputados entre 1989 e 1994;
\citeonline{Neto1997}, que analisa a mesma casa mas entre 1946 e 1964, e;
\citeonline{Neiva2011}, que analisa votações do Senado entre 1989 a 2009. Todos utilizam o Índice de Rice.

\subsection{Discussão}

Explicar que apesar de existirem várias métricas de coesão, a mais usada ainda é a de Rice, mas que usarei ela com algumas modificações como normalizá-lo para manter o mesmo índice independente do número de parlamentares considerados no cálculo. Existe um artigo que explica isso (só preciso encontrá-lo)

Explicar também que, apesar das limitações de analisar a coesão através das votações nominais (muito limitado, ignora o trabalho dos bastidores), ela é uma forma simples e direta de analisar um grande volume de dados, e ainda é muito usada.

\section{Ciência de dados}

Listarei trabalhos sobre ciência de dados, especialmente métodos de detecção de anomalias. Acho que não cabe aqui, mas a parte de sistemas de monitoramento de servidores e a sacada de usá-los nessa outra área é interessante. Também a criação de um pipeline, possivelmente usando o Luigi (\url{https://github.com/spotify/luigi}).


(Alexandre) Essa parte sobre a parte de sistemas de monitoramento entra em outro capítulo e concordo que deve ser mencionada sim na dissertação. Acho que isso pode aparecer no próximo capítulo.


\subsection{Discussão}

\section{Considerações Finais}

Esse trabalho não trás muitas novidades especificamente nas áreas relacionadas, mas a contribuição está na junção dessas ferramentas já existentes. Nas minhas pesquisas, só encontrei o a startup \url{www.fiscalnote.com} que faz um trabalho parecido, mas tenho certeza que existem outras empresas, ou sistemas internos. O que não encontrei é um monitorador legislativo que não foque em uma área ou projeto de Lei específico, mas busque padrões gerais. O que encontrei mais parecido é o monitoramento de servidores, que segue o padrão de analisar o maior número possível de dados buscando anomalias e padrões.


(Alexandre) Então esse deve ser o principal diferencial do seu trabalho em relação aos demais. Essa diferença deve ser reforçada ao apresentar os trabalhos relacionados.

Um negócio que fica muito legal nesta seção é criar uma tabela com um conjunto de características presentes e desejadas e marcar quais trabalhos relacionados atendem a cada uma dessas características.

Assim fica mais fácil constatar onde está o grande diferencial do seu trabalho, que seria justamente ser uma solução com aplicação mais ampla e não focada em áreas específicas.
