\chapter{Trabalhos Relacionados}\label{cap:trabalhos-relacionados}


\section{Análise do comportamento parlamentar}
\label{cap:trabalhos-relacionados:analise-comportamento}

No final da década de 80 e na de 90, diversos autores viam o sistema político
brasileiro como sendo altamente instável, com um Poder Executivo fraco,
obrigado a negociar com cada parlamentar individualmente para executar seu
plano de governo. Os parlamentares, preocupados somente com seus interesses
individuais e regionalistas, seriam indisciplinados, levando a
ingovernabilidade do país
\cite{Abranches1988,Lamounier1994,Mainwaring2001,Ames2003}.

Contrários a esse diagnóstico, \citeonline{Limongi1995} descreveram um cenário
muito diferente. O Executivo, por ter domínio sobre a agenda legislativa,
seria capaz de executar seu plano de governo negociando diretamente com os partidos
políticos, que seriam capazes de disciplinar seus parlamentares a votar
conforme suas indicações. Esse diagnóstico foi feito a partir da análise das
votações nominais na Câmara dos Deputados no período de 1989 a 1994. Nessa
análise, os autores calcularam a coesão interna de cada partido, usando o
\gls{IR}\footnote{O \glsfirst{IR} varia de 0 a 100, e é calculado subtraindo a
proporção dos votos minoritários dos majoritários. Ele é 0 quando o partido
está dividido, com metade votando sim e a outra metade votando não, e é 100
quando todos os parlamentares votam da mesma forma \cite{Rice1924}.}. Todos os
partidos apresentaram níveis de coesão relativamente altos, com o menos coeso,
o PDS, atingindo um \gls{IR} 75,70, e o mais coeso, o PT, com \gls{IR} 95,96.

Essas duas correntes foram tão distintas que \citeonline{Power2010} as dividiu
em ``pessimistas'' e ``otimistas''.

Figueiredo e Limongi continuaram analisando dados relacionados ao mesmo tempo,
focando em períodos e características distintos, que confirmaram o prognóstico.
A coletânea desses artigos gerou o livro ``Executivo e Legislativo na nova
ordem constitucional'' \cite{Figueiredo2001}.

Também usando métricas de coesão, \citeonline{Cheibub2009} concluem que o
Congresso brasileiro, ao contrário do previsto pelos pessimistas, é altamente
centralizado, com partidos e seus líderes capazes de disciplinar seus
parlamentares, evitando que estes ajam somente em busca de benefícios para si
mesmo e seus redutos eleitorais.

Além de cientistas políticos, jornalistas também têm usado técnicas semelhantes
para análise das votações. O Grupo Estado, dono do jornal Estado de São Paulo
(o Estadão), lançou em maio de 2012 a ferramenta Basômetro\footnote{Disponível
em \url{http://estadaodados.com/basometro/}.}, um site interativo que permite a
análise de como os parlamentares, individualmente e agregados por partido,
votaram, com dados a partir do primeiro governo de Lula em 2003. Além da
análise de cada votação, o Basômetro também calcula a taxa de
governismo\footnote{A taxa de governismo é calculada como o percentual de vezes
que os parlamentares votaram de acordo com a posição do governo.} dos partidos
e parlamentares, mostrando suas flutuações em cada mês \cite{Estadao2012}.

\citeonline{Dantas2014} organizaram uma coletânea de artigos escritos baseados
em análises feitas com o Basômetro gerando o livro ``Análise política \&
jornalismo de dados''.

Apesar de serem de interpretação e cálculo mais simples, métricas de coesão têm
alguns problemas. Em primeiro lugar, pela distância entre valores não ser
uniforme (por exemplo, a distância entre 40\% e 50\% não é necessariamente
igual a entre 50\% e 60\%), eles só ordenam os parlamentares. Há também um
baixo valor de números possíveis. Em $p$ votações, há $p + 1$ valores
possíveis. Por exemplo, em 3 votações, os parlamentares só podem assumir os
valores 0\%, 33\%, 66\% e 100\%. Por fim, o índice é calculado em relação a
algum ponto específico. Por exemplo, a taxa de governismo é calculada com base
no voto do governo, que normalmente é definido pela indicação ou voto do líder
do governo. Assim, ninguém pode ser mais governista do que o próprio líder, o
que é um pressuposto que precisa ser testado
\cite{Poole2005,McCarty2011,Izumi2013}. Para evitar essas limitações, diversos
autores usam modelos espaciais de votação (ver seção
\ref{cap:fundamentacao:teoria-espacial-do-voto}).

No Brasil, o primeiro autor a usar esses modelos foi \citeonline{Leoni2002},
que analisou o posicionamento dos deputados federais entre 1991 e 1998 usando o
W-NOMINATE. \citeonline{Desposato2005b} usou o mesmo método para analisar o
efeito da mudança partidária no comportamento dos parlamentares.
\citeonline{Zucco2009} também usa o W-NOMINATE, dessa vez para entender os
fatores que influenciam o comportamento dos parlamentares, encontrando
que a ideologia não explica completamente seus comportamentos, e indicativos
que o presidente da República é um importante influenciador, especialmente
através da distribuição de cargos e recursos.

\citeonline{Freitas2012} analisam o significado da primeira dimensão, estimada
usando W-NOMINATE, na Câmara dos Deputados e no Senado Federal, concluindo que
ela está ligada ao conflito governo e oposição. \citeonline{Izumi2013} repete
essa análise no Senado Federal usando outro modelo, o \emph{Optimal
Classification}, chegando a mesma conclusão.

Além de técnicas específicas para gerar mapas espaciais de votação, alguns
autores usaram técnicas de redução de dimensionalidade mais gerais.
\citeonline{Trento2013} analisaram os dados da Câmara dos Deputados, Senado
Federal e Câmara Municipal de São Paulo usando \gls{ACP}\footnote{Do inglês,
\emph{Principal Component Analysis} (PCA).}, uma técnica estatística de redução
de dimensionalidade. Eles mapeiam um voto sim ou não nos valores 1 e 0, e
transformam o conjunto de votos dos parlamentares em pontos em um espaço
n-dimensional, onde cada dimensão representa uma votação. Como o número de
votações é alto, é difícil visualizar esses pontos. Assim, usam o \gls{ACP}
para colocar esses pontos em um espaço bidimensional, mantendo ao máximo suas
posições relativas. Dessa forma, quanto mais próximo estiverem dois
parlamentares (ou partidos), mais vezes eles votaram da mesma forma. O
resultado dessa pesquisa foi o Radar Parlamentar\footnote{Disponível em
\url{http://radarparlamentar.polignu.org}}, um site interativo no qual é
possível visualizar os diversos gráficos gerados, com as posições dos
legisladores e partidos a cada ano.

Usando uma técnica semelhante, a \gls{MCA}, \citeonline{Andrade2015} colocam os
deputados federais em um espaço bidimensional, mostrando aonde estaria o
presidente da Câmara dos Deputados, Eduardo Cunha (PMDB/RJ). Como ele não vota,
consideraram seu posicionamento como sendo igual ao do líder do PMDB na Câmara
no período. Eles também analisaram a similaridade dos deputados com o Eduardo
Cunha através do percentual de votos que deram iguais a ele, batizando essa
métrica de ``Cunhômetro''.

Apesar dessas técnicas estatísticas como a \gls{ACP} e \gls{MCA} serem
similares a técnicas como o W-NOMINATE, não encontramos trabalhos que validem
seu uso em dados de votação, o que pode dificultar a interpretação dos seus
resultados.

% FIXME: Citar o GovTrack.us e https://opengovdata.io/

\subsection{Análise da mudança de comportamento parlamentar}

O problema básico em comparar mudanças de comportamento ao longo do tempo é
distinguir alterações por causa de uma mudança de agenda das causadas por uma
real mudança das preferências dos parlamentares \cite{Bailey2007}. Em outras
palavras, se em um momento dois parlamentares votaram 90\% das vezes da mesma
forma, e em outro momento eles votaram 50\% das vezes, como definir se essa
mudança se deu porque eles mudaram seus posicionamentos, ou simplesmente porque
eles concordavam nas votações do primeiro momento, mas não nas do segundo?

Segundo \citeonline{Shor2010}, todos os esforços para resolver esse problema
usam ``pontes'', que são parlamentares que estiveram presentes em ambos
momentos e cujo posicionamento se assume ter se mantido estável. Podemos citar
como exemplos parlamentares que foram eleitos para mais de uma legislatura, ou
que mudaram de Casa (deputados federais que se tornam senadores). Projetos de
Lei também podem ser usados como ponte, caso eles tenham sido votados nas
instituições ou períodos de interesse.

No artigo, \citeonline{Shor2010} usam três tipos de pontes para colocar
parlamentares que serviram no nível estadual em ambas as casas\footnote{O
sistema legislativo norte-americano, ao contrário do brasileiro, é bicameral
tanto a nível federal quanto estadual (exceto o estado de Nebrasca, que só
possui um Senado estadual)}, federal em ambas as casas, e no tempo.
Parlamentares que serviram por múltiplas legislaturas tanto a nível estadual
quanto federal servem de ponte entre as legislaturas; legisladores que passam
da casa baixa para a casa alta (ou vice-versa) em nível estadual conectam as
respectivas casas, e; parlamentares que passam do nível estadual para atuar no
nível federal conectam o estado com o Congresso.

No Brasil, \citeonline{Leoni2002} analisa as posições dos deputados federais na
49\textordfeminine{} e 50\textordfeminine{} legislaturas não usando pontes, mas
sim a correlação de Pearson entre os pontos ideais. Ele encontrou correlações
altas, em torno de 0,80. Apesar disso, ele reconhece que uma limitação de seu
trabalho é que, nos períodos estudados, o governo sempre foi de direita, o que
poderia causar essa baixa mudança de pontos ideais. Além disso, como ele não
usou nenhuma técnica para diferenciar mudanças de comportamento de mudanças na
agenda legislativa (como as pontes), não é possível distinguir a razão desse
resultado com segurança.

\citeonline{Desposato2005b}, ao analisar o impacto das mudanças de partido no
comportamento dos parlamentares, usou o posicionamento dos legisladores que não
mudaram de partido como pontes. \citeonline{Izumi2013} faz uma pesquisa
semelhante, mas analisando a mudança de comportamento dos senadores que mudaram
de um partido dentro da coalizão governamental para um fora (ou vice-versa),
para entender o efeito da coalizão no comportamento dos parlamentares.

O Basômetro e o Radar Parlamentar não diferenciam mudanças de comportamento
reais das causadas por mudança da agenda legislativa
\cite{Estadao2012,Trento2013}.

Em geral, trabalhos que usam métricas de coesão interpretam as razões de
mudanças de comportamento usando métodos qualitativos, enquanto os que usam
pontos ideais usam métodos quantitativos.

\section{Aprendizagem de máquina na Ciência Política}
\label{ref:trabalhos-relacionados:data-science-polsci}

Os trabalhos encontrados que usam técnicas de aprendizagem de máquina no
âmbito da Ciência Política dividem-se em duas categorias: os que analisam
textos (como discursos) buscando entender seu conteúdo ou o posicionamento do
autor, e os que analisam projetos de lei tentando prever seus resultados.

Dos que fazem análise textual, \citeonline{Thomas2006} criaram um modelo que
classifica os discursos em debates sobre projetos de lei no congresso
estadunidense como sendo contrários ou favoráveis ao mesmo.
\citeonline{Quinn2006} criaram um modelo que extrai tópicos do texto de
discursos, validando-o nos dados do 105\textordmasculine{} ao
107\textordmasculine{} senado dos Estados Unidos. \citeonline{Yu2008}
determinam a filiação partidária dos parlamentares estadunidense a partir dos
seus discursos. \citeonline{Somasundaran2010} classificam autores de textos
publicados na Internet identificando seu posicionamento político a partir de
suas argumentações. Já \citeonline{Conover2011} fazem o mesmo com os usuários do
Twitter\footnote{Disponível em \url{http://twitter.com}.}, mas a partir da
frequência do uso de certas palavras.

Já na previsão do sucesso de projetos de lei, \citeonline{Gerrish2011} e
\citeonline{Goldblatt2012} desenvolveram modelos que acertaram mais de 90\% dos
resultados nos períodos estudados. \citeonline{Yano2012} fizeram uma análise
semelhante, mas focando em prever os projetos de lei que conseguirão passar das
discussões ocorridas em comissões e ser postos em votação.
\citeonline{Wang2012} preveem os resultados através de uma caminhada aleatória
em 3 grafos heterogêneos interconectados, sendo dois representando os
legisladores e um representando projetos de lei. Os legisladores estão
interligados caso tenham escrito ao menos um projeto juntos\footnote{Do inglês,
\emph{co-sponsorship}.}, e os projetos de lei interligados pela análise da
similaridade textual. Quando um legislador vota em um projeto de lei, um
vértice é criado entre seu nó e o do projeto. Ele criou dois grafos para os
legisladores como forma de criar dois tipos de ligação entre os legisladores e
os projetos de lei: uma para representar votos sim, e outra para votos não.

A empresa americana \citeonline{FiscalNote2015} desenvolve ferramentas para
monitoramento legislativo usando técnicas de ciência de dados. Uma delas, a
\emph{Prophecy}\footnote{Disponível em
\url{https://www.fiscalnote.com/prophecy}.} (do inglês, profecia), supostamente
prevê o resultado de projetos de lei com 94\% de acurácia. No Brasil, o
\gls{ELLO}, empresa ligada ao \gls{CEBRAP}, possui um produto similar.

\section{Considerações Finais}

Como visto na seção \ref{cap:trabalhos-relacionados:analise-comportamento}, o
foco da literatura encontrada é na análise e interpretação do comportamento dos
parlamentares, seja usando métricas de coesão ou modelos espaciais de votação.
Na seção \ref{ref:trabalhos-relacionados:data-science-polsci}, foram
apresentados trabalhos que aplicam técnicas de aprendizagem de máquina em temas
da ciência política, buscando descobrir o posicionamento de pessoas através dos
seus textos ou prever o resultado de votações.

Nesse levantamento bibliográfico, não foi encontrado nenhum autor que
desenvolva o objetivo deste trabalho: a detecção de mudanças de posicionamento
dos legisladores. O que parece ser mais próximo é a pesquisa feita pela empresa
americana \citeonline{FiscalNote2015} para o desenvolvimento do produto
\emph{Prophecy}, mas não foi possível analisar as semelhanças mais a fundo pois
o processo usado por eles não é divulgado.

Nesse sentido, considero a definição da metodologia para o desenvolvimento de
um modelo de detecção de mudança de posicionamento dos deputados federais
brasileiros como sendo o principal diferencial e contribuição deste trabalho.
