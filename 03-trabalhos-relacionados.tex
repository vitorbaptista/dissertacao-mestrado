\chapter{Trabalhos Relacionados}\label{cap:relacionados}


\section{Análise de mudança de comportamento de parlamentares}

O problema básico em comparar mudanças de comportamento ao longo do tempo é
distinguir alterações por causa de uma mudança de agenda das causadas por uma
real mudança das preferências dos parlamentares \cite{Bailey2007}. Em outras
palavras, se em um momento dois parlamentares \emph{A} e \emph{B} votaram 90\%
das vezes da mesma forma, e em outro momento eles votaram 50\% das vezes, como
definir se essa mudança se deu porque eles mudaram seus posicionamentos, ou
simplesmente porque eles concordavam nas votações do primeiro momento, mas não
nas do segundo?

Segundo \citeonline{Shor2010}, todos os esforços para resolver esse problema
usam ``pontes'', que são parlamentares que estiveram presentes em ambos
momentos e cujo posicionamento se assume ter se mantido estável. Podemos citar
como exemplos parlamentares que foram eleitos para mais de uma legislatura, ou
que mudaram de Casa (deputados federais que se tornam senadores). Projetos de
Lei também podem ser usados como ponte, caso eles tenham sido votados nas
instituições ou períodos de interesse.

No artigo, \citeonline{Shor2010} usa três tipos de pontes para colocar
parlamentares que serviram no nível estadual em ambas as casas\footnote{O
sistema legislativo norte-americano, ao contrário do brasileiro, é bicameral
tanto a nível federal quanto estadual (exceto o estado de Nebrasca, que só
possui um Senado estadual)}, federal em ambas as casas, e no tempo.
Parlamentares que serviram por múltiplas legislaturas tanto a nível estadual
quanto federal servem de ponte entre as legislaturas; legisladores que passam
da casa baixa para a casa alta (ou vice-versa) em nível estadual conectam as
respectivas casas, e; parlamentares que passam do nível estadual para atuar no
nível federal conectam o estado com o Congresso.

Os trabalhos se dividem em dois grandes grupos: os que usam medidas de coesão e
os que usam modelos espaciais de votação.

% Dividem-se em dois grupos: os que usam métodos espaciais e os que usam
% métodos "normais" (qual o nome?).
% Radar Parlamentar
% Basômetro
% TODO: O livro do Poole "Congress: A Political-Economic History of Roll Call
% Voting" parece importante demais pra estar de fora.

\citeonline{Desposato2005b} analisou os efeitos da mudança de partidos no
comportamento dos senadores e deputados federais brasileiros durante a
49\textordfeminine{} e 50\textordfeminine{} legislaturas. Ele usou o W-NOMINATE
para estimar os pontos ideais de cara par parlamentar-partido. Ou seja, se o
parlamentar \emph{A} mudar do PT para o DEM, ele terá dois pontos ideais: um
para cada período.

\citeonline{Leoni2002} analisou o comportamento dos partidos políticos na
Câmara dos Deputados entre 1991 e 1998 usando a correlação de Pearson. Ele
estava buscando entender se os deputados federais mantinham suas posições
relativas ao longo do tempo, e qual influência o presidente da República tinha
em alterar essas posições. Nesse período, ele não encontrou uma influência
estatisticamente relevante dos presidentes na composição da Câmara, mas há o
porém que todos os presidentes estudados eram de direita. Isso pode ter
influenciado seu resultado.

\section{Coesão parlamentar}

A coesão entre parlamentares é estudada há décadas pela Ciência Política. Em 1924, Stewart Rice propôs a primeira métrica para medir coesão: o Índice de Rice \cite{Rice1924}. Depois dele, vários outros pesquisadores proporam novas métricas, como XXX, YYY, ZZZ. Apesar disso, o IR continua sendo muito usado.

No Brasil, podemos citar \citeonline{Figueiredo1995} que analisa o padrão de
votação dos parlamentares da Câmara dos Deputados entre 1989 e 1994;
\citeonline{Neto1997}, que analisa a mesma casa mas entre 1946 e 1964, e;
\citeonline{Neiva2011}, que analisa votações do Senado entre 1989 a 2009. Todos utilizam o Índice de Rice.

\subsection{Discussão}

Explicar que apesar de existirem várias métricas de coesão, a mais usada ainda é a de Rice, mas que usarei ela com algumas modificações como normalizá-lo para manter o mesmo índice independente do número de parlamentares considerados no cálculo. Existe um artigo que explica isso (só preciso encontrá-lo)

Explicar também que, apesar das limitações de analisar a coesão através das votações nominais (muito limitado, ignora o trabalho dos bastidores), ela é uma forma simples e direta de analisar um grande volume de dados, e ainda é muito usada.

\section{Ciência de dados}

Listarei trabalhos sobre ciência de dados, especialmente métodos de detecção de anomalias. Acho que não cabe aqui, mas a parte de sistemas de monitoramento de servidores e a sacada de usá-los nessa outra área é interessante. Também a criação de um pipeline, possivelmente usando o Luigi (\url{https://github.com/spotify/luigi}).


(Alexandre) Essa parte sobre a parte de sistemas de monitoramento entra em outro capítulo e concordo que deve ser mencionada sim na dissertação. Acho que isso pode aparecer no próximo capítulo.


\subsection{Discussão}

\section{Considerações Finais}

Esse trabalho não trás muitas novidades especificamente nas áreas relacionadas, mas a contribuição está na junção dessas ferramentas já existentes. Nas minhas pesquisas, só encontrei o a startup \url{www.fiscalnote.com} que faz um trabalho parecido, mas tenho certeza que existem outras empresas, ou sistemas internos. O que não encontrei é um monitorador legislativo que não foque em uma área ou projeto de Lei específico, mas busque padrões gerais. O que encontrei mais parecido é o monitoramento de servidores, que segue o padrão de analisar o maior número possível de dados buscando anomalias e padrões.


(Alexandre) Então esse deve ser o principal diferencial do seu trabalho em relação aos demais. Essa diferença deve ser reforçada ao apresentar os trabalhos relacionados.

Um negócio que fica muito legal nesta seção é criar uma tabela com um conjunto de características presentes e desejadas e marcar quais trabalhos relacionados atendem a cada uma dessas características.

Assim fica mais fácil constatar onde está o grande diferencial do seu trabalho, que seria justamente ser uma solução com aplicação mais ampla e não focada em áreas específicas.
